\documentclass{article}
\usepackage{amsmath}
\usepackage{amssymb}
\usepackage{booktabs}
\usepackage{xintexpr}

\newcommand{\T}{1}
\newcommand{\F}{0}
\newcommand{\TF}[1]{\if1#1\T\else\F\fi}
\newcommand{\xintTF}[1]{\xintifboolexpr{#1}{\T}{\F}}

\newcommand{\logicrule}[2]{
\begin{array}{l}
#1 \\
\midrule
\therefore #2 \\
\end{array}
}

\newcommand{\inv}[1]{#1^{-1}}

\renewcommand{\d}[1]{\,\textnormal{d}#1}
\newcommand{\dd}[2]{\frac{\d{#1}}{\d{#2}}}
\newcommand{\ddd}[2]{\dfrac{\d{#1}}{\d{#2}}}

\DeclareMathOperator{\var}{Var}
\DeclareMathOperator{\E}{\mathcal{E}}

\setlength\parindent{0pt}
\setlength\parskip{1em}

\begin{document}

\section*{What? I don't know.}

He's talking about pullies and levers now. I guess.

Apparently expectation is a lever, and variation is a lever on
steroids. That's cool.

Right, we were looking at something to do with the handout on
variance.

\section*{Normal/Gaussian Distribution}

Let $\eta$ have density:

\[
\rho_{\eta} = \dfrac{1}{\sigma\sqrt{2\pi}} \text{exp}\left(\dfrac{-1}{2}\left(\dfrac{x-\mu}{\sigma}\right)^2\right)
\]

Luckily, the random variable has expectation:

\[
\E\eta =\mu
\]

and variance

\[
\var(\eta)=\sigma^2
\]

\end{document}
