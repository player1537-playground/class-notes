\documentclass{article}
\usepackage{amsmath}
\usepackage{pgfplots}

\setlength\parindent{0pt}
\pgfkeys{/pgfplots/MyAxisStyle/.style={xmin=-10,xmax=10, ymin=-10,ymax=10,height=6cm,width=6cm}}

\begin{document}

\section*{Sample Space}

\begin{quote}
The sample space $S$ of an expirement is a set consisting of all
possible experimental outcomes (e.g. $S = {1, \ldots, 6}$).
\end{quote}

The problem arises because it is impossible to encompass the entire
experimental outcome space, and by modeling the outcomes as 6
integers.

Even in the case of flipping 3 coins, a nickel, dime, and
penny. Assuming nothing lands on the edge, we could devise a format
for representing the outcomes:

\begin{center}
\begin{tabular}{cccl}
D & N & P & \\
\hline
1 & 1 & 1 & All heads \\
0 & 0 & 0 & All tails
\end{tabular}
\end{center}

Naturally, we could model this as an integer, from $[0, 7]$. From
there, we could use this coin model as a 6 sided die model. However,
we run into a problem of the ``all 1s'' case.

For this reason, we should use the quote/definition as:

\begin{quote}
The space $S$ is a non-empty set.
\end{quote}

\section*{Probability Distribution}

One definition would be

\begin{quote}
A set of probability values for an experiement with sample space $S =
{ \omega_1, \ldots, \omega_n }$. consists of some probabilities
$P_1, \ldots, P_n$ such that $0 \le p_1 \le 1$, \ldots, $0 \le p_n \le 1$
and $p_1 + \cdots + p_n$.

The probability of outcome $\omega_i$ occuring is said to be $p_i$ and is
written $P(\omega_i) = p_i$.
\end{quote}

A problem arises if the set of outcomes is infinite. It is possible
that each of the outcomes has a probability of $0$, which means the
$\sum_i p_i$ would also be $0$.

The takeaway is that some outcomes are allowed to have a probability,
and some are not. An outcome that is allowed is called an ``event''.

A better set of definitions would be

\begin{quote}
A sample space is a non-empty set.

An outcome is an element of a sample space.

Events are subsets of the sample space \ldots events have
probabilities.

The probability of outcome $\omega_i$ occuring is said to be $p_i$ and
is written $P({\omega_i})=p_i$.
\end{quote}

We are allowed to shorten $P({\omega_i})$ as $P(\omega_i)$, assuming
we recognize that this is shorthand for the set containing the one
element.

\subsection*{Example}

We have a factory creating resistors. The allowed range for a resistor
is between $50$ and $150$. We can call the sample space of these
values $S={x\mid 50\le x\le 150}$.

An outcome is any $x$ such that $50\le x \le 150$.

If we were to pull out any any resistor, for instance $101.56$, we
would have pulled out an ``outcome''. Does $101.56$ have a
probability? No. Only events can have probabilities, but events must
be subsets of the sample space, not elements. ${101.56}$ would be an
event, so it could have a probability.

We could also look at the event that an outcome is in the set
$\Omega={x\mid 95\le x\le 100}$, with $P(\Omega)=\rho$.

\end{document}
