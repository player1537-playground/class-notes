\documentclass{article}
\usepackage{amsmath}
\usepackage{amssymb}
\usepackage{pgfplots}

\setlength\parindent{0pt}
\setlength\parskip{1em}
\pgfkeys{/pgfplots/MyAxisStyle/.style={xmin=-10,xmax=10, ymin=-10,ymax=10,height=6cm,width=6cm}}

%\newcommand{\checkmark}{✓}

\begin{document}

\section*{Distributive Properties}

\textit{I don't actually know what he was talking about.}

In normal arithmetic, we have

\[
a\cdot(b+c)\overset{\checkmark}{=}(a\cdot b)+(a\cdot c)
\]

and

\[
a+(b\cdot c)\overset{?}{=}(a+b)\cdot(a+c)
\]

Do these same properties apply to sets? The first one indeed does:

\[
A\cap(B\cup C)\overset{\checkmark}{=}(A\cap B)\cup(A\cap C)
\]

The second one we have to look a bit closer to determine if it is
actually valid.

\[
\begin{array}{rcl}
A\cup(B\cap C) & \overset{?}{=} & (A\cup B)\cap(A\cup C) \\
               & \overset{\checkmark}{=} & ((A\cup B)\cap A)\cup((A\cup B)\cap C) \\
\end{array}
\]

\section*{Additional properties}

For $A$, $B$ as members of a sample space $\Omega$.

\[
\begin{array}{c|c}
A\cap A=A & A\cup A=A \\
A\cap B=B\cap A & A\cup\Omega=\Omega \\
A\cup B=B\cup A & A\cap\Omega=A \\
(A')' = A & A\cup A'=\Omega \\
\end{array}
\]

\[
\begin{array}{c}
(A\cup B)\cup C=A\cup(B\cup C) \\
(A\cap B)\cap C = A\cap (B\cap C) \\
\end{array}
\]

\section*{Problems from the book}

Suppose we want to model two swimmers, where each has a mutually
exclusive outcome: he either sinks or floats. We could model this as:

\[
\begin{cases}
S_1:\text{ first sinks} \\
S_2:\text{ second sinks} \\
F_1:\text{ first floats} \\
F_2:\text{ second floats} \\
\end{cases}
\]

We then have to model these events with probabilities. One thing we
have to be careful over is that we can't have a person sink and float
at the same time. We then have to consult what it means for an event
to occur: an outcome has to be an element of that event. This means we
have the following:

\[
S_1\cap F_1 = \varnothing
\]

\[
S_2\cap F_2 = \varnothing
\]

In order for our model to be all-encompassing, we also have to include
the union of separate events:

\[
S_2\cup F_2 = ?
\]

\[
S_1\cup F_1 = ?
\]

We can then start looking at the probabilities assigned to these
events:

\[
\begin{cases}
P(F_1\cap F_2)= 0.24 \\
P(S_1\cap S_2)=0.16 \\
P(S_1\cap F_2)=0.31 \\
P(F_1\cup F_2) = ? \\
P(F_1\cap S_2) = ? \\
\end{cases}
\]

Now we want to acknowledge that the opposite of sinking is floating,
and vice versa. Symbolically:

\[
\begin{cases}
S_2'=F_2 \\
F_2'=S_2 \\
S_1'=F_1 \\
F_1'=S_1 \\
\end{cases}
\]

With this knowledge in hand, we can now evaluate some of the previous
expressions:

\[
\begin{cases}
S_2\cup F_2 = \Omega \\
S_1\cup F_1 = \Omega \\
\end{cases}
\]

\subsection*{Computing $P(F_1\cup{}F_2)$}

Now we can start trying to compute the above probability:

\[
\begin{array}{rcl}
F_1\cup F_2 & = & ((F_1\cup F_2)')' \\
            & = & (F_1'\cap F_2')' \\
            & = & (S_1\cap S_2)' \\
\end{array}
\]

And then bring this back to the given probabilities

\[
\begin{array}{rcl}
P(F_1\cup F_2) & = & P((S_1\cap S_2)') \\
               & = & 1 - P(S_1\cap S_2) \\
               & = & 1 - 0.16 \\
\end{array}
\]

\subsection*{Computing $P(F_1\cap{}S_2)$}

To compute this, we go back to our favorite equation:

\[
P\left(\bigcup_{E\in\zeta} E\right) = \sum\limits_{E\in\zeta} P(E)
\]

Recall that this equation relies on the mutually exclusiveness of
$\zeta$. Our particular set of events is the following:

\[
\zeta=\left\{F_1\cap S_2, F_1\cap F_2, S_1\cap S_2, S_1\cap F_2 \right\}
\]

Not shown here, but these are actually mutually exclusive. You would
have to check this.

Then we can look at:

\[
P(F_1\cap S_2) = P((F_1\cap S_2)\cup (F_1\cap F_2)\cup (S_1\cap S_2)\cup (S_1\cap F_2))
\]

\[
P((F_1\cap S_2)\cup (F_1\cap F_2)\cup (S_1\cap S_2)\cup (S_1\cap F_2)) = P(F_1\cap S_2) + 0.24 + 0.16 + 0.31
\]

Now we want to rewrite the large union using only $F_1$ and $F_2$.

\[
\underbrace{
\underbrace{(F_1\cap F_2')\cup(F_1\cap F_2)}_{
\underbrace{F_1\cap(F_2'\cup F_2)}_{
\underbrace{F_1\cap\Omega}_{
F_1
}
}
}\cup\underbrace{(F1'\cap F_2')\cup(F_1'\cap F_2)}_{
\underbrace{F_1'\cap(F_2'\cup F_2)}_{
\underbrace{F_1'\cap\Omega}_{
F_1'
}
}
}
}_{\Omega}
\]

And obviously, $P(\Omega)=1.0$, so $P(F_1\cap{}S_2)=1.0$.

\subsection*{Extra Credit: Proving that $S2$ and $F2$ are complements}

Given only the following information

\[
\begin{cases}
S_2\cup F_2=\Omega \\
S_1\cup F_1=\Omega \\
S_1\cap F_1=\varnothing \\
S_2\cap F_2=\varnothing \\
\end{cases}
\]

We can prove the following:

\[
\begin{cases}
S_2'=F_2 \\
F_2'=S_2 \\
S_1'=F_1 \\
F_1'=S_1 \\
\end{cases}
\]

\subsection*{Next section}

Suppose we roll 3 dice, with a sample space:

\[
\Omega=\left\{\left\langle r_1, r_2, r_3\right\rangle\mid 0<r_i<7\quad\forall i\right\}
\]

The set of all subsets of $\Omega$ is:

\[
S = 2^\Omega
\]

I don't really know.

\[
P(\{\langle r_1, r_2, r_3\rangle\})=
\]

\end{document}
