\documentclass{article}
%\usepackage[margin=0.75in]{geometry}
\usepackage{amsmath}
\usepackage{amssymb}
\usepackage{xintexpr}
\usepackage{multicol}
\usepackage{empheq}
\usepackage{forest}
\usepackage{booktabs}
\usepackage{ifthen}
\usepackage{xfrac}

\setlength{\columnsep}{0.5cm}
\setlength{\columnseprule}{1pt}

\newcommand{\T}{1}
\newcommand{\F}{0}
\newcommand{\TF}[1]{\if1#1\T\else\F\fi}
\newcommand{\xintTF}[1]{\xintifboolexpr{#1}{\T}{\F}}

\newcommand{\logicrule}[2]{
\begin{array}{l}
#1 \\
\midrule
\therefore #2 \\
\end{array}
}

\newcommand{\problem}[2]{$\boxed{\text{#1.#2}}$}
\newcommand{\subproblem}[3]{$\boxed{\text{(#3)}}$}
\newcommand{\solution}[3]{\boxed{#3\quad(\text{#1.#2})}}
\newcommand{\subsolution}[4]{\boxed{#4\quad(\text{#1.#2#3})}}

\newcommand{\inv}[1]{#1^{-1}}

\renewcommand{\d}[1]{\,\textnormal{d}#1}
\newcommand{\dd}[2]{\frac{\d{#1}}{\d{#2}}}
\newcommand{\ddd}[2]{\dfrac{\d{#1}}{\d{#2}}}

\newcommand{\p}[1]{\,\partial #1}
\newcommand{\pp}[2]{\frac{\p{#1}}{\p{#2}}}
\newcommand{\ppp}[2]{\dfrac{\p{#1}}{\p{#2}}}

\newcommand{\multistep}[1]{\begin{array}{rl} #1 \end{array}}
\newcommand{\subeq}{\subseteq}
\newcommand{\sub}{\subset}

\DeclareMathOperator{\var}{Var}
\DeclareMathOperator{\E}{\mathcal{E}}

\setlength\parindent{0pt}
\setlength\parskip{0em}

\begin{document}

\section*{Homework 8}

%
\problem{4.4}{4}

Let $X$ be a random variable for the time to failure in hours, modeled
by a Weibull distribution with parameters $a=3$ and $\lambda=0.5$.

%
\subproblem{4.4}{4}{a} The median failure time of the circuit can be
found by solving for $\tilde{x}_{0.5}$ in:

\[
F_X(\tilde{x}_{0.5})=0.5
\]

Where the CDF is given by:

\[
F_X(x)=1-e^{-(\lambda x)^a}
\]

This gives a median failure time of:

\[
\subsolution{4.4}{4}{a}{\tilde{x}_{0.5}=1.76999}
\]

%
\subproblem{4.4}{4}{b} If the engineers want to be $99\%$ certain of
how long a circuit will last, then we want to find the $99$th
percentile, or:

\[
F_X(\tilde{x}_{0.99})=0.99
\]

This is given by:

\[
\tilde{x}_{0.99}=\dfrac{(-\ln(1-0.99))^{\sfrac{1}{a}}}{\lambda}
\] \[
\subsolution{4.4}{4}{b}{\tilde{x}_{0.99}=3.32745}
\]

%
\subproblem{4.4}{4}{c} To find the expectation and variance of $X$, we
go to the definitions:

\[
\E X=\frac{1}{\lambda} \Gamma\left(1+\frac{1}{a}\right)
\]

and

\[
\var(X)=\frac{1}{\lambda^2}\left(\Gamma\left(1+\frac{2}{a}\right)-\Gamma\left(1+\frac{1}{a}\right)^2\right)
\]

\[
\subsolution{4.4}{4}{c}{\multistep{
\E X&= 1.78596 \\
\var(X)&= 0.421332 \\
}}
\]

%
\subproblem{4.4}{4}{d} Let $X_1$, $X_2$, $X_3$, and $X_4$ be
independent Weibull distributions with parameters $a=3$ and
$\lambda=0.5$. Let $A_i$ be the event that $X_i$ is less than 3
hours. We want to find the probability $P(A_1A_2A_3A_4)$. Because each
variable, and therefore each event, is independent, then we can
rewrite it by the definition of independent events:

\[
\multistep{
P(A_1A_2A_3A_4)&=P(A_1)P(A_2)P(A_3)P(A_4) \\
&=P(X_1<3)P(X_2<3)P(X_3<3)P(X_4<3) \\
}
\]

Similarly, because each $X_i\sim{}X$, then this simplifies to:

\[
\multistep{
P(A_1A_2A_3A_4)&=(P_X(3))^4 \\
&=\left(1-e^{-(\lambda 3)^a}\right)^4 \\
}
\] \[
\subsolution{4.4}{4}{d}{P(A_1A_2A_3A_4)=0.869994}
\]

%
\problem{4.4}{8}

Let $X$ be a random variable for the number of days up to and
including the day a patient is recovered, modeled by a Weibull
distribution with parameters $a$ and $\lambda$. Let $F_X(x)$ be the
percent of people who have recovered, where:

\[
F_X(x)=1-e^{-(\lambda x)^a}
\]

After $7$ days, it is found that $\sfrac{9}{82}$ patients have
recovered:

\[
F_X(7)=\sfrac{9}{82}
\]

Then, after $14$ days, it is found that $\sfrac{24}{82}$ patients have
recovered:

\[
F_X(14)=\sfrac{24}{82}
\]

We can set up a system of equations and solve for $a$ and $\lambda$:

\[
\begin{cases}
\sfrac{9}{82}=1-e^{-(\lambda 7)^a} \\
\sfrac{24}{82}=1-e^{-(\lambda 14)^a} \\
\end{cases}
\]

Using Sympy, we end up with:

\[
\begin{cases}
%soln = [(1.57457097793974, 0.0364218706301951)]
a=1.5745709 \\
\lambda= 0.0364219 \\
\end{cases}
\]

Then, the median time to recovery is given by:

\[
\tilde{x}_{0.5}=\dfrac{\left(-\ln(1-0.5)\right)^{\sfrac{1}{a}}}{\lambda}
\] \[
\solution{4.4}{8}{\tilde{x}_{0.5}=21.7544}
\]

%
\problem{4.8}{10}

Let $X_i$ for $1\le{}i\le10$ be a random variable for the strength of
a chemical solution. Let $W_i$ be the event $X_i<0.5$, where the
solution is weak. Let $G_i$ be the event $0.5<X_i<0.8$, where the
solution is good. Let $S_i$ be the event $X_i>0.8$, where the solution
is strong. Each random variable, $X_i$, is independently modeled by a
Beta distribution with $a=18$ and $b=11$.

We want to find the probability that one batch is weak, one is strong,
and eight are good, i.e.:

\[
P\left(W_1S_2\bigcap\limits_{i=3}^{10} G_i\right)=?
\]

Because each event is independent, we can multiply these probabilities
together independently:

\[
P(W_1)P(S_2)\prod\limits_{i=3}^{10}P(G_i)
\]

Substituting each event, we get:

\[
P(X_1<0.5)P(X_2>0.8)\prod\limits_{i=3}^{10}P(0.5<X_i<0.8)
\]

These can be rewritten in terms of the CDF, $F_X$. Because each
variable is modeled by the same distribution, we can just use the one
CDF and write them all in terms of it:

\[
F_X(0.5)(1-F_X(0.8))\left(F_X(0.8)-F_X(0.5)\right)^8
\]

Where the CDF is given by:

\[
F_X(z)=\int\limits_{0}^{z} \dfrac{\Gamma(a+b)}{\Gamma(a)\Gamma(b)} x^{a-1}(1-x)^{b-1}\d{x}
\]

Using Scipy, the probability evaluates to:

\[
\solution{4.8}{10}{0.0005541369}
\]

%
\problem{5.1}{10}

Let $X$ be a random variable for the weight of a packet of sugar. Let
$x_0=1$ be the separator between underweight packets and overweight
ones ($X<x_0$ and $X>x_0$, respectively). Let $X$ be modeled by a
normal distribution with mean $\mu=1.03$ and standard deviation
$\omega=0.014$.

%
\subproblem{5.1}{10}{a} We want to find the proportion of packets that
are underweight. Suppose $N$ is the number of packets produced. The
proportion, then, would be:

\[
\% = \frac{N P(X<x_0)}{N}
\]

Therefore, the proportion is just the probability that a packet is
underweight. The PDF for $X$ is given by:

\[
\rho_X(x)=\dfrac{1}{\sigma \sqrt{2\pi}}\,e^{\dfrac{-(x-\mu)^2}{2\sigma^2}}
\]

The CDF, then, is:

\[
F_X(x)=\int\limits_{-\infty}^x \rho_X(z)\d{z}
\]

Thus the percentage we are looking for is:

\[
\subsolution{5.1}{10}{a}{F_X(1)=0.0160623}
\]

%
\subproblem{5.1}{10}{b} Suppose that we have another random variable,
$Y$, defined in the same way as $X$ but with $\mu=1.05$ and
$\omega=0.016$. We want to find the proportion of underweight packets
for $Y$ and compare it to that for $X$. Again, we look at the CDF
function at $x_0=1$.

\[
F_Y(1)=0.00088903
\]

We can see that $F_Y(1)$ is several orders of magnitude smaller than
$F_X(1)$.

%
\subproblem{5.1}{10}{c} The expected excess package weight for $X$ and
$Y$ are given by $\E(X-x_0)$ and $\E(Y-x_0)$ respectively. Because
subtracting a constant from a normal distribution only changes the
mean, then this is the same as defining $X'\sim{}N(0.03,0.014)$ and
$Y'\sim{}N(0.05,0.016)$ and finding $\E{}X'$ and $\E{}Y'$, where the
arguments for $N$ are $\mu$ and $\omega$. Similarly, because normal
distributions are symmetric across the mean, then the expected value
is just the mean, and therefore:

\[
\subsolution{5.1}{10}{c}{\multistep{
\E X'&=0.03 \\
\E Y'&=0.05 \\
}}
\]

%
\problem{5.2}{7} Let $X_1$ be a random variable for the worth of the
first mutual fund after a year with parameter $x_1$, the initial
amount of money you put in the fund, where
$X_1\sim{}N(1.05x_1,0.0002x^2)$, where $N(\mu,\sigma^2)$ represents a
normal distribution with mean $\mu$ and standard deviation $\sigma$.

Let $X_2$ be another random variable for the worth of the second
mutual fund, with parameter $x_2$, where
$X_2\sim{}N(1.05x,0.0003x^2)$.

Suppose that $x_1=y$ and $x_2=1000-y$, and let the random variable
$Y=X_1+X_2$ be the total worth of both funds.

%
\subproblem{5.2}{7}{a} The expected value of $Y$, $\E{}Y$, is given
by:

\[
\multistep{
\E Y&=\E(X_1+X_2) \\
&=\E(N(1.05y,0.0002y^2)+N(1.05(1000-y),0.0003(1-y)^2)) \\
&=1.05y+1000\cdot1.05-1.05y \\
}
\]

\[
\subsolution{5.2}{7}{a}{\E Y=1050}
\]

%
\subproblem{5.2}{7}{b} The variance is given by:

\[
\multistep{
\var(Y)&=\var(X_1+X_2) \\
&=\var(N(1.05y,0.0002y^2)+N(1.05(1000-y),0.0003(1-y)^2)) \\
&=0.0002y^2+0.0003(1-y)^2 \\
}
\]

%
\subproblem{5.2}{7}{c} This is minimized when:

\[
0=\dd{}{y}0.0002y^2+0.0003(1-y)^2
\]

at

\[
y=0.6
\]

With this strategy, the probability $P(Y>1060)$ where
$Y\sim{}N(1050,0.000120)$ is given by:

\[
\multistep{
P(Y>1060)&=1-P_Y(1060) \\
&=\int\limits_{-\infty}^{1060} \rho_Y(z)\d{z} \\
}
\]


\end{document}
