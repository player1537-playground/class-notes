\documentclass{article}
\usepackage{amsmath}
\usepackage{pgfplots}
\usepackage{amssymb}

\newcommand{\RN}[1]{\textrm{\uppercase\expandafter{\romannumeral #1\relax}}}

\setlength\parindent{0pt}
\setlength\parskip{1em}
\pgfkeys{/pgfplots/MyAxisStyle/.style={xmin=-10,xmax=10, ymin=-10,ymax=10,height=6cm,width=6cm}}

\begin{document}

\section*{Countable Additivity of Mutually Exclusive Events}

If we have a sample space $\Omega$ with

\[
\Omega=\left\{\RN{1}, \RN{2}, \RN{3}, \RN{4}, \RN{5}, \RN{6}\right\}
\]

If $\zeta$ is a countable, mutually exclusive collection of events, such as

\[
\zeta=\left\{
\{\RN{1}\},
\{\RN{2}\},
\{\RN{3}\},
\{\RN{4}\},
\{\RN{6}\}
\right\}
\]

(Note that element $\{\RN{5}\}$ is missing).

then

\[
P\left(\bigcup_{A\in{}\zeta} A\right) = \sum\limits_{A\in{}\zeta} P(A)
\]

Or in other words, the probability of the union of all of the events
in a set of mutually exclusive events is the sum of the probabilities
of each event. Which is pretty obvious.

\section*{Elementary Events}

An ``elementary event'' is a set of the form $\{x\}$ where
$x\in{}\Omega$.

\section*{Automagically Mutually Exclusive}

If our collection $\zeta$ is a set of elementary events, then it is
automagically mutually exclusive. Think: if the set contained some
elementary event twice, then the set would take care of squashing them
together, so that there is only $1$ of those elementary events.

\section*{Not-Automagically Mutually Exclusive}

However, if our collection contained one event that was not
elementary, then we would have to check each element of the
collection.

For example, if we had $\zeta=\{\{a\}, A, \{b\}\}$, then we would have
to check that every combination of two events, intersected, would be
the empty set. In other words, we would check:

\[
\forall x\in\zeta\! \forall y\in\zeta\! x\cap y=\varnothing
\]

\end{document}
