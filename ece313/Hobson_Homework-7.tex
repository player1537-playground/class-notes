\documentclass{article}
%\usepackage[margin=0.75in]{geometry}
\usepackage{amsmath}
\usepackage{amssymb}
\usepackage{xintexpr}
\usepackage{multicol}
\usepackage{empheq}
\usepackage{forest}
\usepackage{booktabs}
\usepackage{ifthen}
\usepackage{xfrac}

\setlength{\columnsep}{0.5cm}
\setlength{\columnseprule}{1pt}

\newcommand{\T}{1}
\newcommand{\F}{0}
\newcommand{\TF}[1]{\if1#1\T\else\F\fi}
\newcommand{\xintTF}[1]{\xintifboolexpr{#1}{\T}{\F}}

\newcommand{\logicrule}[2]{
\begin{array}{l}
#1 \\
\midrule
\therefore #2 \\
\end{array}
}

\newcommand{\problem}[2]{$\boxed{\text{#1.#2}}$}
\newcommand{\subproblem}[3]{$\boxed{\text{(#3)}}$}
\newcommand{\solution}[3]{\boxed{#3\quad(\text{#1.#2})}}
\newcommand{\subsolution}[4]{\boxed{#4\quad(\text{#1.#2#3})}}

\newcommand{\inv}[1]{#1^{-1}}

\renewcommand{\d}[1]{\,\textnormal{d}#1}
\newcommand{\dd}[2]{\frac{\d{#1}}{\d{#2}}}
\newcommand{\ddd}[2]{\dfrac{\d{#1}}{\d{#2}}}

\newcommand{\p}[1]{\,\partial #1}
\newcommand{\pp}[2]{\frac{\p{#1}}{\p{#2}}}
\newcommand{\ppp}[2]{\dfrac{\p{#1}}{\p{#2}}}

\newcommand{\multistep}[1]{\begin{array}{rl} #1 \end{array}}
\newcommand{\subeq}{\subseteq}
\newcommand{\sub}{\subset}

\DeclareMathOperator{\var}{Var}
\DeclareMathOperator{\E}{\mathcal{E}}

\setlength\parindent{0pt}
\setlength\parskip{0em}

\begin{document}

\section*{Homework 7}

%
\problem{4.1}{2}

Let $V$ be a random variable with a uniform distribution of:

\[
V\sim U(1.43, 1.60)
\]

%
\subproblem{4.1}{2}{a} We want to find the expected voltage level,
$\E{}V$, which is defined by:

\[
\E V=\frac{1.43+1.60}{2}
\] \[
\subsolution{4.1}{2}{a}{\E V=1.515}
\]

%
\subproblem{4.1}{2}{b} The standard deviation, $\sqrt{\var(V)}$, is
given by:

\[
\sqrt{\var(V)}=\sqrt{\frac{(1.60-1.43)^2}{12}}
\] \[
\subsolution{4.1}{2}{b}{\sqrt{\var(V)}=0.0490747}
\]

%
\subproblem{4.1}{2}{c} The cumulative distribution function, $F_V(x)$,
is:

\[
F_V(x)=\lbrack1.43\le x\le1.60\rbrack\frac{x-1.43}{1.60-1.43}
\] \[
\subsolution{4.1}{2}{c}{F_V(x)=\lbrack1.43\le x\le1.60\rbrack\frac{x-1.43}{0.17}}
\]

%
\subproblem{4.1}{2}{d} Using the cumulative distribution function, we
want to find $F_V(1.48)$:

\[
F_V(1.48)=\lbrack1.43\le 1.48\le1.60\rbrack\frac{1.48-1.43}{0.17}
\] \[
\subsolution{4.1}{2}{d}{F_V(1.48)=0.2941176}
\]

%
\problem{4.1}{5}

Let $D$ be a random variable for the diameter of a pin (in
millimeters) with a uniform distribution with bounds $a=4.182$ and
$b=4.185$, i.e.:

\[
D\sim U(a,b)
\]

%
\subproblem{4.1}{5}{a} We want to find the probability that a pin will
fit into a hole of diameter $h=4.184$. For a pin to fit, its diameter
must be smaller than the hole's, so we are looking for
$P(D<h)=F_D(h)$. We can compute this value using the definition for
the CDF for a uniform distribution function:

\[
F_D(x)=\frac{x-a}{b-a}
\]

Then we can substitute the value, $x=h$, that we want:

\[
F_D(h)=\frac{h-a}{b-a}
\] \[
\subsolution{4.1}{5}{a}{F_D(h)=0.666666}
\]

%
\subproblem{4.1}{5}{b} We want to find the probability that the
difference between the diameter of the pin and the diameter of the
whole is within some tolerance $t=0.0005$, given that the pin fits in
the whole. If we introduce an event $H$ that the pin fits into the
hole and an event $T$ that it is within the tolerance, then we are
looking for $P(T|H)$, which is:

\[
P(T|H)=\frac{P(TH)}{P(H)}
\]

We already found $P(H)=F_D(h)$. To find $P(TH)$, we are looking for:

\[
\multistep{
P(|D-h|<t \wedge D<h)&=P(-t<D-h<t \wedge D<h) \\
&=P(h-t<D<t+h \wedge D<h) \\
&=P(h-t<D \wedge D<t+h\wedge D<h) \\
&=P(h-t<D \wedge D<h) \\
&=P(h-t<D<h)
}
\]

We can then separate $P(h-t<D<h)$ as:

\[
\multistep{
P(h-t<D<h)&=P(D<h)-P(D<h-t) \\
&=F_D(h)-F_D(h-t) \\
}
\]

So now the probability we are looking for is given by:

\[
P(T|H)=\frac{F_D(h)-F_D(h-t)}{F_D(h)}
\] \[
\subsolution{4.1}{5}{b}{P(T|H)=0.245}
\]

%
\problem{4.2}{6}

Let $X$ be a random variable for the distance between imperfections in
an strand of optical fiber and be modeled by an exponential
distribution with parameter $\lambda=2$:

%
\subproblem{4.2}{6}{a} The expected distance between imperfections,
$\E{}X$, can be calculated with:

\[
\E X=\frac{1}{\lambda}
\] \[
\subsolution{4.2}{6}{a}{\E X=0.5}
\]

%
\subproblem{4.2}{6}{b} The probability that the distance between two
imperfections being longer than $1$ meter, $P(X>1)$, can be computed
using the CDF:

\[
\multistep{
P(X>1)&=1-P(X\le 1) \\
&=1-F_X(1) \\
&=1-\left(1-e^{-\lambda\cdot1}\right) \\
&=e^{-2} \\
}
\]

\[
\subsolution{4.2}{6}{b}{P(X>1)=0.135335}
\]

%
\subproblem{4.2}{6}{c} The distribution of the number of imperfections
in a $3$ meter stretch of fiber, called $Y$, is a Poisson
distribution:

\[
Y\sim \text{Poisson}(\lambda\cdot3)
\] \[
\subsolution{4.2}{6}{c}{Y\sim\text{Poisson}(6)}
\]

%
\subproblem{4.2}{6}{d} The probability that a $3$ meter stretch of
fiber having no more than $4$ imperfections is given by
$P(Y\le4)=F_Y(4)$:

\[
F_Y(4)=\sum\limits_{i=0}^4 \frac{e^{-6}6^i}{i!}
\] \[
\subsolution{4.2}{6}{d}{F_Y(4)=0.285056}
\]

%
\problem{4.2}{8}

Let $X$ be a random variable for the time of failure of a granite
sample when subjected to stress. We know that $90\%$ of granite
samples fail in the first $5$ hours, or:

\[
F_X(5)=0.90
\]

%
\subproblem{4.2}{8}{a} If $X$ is modeled by an exponential
distribution, we can use the defined formula for the $p$th percentile
to determine a suitable value for $\lambda$. The $p$th percentile is
given by:

\[
F_X(\tilde{x}_p)=p \Leftrightarrow \tilde{x}_p=\frac{-\ln(1-p)}{\lambda}
\]

Using the above definition for the 90th percentile, we have:

\[
F_X(5)=0.90 \Leftrightarrow 5=\frac{-\ln(1-0.90)}{\lambda}
\] \[
\subsolution{4.2}{8}{a}{\lambda = 0.460517}
\]

%
\subproblem{4.2}{8}{b} Using this model, we want to find the
probability that a fracture occurs within 3 hours, or $P(X\le3)$. This
can be computed using the definition of the CDF:

\[
\multistep{
P(X\le3)&=F_X(3) \\
&=1-e^{-\lambda\cdot3} \\
}
\]

\[
\subsolution{4.2}{8}{b}{P(X\le3)=0.748811}
\]

%
\problem{4.3}{5}

Let each $X_i$ be a random variable with a exponential distribution
with parameter $\lambda=2$.

%
\subproblem{4.3}{5}{a} The distribution of the distance between the
first and the fifth imperfection. Each $X_i$ is the distance from the
$i$th imperfection to the $i+1$th imperfection. So because we want to
find the distribution between the first and fifth, then that's
actually the sum of 4 $X_i$'s.

Let $X$ be this random variable, which is a gamma distribution with
$\lambda=2$ and $k=4$.

%
\subproblem{4.3}{5}{b} The expectation of the distance between the
first and fifth imperfections, $\E{}X$, is:

\[
\E X=\frac{k}{\lambda}
\] \[
\subsolution{4.3}{5}{b}{\E X=2}
\]

%
\subproblem{4.3}{5}{c} The standard deviation is given by:

\[
\multistep{
\sqrt{\var(X)}&=\sqrt{\frac{k}{\lambda^2}} \\
&=\frac{\sqrt{k}}{\lambda} \\
}
\]

\[
\subsolution{4.3}{5}{c}{\sqrt{\var(X)}=1}
\]

%
\subproblem{4.3}{5}{d} Suppose the distance between the first and
fifth imperfections is greater than 3 meters. The probability that
this occurs, $P(X>3)$, can be computed with the CDF:

\[
\multistep{
P(X>3)&=1-P(X\le3) \\
&=1-F_X(3) \\
&=1-\int\limits_{0}^3 \dfrac{\lambda^k x^{k-1} e^{-\lambda x}}{\Gamma(k)} \d{x} \\
&=0.787801 \\
}
\]

%
\problem{4.3}{6}

Let $Y$ be the distribution for the time between the first and fourth
arrivals of the day, which is a gamma distribution. We know that an
average of $1.8$ people arrive, so we can compute the $\lambda$
parameter with the definition of the expectation of a Poisson process, $X$:

\[
\E X=1.8=\frac{1}{\lambda}
\] \[
\lambda=0.555555
\]

%
\subproblem{4.3}{6}{a} The distribution of the time between the first
and the fourth, $Y$, is a gamma distribution with $\lambda=0.555555$
and $k=3$.

%
\subproblem{4.3}{6}{b} The expectation of $Y$, $\E{}Y$, is given by:

\[
\E Y=\frac{k}{\lambda}
\] \[
\subsolution{4.3}{6}{b}{\E Y=5.4}
\]

%
\subproblem{4.3}{6}{c} The variance of $Y$, $\var(Y)$, can be computed
with:

\[
\var(Y)=\frac{k}{\lambda^2}
\] \[
\subsolution{4.3}{6}{c}{\var(Y)=9.72}
\]

%
\subproblem{4.3}{6}{d}

\end{document}
