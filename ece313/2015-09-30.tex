\documentclass{article}
\usepackage{amsmath}
\usepackage{amssymb}
\usepackage{booktabs}
\usepackage{xintexpr}

\newcommand{\T}{1}
\newcommand{\F}{0}
\newcommand{\TF}[1]{\if1#1\T\else\F\fi}
\newcommand{\xintTF}[1]{\xintifboolexpr{#1}{\T}{\F}}

\newcommand{\logicrule}[2]{
\begin{array}{l}
#1 \\
\midrule
\therefore #2 \\
\end{array}
}

\newcommand{\inv}[1]{#1^{-1}}

\renewcommand{\d}[1]{\,\textnormal{d}#1}
\newcommand{\dd}[2]{\frac{\d{#1}}{\d{#2}}}
\newcommand{\ddd}[2]{\dfrac{\d{#1}}{\d{#2}}}

\setlength\parindent{0pt}
\setlength\parskip{1em}

\begin{document}

\section*{Example 2.1.10}

We have a random variable $X$ that takes on the values
$\{1,2,\cdots\}$.

\subsection*{Example 2.1.10 (part a)}

We want to know if it's possible to have a PMF such that:

\[
P_X(i)=\frac{c}{i^{1.5}}
\]

We might start be constructing a probability space using $X$:

\[
X:\Omega\rightarrow\mathbb{R}
\] \[
\text{probability space}: (\Omega,S,P)
\] \[
P:S\rightarrow\lbrack0,1\rbrack
\]

We could also apply the nice formula:

\[
P(\alpha(X))=\sum\limits_{\alpha(z)}P_X(z)
\]

Where $\alpha(z)=``z\in\mathbb{R}''$

We also apply the relation:

\[
P(X\in\mathbb{R})=P(\inv{X}\mathbb{R})
\]

But we know that this must be the probability space:

\[
P(\Omega)=1
\]

That last step might not be clear, so let's look at the definition of
the inverse image.

\textbf{Note}: Midterm is before spring break. Make a study sheet.

Suppose we have some function from a domain to a codomain:

\[
f:D\rightarrow C
\]

Then the inverse image is defined to be:

\[
\inv{f}A\overset{\text{def}}{=}\{x|f(x)\in A\}
\]

We can apply this to the value $\inv{X}\mathbb{R}$:

\[
\inv{X}\mathbb{R}=\{x|X(x)\in\mathbb{R}\}
\] \[
\inv{X}\mathbb{R}=\Omega
\]

Who cares? Because we are showing that, if we choose our $\alpha(x)$
in this particular way, then it must have the value $1$. Now we can
look at the summation on the right. In other words:

It must be the case that:

\[
1=\sum\limits_{z\in\mathbb{R}}P_X(z)
\]

Now we can start looking back at the original PMF:

\[
P_X(i)=\frac{c}{i^{1.5}}
\]

It's often assumed that, if a PMF is given in this way, then we assume
that $i$ must be an integer. Now what happens if it's not an integer?
It's often implied that the function actually means:

\[
P_X(i)=\begin{cases}\dfrac{c}{i^{1.5}} & \text{if $i\in\{1,2,\cdots\}$} \\ 0 & \text{otherwise} \\\end{cases}
\]

Looking back at the sum, we think about what it means if $z$ is a
positive integer or not. If it is, then it takes on an actual value,
given by $cz^{-1.5}$. Otherwise, it is the value $0$. Adding up a
bunch of $0$s doesn't change the value of the summation. Therefore, we
can limit the summation bounds, and get:

\[
1=\sum\limits_{z\in\mathbb{Z}^+}\frac{c}{z^{1.5}}
\]

And furthermore:

\[
1=\sum\limits_{z=1}^\infty \frac{c}{z^{1.5}}
\] \[
1=c\sum\limits_{z=1}^\infty z^{-1.5}
\]

Suppose we introduce the variable $v$ to equal the summation:

\[
v=\sum\limits_{z=1}^\infty z^{-1.5}
\]

Then if we let $c=v^{-1}$, then the statement is now:

\[
1=v^{-1}v
\] \[
1=1
\]

\subsection*{Example 2.1.10 (part b)}

We want to know if it's possible to have a PMF such that:

\[
P_X(i)=\frac{c}{i\log i}
\]

\end{document}
