\documentclass{article}
%\usepackage[margin=0.75in]{geometry}
\usepackage{amsmath}
\usepackage{amssymb}
\usepackage{xintexpr}
\usepackage{multicol}
\usepackage{empheq}
\usepackage{forest}
\usepackage{booktabs}
\usepackage{ifthen}
\usepackage{xfrac}

\setlength{\columnsep}{0.5cm}
\setlength{\columnseprule}{1pt}

\newcommand{\T}{1}
\newcommand{\F}{0}
\newcommand{\TF}[1]{\if1#1\T\else\F\fi}
\newcommand{\xintTF}[1]{\xintifboolexpr{#1}{\T}{\F}}

\newcommand{\logicrule}[2]{
\begin{array}{l}
#1 \\
\midrule
\therefore #2 \\
\end{array}
}

\newcommand{\problem}[2]{$\boxed{\text{#1.#2}}$}
\newcommand{\subproblem}[3]{$\boxed{\text{(#3)}}$}
\newcommand{\solution}[3]{\boxed{#3\quad(\text{#1.#2})}}
\newcommand{\subsolution}[4]{\boxed{#4\quad(\text{#1.#2#3})}}

\newcommand{\inv}[1]{#1^{-1}}

\renewcommand{\d}[1]{\,\textnormal{d}#1}
\newcommand{\dd}[2]{\frac{\d{#1}}{\d{#2}}}
\newcommand{\ddd}[2]{\dfrac{\d{#1}}{\d{#2}}}

\newcommand{\p}[1]{\,\partial #1}
\newcommand{\pp}[2]{\frac{\p{#1}}{\p{#2}}}
\newcommand{\ppp}[2]{\dfrac{\p{#1}}{\p{#2}}}

\newcommand{\multistep}[1]{\begin{array}{rl} #1 \end{array}}
\newcommand{\subeq}{\subseteq}
\newcommand{\sub}{\subset}

\DeclareMathOperator{\var}{Var}
\DeclareMathOperator{\E}{\mathcal{E}}

\setlength\parindent{0pt}
\setlength\parskip{0em}

\begin{document}

\section*{Homework 6}

%
\problem{3.2}{4}

Suppose $X_1,X_2,\cdots,X_r$ are independent random variables, each
with a geometric distribution parameter $p$.

%
\subproblem{3.2}{4}{a} We want to show that the sum of random
variables, $Y=X_1+X_2+\cdots+X_r$, is the same as the negative
binomial distribution with parameters $p$ and $r$. First, we want to
recognize that each $X_i$ represents the number of tries before we
reach the first success. Because the parameter $p$ is fixed for both a
geometric and negative binomial distribution, we can lay out each of
the independent successes next to each other, as follows, where $0$
represents a failure, and $1$ represents a success.

\[
\left\langle \underbrace{0, 0, 1}_{X_1}, \underbrace{0, 1}_{X_2}, \cdots, \underbrace{1}_{X_{r-1}}, \underbrace{0, 0, 0, 1}_{X_r}\right\rangle
\]

Laid out in this way, it's clear that the sum of each $X_i$ is the
same thing as the negative binomial distribution, where the latter
counts the number of attempts before $r$ successes occurs.

%
\subproblem{3.2}{4}{b} The expectation we want to calculate is:

\[
\E Y=\E (X_1+X_2+\cdots+X_r)
\]

We know that, for expectations of sums of random variables, we can
rewrite it as the sum of individual expectations of those random
variables.

\[
\E Y = \E X_1 + \E X_2 + \cdots + \E X_r
\]

Because each random variable is modeled after the same geometric
distribution, they all have the same expectation:

\[
\E Y = r\frac{1}{p}
\]

This is the same as we'd expect from a negative binomial distribution.

%
\subproblem{3.2}{4}{c} Now we want to compute the variance of $Y$ and
show that it is the same as we'd get from a negative binomial
distribution. We first write:

\[
\var(Y)=\var(X_1+X_2+\cdots+X_r)
\]

Because each random variable is independent from the others, we can
separate the variance of sums into a sum of variances:

\[
\var(Y)=\var(X_1)+\var(X_2)+\cdots+\var(X_r)
\]

Again, each random variable has the same variance, so we can write:

\[
\var(Y)=r\frac{1-p}{p^2}
\]

Which is what we'd expect from a negative binomial distribution.

%
\problem{3.2}{6}

We can model the random variable, $X$, as a geometric distribution
with probability $p=0.37$, which is the number of attempts up to and
including the one that the container lands on the target.

%
\subproblem{3.2}{6}{a} The expected number of container drops needed
to hit a target, $\E{}X$, is defined for a geometric distribution as:

\[
\multistep{
\E X&=\frac{1}{p} \\
&=2.7027 \\
}
\]

We want to round this up, because the number of container drops must
be an integer (it doesn't make sense to half drop a container, or for
a container to half land on the target). This gives a final expected
number of drops as:

\[
\subsolution{3.2}{6}{a}{\E X=3}
\]

%
\subproblem{3.2}{6}{b} In this case, we create a new random variable,
$Y$, which is a negative binomial distribution with $r=3$. The
expected value of a negative binomial distribution is defined as:

\[
\multistep{
\E Y&=\frac{r}{p} \\
&=8.108108 \\
}
\]

Again, we round up to get the actual expected value:

\[
\subsolution{3.2}{6}{b}{\E Y=9}
\]

%
\subproblem{3.2}{6}{c} If sufficient supplies are dropped by the 10th
drop, then that means that the event must have occurred for some
$x\le10$, or in other words, $P(X\le10)$. But this is the same thing
as the CDF at 10, or $F_X(10)$. This is defined, for a geometric
distribution as:

\[
F_X(10)=1-(1-p)^{10}
\] \[
\subsolution{3.2}{6}{c}{F_X(10)=0.990151}
\]

%
\subproblem{3.2}{6}{d} The probability that the tenth crate is that
one that successfully lands on the target, which is the value of the
PMF at 10: $P_X(10)$. This is defined as:

\[
P_X(10)=(1-p)^{10-1}p
\] \[
\subsolution{3.2}{6}{d}{P_X(10)=0.0057845}
\]

%
\problem{3.2}{14}

We are given the probability that an event will occur (a user will
bounce), that is $p_0=0.93$. We want to find the expected number of
users such that for 10, the event does not occur (these users will not
bounce). To do this, we can create a negative binomial distribution
$Y$ with $p=1-p_0$ and $r=10$, and compute the expected value of this
distribution.

\[
\multistep{
\var(Y)&=\frac{r}{p} \\
&=\frac{r}{1-p_0} \\
}
\]

\[
\solution{3.2}{14}{\var(Y)=142.857}
\]

%
\problem{3.3}{10}

We are given the total number of sites, $N=25$, the number of
contaminated sites, $n=19$, and the number of sampled sites, $r=5$. We
can create a random variable with a hypergeometric distribution with
these parameters. We want to find the value of $P_X(4)$, the
probability that 4 of the sites are contaminated. This is computed
with:

\[
P_X(4)=\dfrac{\binom{r}{4}\cdot\binom{N-r}{n-4}}{\binom{N}{n}}
\] \[
\solution{3.3}{10}{P_X(4)=0.437718}
\]

%
\problem{3.3}{11}

We are told that there are $N=10$ total companies, $n=3$ successful
companies, $r=4$ sampled companies. A hypergeometric distribution $X$
with these parameters is $X$. We want to find $P(X\ge2)$, which is
equivalent to $P(2\le{}X\le 4)$. We can rewrite this as:

\[
P(2\le X\le 4)=\sum\limits_{x=2}^4 P_X(x)
\]

Based on this being a hypergeometric distribution, we know that the
PMF is defined as:

\[
P_X(x) = \dfrac{\binom{r}{x}\cdot\binom{N-r}{n-x}}{\binom{N}{n}}
\]

However, this is only defined when $x\le{}\text{min}\{n,r\}$, so we
can simplify our sum to:

\[
P(2\le X\le 4)=\sum\limits_{x=2}^3 P_X(x)
\]

This ultimately evaluates to:

\[
\solution{3.3}{11}{P(2\le X\le 4)=0.333333}
\]

%
\problem{3.4}{4}

Suppose the random variable $X$ represents the number of cracks in a
tile, modeled by a Poisson distribution with parameter $\lambda=2.4$.

%
\subproblem{3.4}{4}{a} We first want to find the probability that
there are no cracks in the title, or: $P_X(0)$. The definition for the
PMF of a Poisson distribution is:

\[
P_X(x)=\frac{e^{-\lambda}\lambda^x}{x!}
\]

Using this, we can calculate $P_X(0)$:

\[
P_X(0)=\frac{e^{-\lambda}\lambda^0}{0!}
\] \[
\subsolution{3.4}{4}{a}{P_X(0)=0.0907180}
\]

%
\subproblem{3.4}{4}{b} This time we want to find the probability that
there are 4 or more cracks, or $P(X\ge4)$. We can rewrite this as:

\[
\multistep{
P(X\ge4)&=1-P(X<4) \\
&=1-P(X\le3) \\
}
\]

Because there is no general formula for the CDF of a Poisson
distribution, we just have to sum the PMF:

\[
P(X\ge4)=1-\sum\limits_{x=0}^3 P_X(x)
\] \[
\subsolution{3.4}{4}{b}{P(X\ge4)=0.221227}
\]

%
\problem{3.4}{6}

The random variable $X$ is modelled by a Poisson distribution with
$\lambda=4$.

%
\subproblem{3.4}{6}{a} Like 3.4.4, we just apply the formula to get:

\[
\subsolution{3.4}{6}{a}{P_X(0)=0.0183156}
\]

%
\subproblem{3.4}{6}{b} Again, like 3.4.4, we rewrite $P(X\ge6)$ as a
sum of PMF values:

\[
\multistep{
P(X\ge6)&=1-P(X<6) \\
&=1-P(X\le5) \\
&=1-\sum\limits_{x=0}^5 P_X(x) \\
}
\]

\[
\subsolution{3.4}{6}{b}{P(X\ge6)=0.21487}
\]

%
\problem{3.5}{1}

Let there be three types of tires, $a,b,c$, with probabilities
$p_a=0.24$, $p_b=0.48$, and $p_c=0.29$.

%
\subproblem{3.5}{1}{a} The probability that the next 11 customers buy
$n_a=4$ sets of A tires, $n_b=5$ sets of B tires, and $n_c=2$ sets of
A tires is given by:

\[
P(X_a=n_a,X_b=n_b,X_c=n_c)=\frac{(n_a+n_b+n_c)!}{n_a!n_b!n_c!} p_a^{n_a} p_b^{n_b} p_c^{n_c}
\] \[
\subsolution{3.5}{1}{a}{P(X_a=n_a,X_b=n_b,X_c=n_c)=0.0492698}
\]

%
\subproblem{3.5}{1}{b} The probability that fewer than 3 sets of type
A are sold in the next 7 customers is $P(X_a<3)$, and can be
calculated with:

\[
\multistep{
P(X_a<3)&=\sum\limits_{n_a=0}^2 \sum\limits_{n_b=0}^{7-n_a} P(X_a=n_a,X_b=n_b,X_c=7-n_a-n_b) \\
&=\sum\limits_{n_a=0}^2 \sum\limits_{n_b=0}^{7-n_a} \frac{7!}{n_a!n_b!(7-n_a-n_b)!} p_a^{n_a} p_b^{n_b} p_c^{7-n_a-n_b} \\
}
\]

\[
\subsolution{3.5}{1}{b}{P(X_a<3)=0.838047}
\]

%
\problem{3.5}{5}

There are two possibilities for where an order is placed: online and
offline. For each of these, there are two possibilities for the size
of the order: large and small. Let the random variables
$X_1,X_2,X_3,X_4$ represent an online large order, an online small
order, an offline large order, and an offline small order,
respectively. The probabilities for each are:

\[
\begin{array}{c|c|c}
Variable & Expression & Value \\
\midrule
p_1 & 0.60\cdot0.30 & 0.18 \\
p_2 & 0.60\cdot0.70 & 0.42 \\
p_3 & 0.40\cdot0.40 & 0.16 \\
p_4 & 0.40\cdot0.60 & 0.24 \\
\end{array}
\]

We want to find the probability that each type of order has 2 orders
placed. This is:

\[
P(\cdots)=\frac{8!}{2!2!2!2!} 0.18^2 0.42^2 0.16^2 0.24^2
\] \[
\solution{3.5}{5}{P(\cdots)=0.0212377}
\]

\end{document}
