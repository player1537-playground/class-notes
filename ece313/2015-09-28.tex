\documentclass{article}
\usepackage{amsmath}
\usepackage{amssymb}
\usepackage{booktabs}
\usepackage{xintexpr}

\newcommand{\T}{1}
\newcommand{\F}{0}
\newcommand{\TF}[1]{\if1#1\T\else\F\fi}
\newcommand{\xintTF}[1]{\xintifboolexpr{#1}{\T}{\F}}

\newcommand{\logicrule}[2]{
\begin{array}{l}
#1 \\
\midrule
\therefore #2 \\
\end{array}
}

\newcommand{\inv}[1]{#1^{-1}}

\renewcommand{\d}[1]{\,\textnormal{d}#1}
\newcommand{\dd}[2]{\frac{\d{#1}}{\d{#2}}}
\newcommand{\ddd}[2]{\dfrac{\d{#1}}{\d{#2}}}

\setlength\parindent{0pt}
\setlength\parskip{1em}

\begin{document}

\section*{Review of Known Concepts}

We talk about a random variable $f:\Omega\rightarrow\mathbb{R}$. How
do we really quantify discrete variables?

$f$ is discrete iff the range of $f$ is countable.

If $f$ is discrete, we can use this cool formula:

\[
P(\alpha(f))=\sum\limits_{\alpha(x)}P_f(x)
\]

A random variable, $f$, is continuous iff $f$ is not discrete.

Does this mean we can use the following cool formula?

\[
P(\alpha(f))=\int\limits_{\alpha(x)}\rho_f(x)\d{x} = \int\lbrack\alpha(x)\rbrack\rho_f(x)\d{x}
\]

\textbf{No}. But we're going to pretend we can, because why not?

Remember: The PMF is $P_f(x)$. The density function is
$\rho_f(x)$. The CDF function is $F_f(x)$, where:

\[
F_f(x)=P(f\le x)
\]

\subsection*{CDF of Continuous Random Variable}

We want to know the value of $F_f(x)$. We can rewrite it as:

\[
F_f(x)=P(F\le x)=P(\alpha(f))
\] \[
\int\lbrack\alpha(x)\rbrack\rho_f(z)\d{z}
\] \[
\int\lbrack z\le x\rbrack\rho_f(z)\d{z}
\] \[
\int\limits_{-\infty}^x\rho_f(z)\d{z}
\]

\textbf{Note}: It's important to realize that we can find the
probability density function from the CDF.

\[
\dd{}{x}F_f(x)=\dd{}{x}\int_{-\infty}^x\rho_f(z)\d{z}=\rho_f(x)
\]

\section*{Pseudo-Example 2.1.9}

Suppose there are 9 warehouses, and a particular item can be in 3 of
them.

Let $f$ be the random variable which represents the number of calls up
to and including the one to a warehouse that has the item we're
interested in.

We might be interested in the probability that we call the right
warehouse on our first try. In other words:

\[
P(f=1)=P_f(1)=\frac{3}{9}
\]

Similarly, we can look at the probability that it takes two calls:

\[
P(f=2)=P_f(2)
\]

We note that this can be represented as a conditional probability:

\[
P_f(2)=P(\text{2nd call succeeds}|\text{1st failed})
\]

If the first call failed, then we're not going to look at it again, so
now we have only 8 warehouses that we could possibly call. Because we
didn't find the item, then there are still 3 warehouses that house the
item.

Anyways, stuff happens, so we find that:

\[
P_f(2)=(\frac{3}{8})\cdot(1-\frac{3}{9})
\]

\end{document}
