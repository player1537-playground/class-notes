\documentclass{article}
\usepackage{amsmath}
\usepackage{amssymb}
\usepackage{booktabs}
\usepackage{xintexpr}
\usepackage{forest}

\newcommand{\T}{1}
\newcommand{\F}{0}
\newcommand{\TF}[1]{\if1#1\T\else\F\fi}
\newcommand{\xintTF}[1]{\xintifboolexpr{#1}{\T}{\F}}

\newcommand{\logicrule}[2]{
\begin{array}{l}
#1 \\
\midrule
\therefore #2 \\
\end{array}
}

\setlength\parindent{0pt}
\setlength\parskip{1em}

\begin{document}

\section*{Last class}

Apparently, last class, we covered this formula or something:

\[
P\left(\bigcap\limits_{i=1}^n A_i\right)
=
\prod\limits_{j=1}^n P\left(A_j \middle| \bigcap\limits_{k=j+1}^n A_k\right)
\]

To see if this is even remotely correct, let's try different input
values. For $n=1$:

\[
\begin{array}{rcl}
P(A_1) & \overset{?}{=} & P\left(A_i\middle|\Omega\right) \\
       & \overset{?}{=} & \dfrac{P(A_1\cap\Omega)}{P(\Omega)} \\
       & = & P(A_1) \\
\end{array}
\]

Now let's try $n=2$.

\[
\begin{array}{rcl}
P(A_1\cap A_2) & \overset{?}{=} & P\left(A_1\middle|A_2\right) P\left(A_2\middle|\Omega\right) \\
               & \overset{?}{=} & \dfrac{P(A_1\cap A_2)}{P(A_2)}\cdot\dfrac{P(A_2\cap\Omega)}{P(\Omega)} \\
               & = & \dfrac{P(A_2\cap A_2)}{P(\Omega)} \\
\end{array}
\]

How about $n=3$:

\[
P(A_1\cap A_2\cap A_3) = P(A_1\mid A_2\cap A_3) \cdot P(A_2\mid A_3)\cdot P(A_3\mid\Omega)
\]

Yadda yadda, it works, yo.

\section*{Probability Tree}

A probability tree is a means to describe the probability of a certain
set of outcomes. For example, if we are looking at a factory that
produces chips that can be either good or bad. If we pick out 3 chips
at random, say $x_1$, $x_2$, $x_3$. We can form this as a tuple in the
form $(x_1,x_2,x_3)$. Note: we are sampling without replacement.

Now we are told that 9 out of 500 chips are defective. We want to
produce a tree that shows what the probability of the first chip being
defective, and then based on that, what the probability of the second
chip being defective, and so on.

Note in the following, going to the right means picking out a good
chip, but to the left means a bad chip.

\begin{forest}
  [\text{.}
    [$\dfrac{491}{500}$
      [$\dfrac{490}{499}$
        [$\dfrac{489}{498}$]
        [$\dfrac{9}{498}$]
      ]
      [$\dfrac{9}{499}$
        [$\dfrac{490}{498}$]
        [$\dfrac{8}{498}$]
      ]
    ]
    [$\dfrac{9}{500}$
      [$\dfrac{491}{499}$
        [$\dfrac{490}{498}$]
        [$\dfrac{8}{498}$]
      ]
      [$\dfrac{8}{499}$
        [$\dfrac{491}{498}$]
        [$\dfrac{7}{498}$]
      ]
    ]
  ]
\end{forest}

Now that that hell is over, we can tie it back to that other
formula. First we define the events $G_i\forall{}i\in\{1,2,3\}$ to be
the event that the $i$'th chip is good. With this in mind, we can
label the events in our tree as follows:

\begin{forest}
  [\text{.}
    [$\dfrac{491}{500}$
      [$\dfrac{490}{499}$
        [$\dfrac{489}{498}$ [$G_1\cap{}G_2\cap{}G_3$, rotate=90]]
        [$\dfrac{9}{498}$ [$G_1\cap{}G_2\cap{}G_3'$, rotate=90]]
      ]
      [$\dfrac{9}{499}$
        [$\dfrac{490}{498}$ [$G_1\cap{}G_2'\cap{}G_3$, rotate=90]]
        [$\dfrac{8}{498}$ [$G_1\cap{}G_2'\cap{}G_3'$, rotate=90]]
      ]
    ]
    [$\dfrac{9}{500}$
      [$\dfrac{491}{499}$
        [$\dfrac{490}{498}$ [$G_1'\cap{}G_2\cap{}G_3$, rotate=90]]
        [$\dfrac{8}{498}$ [$G_1'\cap{}G_2\cap{}G_3'$, rotate=90]]
      ]
      [$\dfrac{8}{499}$
        [$\dfrac{491}{498}$ [$G_1'\cap{}G_2'\cap{}G_3$, rotate=90]]
        [$\dfrac{7}{498}$ [$G_1'\cap{}G_2'\cap{}G_3'$, rotate=90]]
      ]
    ]
  ]
\end{forest}

Suppose we were interested in the event that $(x_1,x_2,x_3)=(1,1,0)$,
or in other words, $P(G_1\cap{}G_2\cap{}G_3')$.

We can see that, if we follow this path on the tree, that if we
multiply the values along the tree, it's the same as this
probability. For example, the first node to the right on the tree is
the value of $P(G_1\mid\Omega)$. Going right and then right, we have
$P(G_2\mid{}G_1)$. And finally, right, right, left,
$P(G_3\mid{}G_1\cap{}G_2)$.

\end{document}
