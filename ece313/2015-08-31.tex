\documentclass{article}
\usepackage{amsmath}
\usepackage{amssymb}

\newcommand{\T}{1}
\newcommand{\F}{0}
\newcommand{\TF}[1]{\if1#1\T\else\F\fi}
\newcommand{\xintTF}[1]{\xintifboolexpr{#1}{\T}{\F}}

\newcommand{\logicrule}[2]{
\begin{array}{c}
#1 \\
\midrule
\therefore #2 \\
\end{array}
}

\setlength\parindent{0pt}
\setlength\parskip{1em}

\begin{document}

\section*{Rolling Dice}

Suppose we roll 3 dice, with a sample space:

\[
\Omega=\left\{\left\langle r_1, r_2, r_3\right\rangle\mid 0<r_i<7\quad\forall i\right\}
\]

We are also told that the dice are fair. This means we can assign
probabilities to each tuple of dice rolls trivially. For any outcome,
$\langle{}r_1,r_2,r_3\rangle$, we can assign a probability to the
elementary event:

\[
P(\{\langle r_1,r_2,r_3\rangle\})=p\text{ for all choices of $r_1$, $r_2$, $r_3$}
\]

We are going to use that weird counting probability formula:

\[
P\left(\bigcup\limits_{E\in\zeta}E\right)=\sum\limits_{E\in\zeta}P(E)
\]

With

\[
\zeta=\left\{\left\{\left\langle r_1,r_2,r_3\right\rangle\right\}\mid 0<r_i<7\forall i\right\}
\]

These events have to be mutually exclusive to use Weird Formula. We
can show this symbolically with:

\[
\{\langle r_1,r_2,r_3\rangle\}\cap\{\langle r_1', r_2', r_3'\rangle\} = \varnothing
\]

I don't really know how to prove this, but he sure does.

We're going to show lots of steps to calculate everything.

We start with the inside of $P(\dots)$

\[
\bigcup\limits_{E\in\zeta}E
\]

This union can be rewritten as:

\[
\bigcup\limits_{\{\langle r_1,r_2,r_3\rangle\}\in\zeta} \{\langle r_1,r_2,r_3\rangle\}
\]

But this is the same as:

\[
\left\{\langle r_1,r_2,r_3\rangle\mid 0<r_i<7\forall i\right\}
\]

Which is exactly the same as the sample space $\Omega$.

Now we start looking at the summation on the right.

\[
P\left(\bigcup\limits_{E\in\zeta}E\right) = \sum\limits_{E\in\zeta} P(E)
\]

The inside of the left is $\Omega$ and $P(\Omega)=1$. Now we have

\[
1 = \sum\limits_{E\in\zeta} p
\]

But this is just the number of events in the sample space times $p$, or

\[
1=216 p
\]

We can trivially solve this for what the individual probability of
each elementary event is.

\subsection*{Slightly different problem}

Now we can begin to focus on a different type of problem. We're
interested in the number of outcomes where there are two fives in a
row.

This brings up an interesting problem: is it two fives only, or two or
more fives.

The specific wording is that there are two or more fives in
succession.

The event we're interested in is:

\[
E =
\left\{\langle 5, 5, r_3 \rangle\mid 0<r_3<7\right\}
\cup
\left\{\langle r_1, 5, 5 \rangle\mid 0<r_1<7\right\}
\]

If we were only interested in the case that exactly two fives appeared
in succession, we would have to add additional restrictions on $r_1$
and $r_3$.

Our collection of events will be as follows:

\[
\zeta=\left\{\{e\}\mid e\in E\right\}
\]

Let's be a little more rigorous and think about what kinds of things
are in $\zeta$. Firstly, the elements $e$ come from $E$, which is a
set of tuples. Looking just at the left side of the union, we have
values like:

\[
\{\langle 5,5,1\rangle\} \in \zeta
\]

and

\[
\{\langle 5,5,2\rangle\} \in \zeta
\]

and so on.

What happens when we union up all of these individual sets?

\[
\bigcup\limits_{A\in\zeta}A
\]

The end result is that we will get a set:

\[
\{\dots\}
\]

But these elements are from $E$, which is a set of triples, so we now
have something like:

\[
\left\{\langle 5,5,r_3\mid 0<r_3<7\right\}
\]

But, we also have things from the right side of the union (for $E$),
so we also have that the union of everything in $\zeta$.

\[
\left\{\langle r_1,5,5\mid 0<r_1<7\right\}
\]

Which means our total union is:

\[
\left\{\langle 5,5,r_3\mid 0<r_3<7\right\}
\cup
\left\{\langle r_1,5,5\mid 0<r_1<7\right\}
\]

Duh.

We notice that when we union everything in $\zeta$, we get $E$ back
exactly. Let's show why, or something. Now he's confused... Yeah, stop
using $E$ as your dummy variable and as your event variable. I don't
think that's a problem.

We did something kind of weird, we took $E$ as a set of tuples, and
then we constructed $\zeta$ as a set of elementary sets. So when we
take the union of all of these, we're going to just get $E$ back,
because we didn't do anything. We just rewrote stuff. Y'know,
\textbackslash bigcup for days.

Anyways, we have the second side of that Weird Equation now.

\[
\sum\limits_{A\in\zeta}P(A)
\]

Which is just

\[
\sum\limits_{A\in\zeta} p
\]

Which is trivially

\[
p\left|\zeta\right| = \frac{|E|}{216}
\]

Now we just need to compute the size of $E$. Easy! Just

\[
|E| = 12
\]

But! We've overcounted, because there is the value
$\langle{}5,5,5\rangle$, which is counted in each one.

Finally,

\[
P\left(\bigcup\limits_{A\in\zeta} A\right) = \frac{11}{216}
\]

\section*{Sample spaces are sometimes inconvenient, yo}

Night clubs! We're so hip!

Anyways. Let's say that there is a night club which admits people,
people with a gender (either male or female, not both, in this
problem), and with an age (either younger than 30 or older than 30,
but not both), and that the people that come in are either admitted or
not admitted (not both wink-y smiley face).

So. By stating that you're either male or female, that means that they
are complements

\[
M = F'
\]

Similarly, young/old

\[
Y'=\text{ old}
\]

and admittance (sp?)

\[
A'=\text{ not admitted}
\]

Now we can start using what is called ``empirical probability'',
because we are given percentages of people that come in matching a
given profile, so we pretend that this is a percentage. I think that's
what empirical probability is. It's not really clear.

So now we have some givens. The percentage of people that come in and
are male are:

\[
P(F')=0.55
\]

The percentage of young people that come in are:

\[
P(Y)=0.60
\]

The percentage of old, female, admitted people are:

\[
P(Y'\cap F\cap A)=0.25
\]

The percentage of young, female people are

\[
P(Y\cap F)=0.20
\]

What we're interested in is the percentage of people that are old,
female, and not admitted:

\[
P(Y'\cap F\cap A') = ?
\]

\subsection*{Actually solving the thing}

Suppose we have a collection of events $\zeta$ with

\[
\zeta = \{F', Y, Y'\cap F\cap A, Y\cap F\}
\]

To use our Weird Formula, we need to have that the intersection of
each event is mutually exclusive. Unfortunately, it's difficult to
ensure that these are true. For example, $Y$ and $Y\cap{}F$ are not
necessarily mutually exclusive.

Suppose we looked at a different set of events. For example, we have
that

\[
Y\cup Y'
\]

Which we know for a fact is the entire sample space, because they are
complements. Similarly, we can look at:

\[
\underset{\Omega}{(Y\cup Y')}\cap\underset{\Omega}{(F\cup F')}\cap\underset{\Omega}{(A\cup A')} = \Omega
\]

Now we can try applying the distributive property to this set,
pretending that $\cup$ is addition, and $\cap$ is multiplication (I
think.)

\[
A_1 = (Y\cap F\cap A)\cup(Y\cap F\cap A')\cup(Y\cap F'\cap A)\cup(Y\cap F'\cap A')
\]

We also have the following, with $Y'$ instead of $Y$.

\[
A_2 = (Y'\cap F\cap A)\cup(Y'\cap F\cap A')\cup(Y'\cap F'\cap A)\cup(Y'\cap F'\cap A')
\]

Since we know that this set of pieces should be mutually exclusive,
because we built them that way. Furthermore, $A_1\cup A_2$ is still
the sample space $\Omega$.

We can show that these are mutually exclusive also by taking any two
sets, for instance
$(Y\cap{}F\cap{}A)\cap(Y'\cap{}F\cap{}A)=\varnothing$.

\section*{Look at distributive laws}

Or something.

\end{document}
