\documentclass{article}
%\usepackage[margin=0.75in]{geometry}
\usepackage{amsmath}
\usepackage{amssymb}
\usepackage{xintexpr}
\usepackage{multicol}
\usepackage{empheq}
\usepackage{forest}
\usepackage{booktabs}
\usepackage{ifthen}
\usepackage{xfrac}

\setlength{\columnsep}{0.5cm}
\setlength{\columnseprule}{1pt}

\newcommand{\T}{1}
\newcommand{\F}{0}
\newcommand{\TF}[1]{\if1#1\T\else\F\fi}
\newcommand{\xintTF}[1]{\xintifboolexpr{#1}{\T}{\F}}

\newcommand{\logicrule}[2]{
\begin{array}{l}
#1 \\
\midrule
\therefore #2 \\
\end{array}
}

\newcommand{\problem}[2]{$\boxed{\text{#1.#2}}$}
\newcommand{\subproblem}[3]{$\boxed{\text{(#3)}}$}
\newcommand{\solution}[3]{\boxed{#3\quad(\text{#1.#2})}}
\newcommand{\subsolution}[4]{\boxed{#4\quad(\text{#1.#2#3})}}

\newcommand{\inv}[1]{#1^{-1}}

\renewcommand{\d}[1]{\,\textnormal{d}#1}
\newcommand{\dd}[2]{\frac{\d{#1}}{\d{#2}}}
\newcommand{\ddd}[2]{\dfrac{\d{#1}}{\d{#2}}}

\setlength\parindent{0pt}
\setlength\parskip{0em}

\begin{document}

\section*{Homework 3}

%
\problem{2.1}{7}

Each event can be represented as a 2-tuple $(x,y)$, where $x$ is the
picture quality, and $y$ is the appearance quality. Each can take on
the values $x,y\in(D=\{p,g,s,f\})$, which stands for perfect, good,
satisfactory, and failed (respectively). Furthermore, we can determine
a score for each one: $C((x,y))$ for piCture quality, and $A((x,y))$
for appearance quality. These are defined as follows:

\[
C((x,y)):(D,D)\rightarrow\mathbb{R}
\] \[
C((x,y))=\begin{cases}4 & \text{if $x=p$} \\
2 & \text{if $x=g$} \\
1 & \text{if $x=s$} \\
0 & \text{if $x=f$}
\end{cases}
\]

and

\[
A((x,y)):(D,D)\rightarrow\mathbb{R}
\] \[
A((x,y))=\begin{cases}3 & \text{if $y=p$} \\
2 & \text{if $y=g$} \\
1 & \text{if $y=s$} \\
0 & \text{if $y=f$}
\end{cases}
\]

Finally, we can attach a total score to each event, $T(\sigma)$,
defined as:

\[
T(\sigma):(D,D)\rightarrow\mathbb{R}
\] \[
T(\sigma)=C(\sigma)\cdot A(\sigma)
\]

%
\subproblem{2.1}{7}{a} To construct the probability mass function, we
can create a table for all the event scores $(x,y)$, where the columns
are the $x$ values, and the rows are the $y$ values. Each entry is the
total score $T((x,y))$

\[
\begin{array}{c|cccc}
  & x=p & x=g & x=s & x=f \\
\midrule
y=p & 12 & 6 & 3 & 0 \\
y=g & 8 & 4 & 2 & 0 \\
y=s & 4 & 2 & 1 & 0 \\
y=f & 0 & 0 & 0 & 0 \\
\end{array}
\]

Then we can determine the range of the discrete random variable $T$,
which is:

\[
T(\sigma):(D,D)\rightarrow\{12,8,6,4,3,2,1,0\}
\]

Then we can construct the inverse image of the function as:

\[
\begin{array}{l|l}
e & \inv{T}\{e\} \\
\midrule
0 & \{(f,p),(f,g),(f,s),(p,f),(g,f),(s,f),(f,f)\} \\
1 & \{(s,s)\} \\
2 & \{(s,g),(g,s)\} \\
3 & \{(s,p)\} \\
4 & \{(g,g),(p,s)\} \\
6 & \{(g,p)\} \\
8 & \{(p,g)\} \\
12 & \{(p,p)\} \\
\end{array}
\]

And the PMF as:

\[
\subsolution{2.1}{7}{a}{\begin{array}{l|l}
e & P_T(e) \\
\midrule
0& 0.061 \\
1& 0.013 \\
2& 0.195 \\
3& 0.067 \\
4& 0.298 \\
6& 0.124 \\
8& 0.102 \\
12& 0.140 \\
\end{array}}
\]

%
\subproblem{2.1}{7}{b} We define the CDF, $F_T(e)$, as:

\[
F_T(e)=\sum\limits_{x\le e}P_T(e)
\]

Using this, we can construct the CDF as follows:

\[
\subsolution{2.1}{7}{b}{\begin{array}{l|l}
e & F_T(e) \\
\midrule
0& 0.061 \\
1& 0.074 \\
2& 0.269 \\
3& 0.336 \\
4& 0.634 \\
6& 0.758 \\
8& 0.860 \\
12& 1.000 \\
\end{array}}
\]

%
\subproblem{2.1}{7}{c} The outcomes that can be shipped are:

\[
\beta=\{(s,g),(g,s),(s,p),(g,g),(p,s),(g,p),(p,g),(p,p)\}
\]

We can interpret the most common score as the one with the highest
probability from the PMF. We note that $\beta$ can also be represented
as:

\[
\beta=\inv{T}\{2,3,4,6,8,12\}
\]

So we can just look at the PMF for each of these values and see that
the most likely score is

\[
\subsolution{2.1}{7}{c}{T(e)=4}
\]

For the second part, we can look at $\beta'$, where:

\[
\beta'=\{(f,p),(f,g),(f,s),(g,f),(s,f),(p,f),(f,f),(s,s)\}
\]

We note that this the same as:

\[
\beta'=\inv{T}\{0,1\}
\]

We can then relate this to the PMF via:

\[
P(\beta')=P(\inv{T}\{0,1\})
\] \[
P(\beta')=\sum\limits_{z\le1}P_T(z)
\]

But this is exactly the same as the CDF for $1$:

\[
\subsolution{2.1}{7}{c}{P(\beta')=F_T(1)}
\]

%
\problem{2.1}{9}

We are told the first value of the PMF, $P_X(1)$, is
$\frac{2}{5}$. Similarly, we can deduce that the value of $P_X(0)$ is
$0$, because there's no way we can pick the right warehouse if we
never call any of them.

We can construct a tree as follows, where the events to the left are
that the item is in the warehouse we call. We also introduce the
events $C_i$ for $i\in\{1,2,3,4\}$, which stand for the event that
call $i$ succeeds, and relate it to the PMF by $P_X(i)=P(C_i)$.

\begin{center}
\begin{forest}
  for tree={
    l sep=0.5cm,
    s sep=3cm,
    delay={
      edge label/.wrap value={node[midway,fill=white]{#1}}
    }
  }
  [$\circ$
    [$\dfrac{2}{5}$, edge label={$C_1$}]
    [$\dfrac{3}{5}$, edge label={$C_1'$}
      [$\dfrac{2}{4}$, edge label={$C_2$}]
      [$\dfrac{2}{4}$, edge label={$C_2'$}
        [$\dfrac{2}{3}$, edge label={$C_3$}]
        [$\dfrac{1}{3}$, edge label={$C_3'$}
          [$\dfrac{2}{2}$, edge label={$C_4$}]
          [$0$, edge label={$C_4'$}]
        ]
      ]
    ]
  ]
\end{forest}
\end{center}

The probability of the first one can be computed trivially, as just
the value of the left node from the top (abbreviated as the path L).

\[
P_X(1)=P(A)
\] \[
P_X(1)=\frac{2}{5}
\]

We can then compute the probability of each remaining event using the
law of total probability. If we let $\zeta$ be the set:

\[
\zeta=\{C_1,C_1'\}
\]

Then all the elements in $\zeta$ are mutually exclusive, and the next
event, $C_2$, is fully contained within the union of each element in
$\zeta$. In other words:

\[
\zeta\subset\bigcup\limits_{A\in\zeta}A
\]

Therefore, we can use the law of total probability to determine:

\[
P(C_2)=\sum\limits_{A\in\zeta}P(C_2|A)P(A)
\] \[
P(C_2)=\underbrace{P(C_2|C_1)P(C_1)}_{z_1}+\underbrace{P(C_2|C_1')P(C_1')}_{z_2}
\]

For the first part $z_1$, we recognize that the event that the second
call succeeds or fails cannot occur if the first call
succeeds. Therefore, we know that

\[
P(C_2|C_1)=0
\]

The second part, $z_2$, can be computed using the tree above, by
multiplying the fractions as we travel along that tree. We would
follow the right branch, and then the left (i.e. RL), so we would
have:

\[
P(C_2|C_1')P(C_1')=\frac{3}{5}\cdot\frac{2}{4}=\frac{3}{10}
\]

Finally, we can substitute these back into the original expression
using $z_1$ and $z_2$ to get:

\[
P(C_2)=0+\frac{3}{10}
\] \[
P(C_2)=\frac{3}{10}
\]

Using this process, we can compute the rest of the probabilities as
follows:

\[
\begin{array}{l|l|l}
C_i & \text{Expression} & P(C_2) \\
\midrule
3 & \dfrac{3\cdot2\cdot2}{5\cdot4\cdot3} & \dfrac{1}{5} \\
\midrule
4 & \dfrac{3\cdot2\cdot1\cdot2}{5\cdot4\cdot3\cdot2} & \dfrac{1}{10} \\
\midrule
5 & \dfrac{3\cdot2\cdot1\cdot0}{5\cdot4\cdot3\cdot1} & 0 \\
\end{array}
\]

Finally, we can use these to define the PMF and CDF as:

\[
\solution{2.1}{9}{\begin{array}{l|l|l}
i & P_X(i) & F_X(i) \\
\midrule
0 & 0 & 0 \\
1 & \sfrac{2}{5} & \sfrac{2}{5} \\
2 & \sfrac{3}{10} & \sfrac{7}{10} \\
3 & \sfrac{1}{5} & \sfrac{9}{10} \\
4 & \sfrac{1}{10} & 1 \\
5 & 0 & 1 \\
\end{array}}
\]

%
\problem{2.1}{10}

Let $X$ be a random variable, which can take on any value in the
positive integers $\mathbb{Z}^+=\{1,2,\cdots\}$.

%
\subproblem{2.1}{10}{a} Suppose the PMF is defined, for some constant
$c$, as:

\[
P(X=i)=\frac{c}{i^2}
\] \[
P_X(i)=\frac{c}{i^2}
\]

We want to determine if this is a possible definition of the PMF. One
way to check is to look at the corresponding CDF:

\[
F_X(i)=\sum\limits_{z\le{}i}P_X(z)
\] \[
F_X(i)=\lbrack i\ge1\rbrack\sum\limits_{z=1}^i P_X(z)
\]

We know that the CDF should be $0$ for anything less than the minimum
value of the range of $i$. Suppose $k$ is an arbitrary integer less
than $1$. The value $F_X(k)$ can be found by:

\[
F_X(k)=\lbrack k\ge 1\rbrack\sum\limits_{z=1}^i P_X(z)
\]

Because we chose $k$ such that it is less than $1$, the Iverson
bracket at the beginning evaluates to $0$, so the CDF also evaluates
to $0$.

\[
F_X(k)=0
\]

We also know that the CDF should be $1$ on the other end of the range,
greater than the maximum of the range. Suppose we let $n$ be an
integer greater than $1$. The CDF for $n$ would be:

\[
F_X(n)=\lbrack n\ge 1\rbrack\sum\limits_{z=1}^n P_X(z)
\] \[
F_X(n)=\sum\limits_{z=1}^n P_X(z)
\]

By substituting the value for the PDF in question, we get:

\[
F_X(n)=\sum\limits_{z=1}^n\frac{c}{z^2}
\]

We can then take a limit to find the value of this sum as $n$ goes to
infinity:

\[
\lim\limits_{n\rightarrow\infty}F_X(n)=\lim\limits_{n\rightarrow\infty}\sum\limits_{z=1}^n\frac{c}{z^2}
\] \[
\lim\limits_{n\rightarrow\infty}F_X(n)=c\sum\limits_{z=1}^\infty\frac{1}{z^2}
\]

We know that the infinite sum of inverse squares converges, so if we
let $v$ equal that sum, then we have:

\[
\lim\limits_{n\rightarrow\infty}F_X(n)=cv
\]

If we chose $c=v^{-1}$, then this means that the CDF equals $1$ at
infinity, which means
$\subsolution{2.1}{10}{a}{\text{the PMF is valid}}$.

%
\subproblem{2.1}{10}{b} Now we want to look at the possibility of the
PDF being, for some constant c:

\[
P_X(i)=\frac{c}{i}
\]

By the same logic we used above, we will arrive at the infinite sum:

\[
\lim\limits_{n\rightarrow\infty}F_X(n)=c\sum\limits_{z=1}^\infty \frac{1}{i}
\]

However, we know that the infinite sum of inverse integers
diverges. Therefore, no matter what $c$ we choose, this sum will never
converge to $1$, so $\subsolution{2.1}{10}{b}{\text{this PMF is not valid}}$

%
\problem{2.1}{11}

Let the random variable $f$ be the number of appointments up to and
including when Monday's schedule is filled. Let $M$ and $T$ represent
whether a particular appointment is scheduled for a Monday or a
Tuesday, respectively.

\subproblem{2.1}{11}{a} We can define the sample space as:

\[
\Omega=\left\{\langle a_1,a_2,\cdots,a_6\rangle\middle| a_i\in\{M,T\}, 3=\sum\limits_i\lbrack a_i=M\rbrack, 3=\sum\limits_i\lbrack a_i=T\rbrack\right\}
\]

For simplicity, we let $M=1$ and $T=0$, so now our sample space (aka,
our state space) is:

\[
\subsolution{2.1}{11}{a}{\Omega=\left\{\langle a_1,a_2,\cdots,a_6\rangle\middle| a_i\in\{M,T\},3=\sum\limits_ia_i\right\}}
\]

\subproblem{2.1}{11}{b} To compute the PDF (and also the CDF), we need
to look at what our random variable actually means. We note that, if a
particular day $k$ is the total number of appointments up to and
including the one that filled Monday's schedule, then it must be that:

\[
3\le k\le 6
\]

Furthermore, for a schedule $\langle{}a_1,a_2,\cdots,a_6\rangle$, we
know that the $k$th appointment must be for a Monday
(i.e. $a_k=1$). We also note that $k$ is the output of the random
variable:

\[
k=f(\langle a_1,a_2,\cdots,a_6\rangle)
\]

Because we are told that elementary events, $\{a_1,a_2,\cdots,a_6\}$,
are equally likely, it must be for any subset of the sample space,
$A$, that:

\[
P(A)=\frac{|A|}{|\Omega|}
\]

If we let $A=\inv{f}\{k\}$, then we have:

\[
P(\inv{f}\{k\})=\frac{|\inv{f}\{k\}|}{|\Omega|}
\]

However, the left side is the same as the PMF of $k$:

\[
P_f(k)=\frac{|\inv{f}\{k\}|}{|\Omega|}
\]

We can easily determine the size of $\Omega$ by realizing that we
could create a set $\zeta=\{1,2,\cdots,6\}$ and then randomly choose 3
(where order does not matter). This corresponds to a combination and
can be computed via:

\[
|\Omega|=\binom{6}{3}
\]

Now we just need to determine how many items are in the set
$\inv{f}\{k\}$, or in other words, how many different schedules there
are when it takes $k$ days to fill Monday's schedule. We can do this
by looking back at our sample space and noting that, if $a_k$ is the
last 1 in the tuple, then the summation can be simplified:

\[
3=\sum\limits_{i=1}^6a_i
\] \[
3=\sum\limits_{i=1}^ka_i
\]

And furthermore, since $a_k=1$:

\[
2=\sum\limits_{i=1}^{k-1}a_i
\]

Again, we can let $\zeta=\{1,2,\cdots,k-1\}$ and randomly choose 2
elements, corresponding to binomial coefficients, and allowing us to
determine a value for $|\inv{f}\{k\}|$:

\[
|\inv{f}\{k\}|=\binom{k-1}{2}
\]

Finally, we can go back to the earlier equation for $P_f(k)$:

\[
P_f(k)=\dfrac{\binom{k-1}{2}}{\binom{6}{3}}
\] \[
P_f(k)=\frac{(k-1)!}{(k-3)!2!}\cdot\frac{3!3!}{6!}
\]

This gives us the PMF for any $3\le{}k\le6$. For $1\le{}k\le2$ or
$7\le{}k$, the PMF would have the value $0$.

\[
\subsolution{2.1}{11}{b}{\begin{array}{l|l|l}
k & P_f(k) & F_f(k) \\
\midrule
1 & 0 & 0 \\
2 & 0 & 0 \\
3 & \sfrac{1}{20} & \sfrac{1}{20} \\
4 & \sfrac{3}{20} & \sfrac{4}{20} \\
5 & \sfrac{6}{20} & \sfrac{10}{20} \\
6 & \sfrac{10}{20} & 1 \\
7 & 0 & 1 \\
\end{array}}
\]

%
\problem{2.2}{6c}

Let the random variable $X$ be continuous. We also have the
probability density function defined, for some value of $A$, as:

\[
f(x)=A(0.5-(x-0.25)^2)
\]

for $x\in\lbrack0.125,0.5\rbrack$, and $f(x)=0$ for any other value of
$x$ outside of that range. To determine the value of $A$, we need to
look at the CDF, defined as:

\[
F_X(x)=\int\limits_{-\infty}^xf(x)\d{x}
\]

It is helpful to look at value of the CDF at infinity:

\[
F_X(\infty)=\int\limits_{-\infty}^\infty f(x)\d{x}
\]

We note that $f(x)$ is only defined between $0.125$ and $0.5$, so we
can simplify the integral to:

\[
F_X(\infty)=\int\limits_{0.125}^{0.5} f(x)\d{x}
\] \[
F_X(\infty)=-\frac{A x^3}{3}+\frac{A x^2}{4}+\frac{7 A x}{16}
\]

We know that the CDF at infinity should be equal to 1, so we can solve
for $A$ and get:

\[
A=5.50538
\]

To find the probability that the paint thickness is less than $0.2$mm,
then we are looking for the value of:

\[
\int\limits_{x\le0.2} f(x)\d{x}
\]

However, this is equivalent to the CDF of $0.2$ (i.e. $F_X(0.2)$):

\[
F_X(0.2)=\int\limits_{-\infty}^{0.2}f(x)\d{x}
\]

We can simplify this with the reasoning that $f(x)=0$ for $x\le0.125$:

\[
F_X(0.2)=\int\limits_{0.125}^{0.2}f(x)\d{x}
\]

We can calculate this value directly, based on the value of $A$ we
calculated earlier:

\[
\solution{2.2}{6c}{F_X(0.2)=0.203097}
\]

%
\problem{2.2}{7b}

Let $X$ be a random continuous variable, with a CDF defined as:

\[
F_X(x)=\begin{cases}A+B\ln(3x+2) & \text{if $0\le{}x\le10$} \\ 0 & \text{otherwise}\end{cases}
\]

We can also compute PDF as:

\[
\rho_X(x)=\dd{F_X(x)}{x}
\] \[
\rho_X(x)=\begin{cases}B\frac{3}{3x+2} & \text{if $0\le{}x\le10$} \\ 0 & \text{otherwise}\end{cases}
\]

We can also simplify $\rho_X(x)$ using an Iverson bracket:

\[
\rho_X(x)=\lbrack0\le x\le 10\rbrack B\frac{3}{3x+2}
\]


From here, we can use the integral equation for the CDF using the PDF
we just calculated, and evaluate it at infinity (where the CDF should
be 1). In other words, we are solving the following for $B$.

\[
1=\int\limits_{-\infty}^\infty\rho_X(x)\d{x}
\]

We can substitute the value for $\rho_X(x)$ we calculated earlier:

\[
1=\int\limits_{-\infty}^\infty \lbrack0\le x\le 10\rbrack B\frac{3}{3x+2}
\]

Because of the Iverson bracket, we can simplify the limits of the
integral and factor out the $B$ term:

\[
1=B\int\limits_{0}^{10}\frac{3}{3x+2}
\] \[
B=\frac{1}{\ln(16)}
\]

For the remainder of the problem, we won't need to know the value for
$A$, so we'll leave it uncalculated. We want to find the probability
that a job takes longer than two hours (i.e. $P(x\ge2)$). If we let
$\alpha(x)=``x\ge2''$, then we want to find:

\[
P(x\ge2)=\int\limits_{\alpha(x)}\rho_X(x)\d{x}
\]

We can expand this using an Iverson bracket:

\[
P(x\ge2)=\int\limits_{-\infty}^\infty\lbrack\alpha(x)\rbrack\rho_X(x)\d{x}
\]

Then simplify the range using the known value for $\alpha(x)$ and the
known values for $\rho_X(x)$:

\[
P(x\ge2)=\int\limits_{2}^{10}\rho_X(x)\d{x}
\]

We could expand this into two integrals:

\[
P(x\ge2)=\int\limits_{0}^{10}\rho_X(x)\d{x}-\int\limits_{0}^{2}\rho_X(x)\d{x}
\]

However, this is just the value of the CDF at $10$ minus the CDF at
$2$:

\[
P(x\ge2)=F_X(10)-F_X(2)
\] \[
P(x\ge2)=A+B\ln(3\cdot10+2)-A-B\ln(3\cdot2+2)
\] \[
P(x\ge2)=\frac{\ln(32)-\ln(8)}{\ln(16)}
\] \[
\solution{2.2}{7b}{P(x\ge2)=0.5}
\]

%
\problem{2.2}{8c}

Let $\theta$ be a random continuous variable. The PDF of $\theta$ is
defined, for some value of $A$, as:

\[
f(\theta)=\lbrack0\le\theta\le10\rbrack A(e^{10-\theta}-1)
\]

To determine the value of $A$, we can look at the CDF defined as:

\[
F(\theta)=\int\limits_{-\infty}^\theta f(\phi)\d{\phi}
\]

We know that the CDF should be 1 at infinity, so we can solve for $A$:

\[
F(\infty)=\int\limits_{-\infty}^\infty f(\theta)\d{\theta}
\] \[
F(\infty)=\int\limits_{-\infty}^\infty \lbrack0\le\theta\le10\rbrack A(e^{10-\theta}-1)\d{\theta}
\]

Because of the Iverson bracket, we can simplify the limits of our
integral:

\[
F(\infty)=\int\limits_{0}^{10} A(e^{10-\theta}-1)\d{\theta}
\]

And factor out the $A$ and solve:

\[
F(\infty)=A\int\limits_{0}^{10} e^{10-\theta}-1\d{\theta}
\] \[
1=A\int\limits_{0}^{10} e^{10-\theta}-1\d{\theta}
\] \[
A=\frac{1}{e^{10}-1}
\]

The probability we want to find is the probability $P(\alpha(\theta))$
for $\alpha(\theta)=``0\le\theta\le8''$. We can compute this as:

\[
P(\alpha(\theta))=\int\limits_{\alpha(\theta)}f(\theta)\d{\theta}
\]

Which we can expand using an Iverson bracket:

\[
P(\alpha(\theta))=\int\limits_{-\infty}^\infty \lbrack0\le\theta\le8\rbrack f(\theta)\d{\theta}
\]

We can simplify the limits then:

\[
P(\alpha(\theta))=\int\limits_{0}^{8} f(\theta)\d{\theta}
\]

We can directly compute this value:

\[
\solution{2.2}{8c}{P(\alpha(\theta))=0.999710}
\]

%
\problem{2.2}{11}

Let $X$ be a continuous random variable with PDF defined, for some
$A$, as:

\[
f(x)=\lbrack10\le x\le11\rbrack Ax(130-x^2)
\]

\subproblem{2.2}{11}{a} We can determine the value of $A$ using the
CDF at infinity:

\[
\begin{array}{l}
1=\int\limits_{-\infty}^\infty f(x)\d{x} \\
1=\int\limits_{10}^{11} Ax(130-x^2)\d{x} \\
1=A\int\limits_{10}^{11} x(130-x^2)\d{x}
\end{array}
\]

\[
\subsolution{2.2}{11}{a}{A=\dfrac{4}{819}}
\]

\subproblem{2.2}{11}{b} We can then compute the CDF using the integral
form:

\[
F(x)=\int\limits_{-\infty}^x f(z)\d{z}
\]

We can then substitute $f$:

\[
F(x)=\int\limits_{-\infty}^x \lbrack10\le z\le11\rbrack Az(130-z^2)
\]

We can factor out the Iverson bracket if we limit the range of the
integral:

\[
F(x)=\lbrack10\le x\le11\rbrack A\int\limits_{10}^x z(130-z^2)
\]

If we substitute the known value for $A$ we get the CDF:

\[
\subsolution{2.2}{11}{b}{F(x)=\lbrack10\le x\le11\rbrack \frac{4}{819}\int\limits_{10}^x z(130-z^2)}
\]

\subproblem{2.2}{11}{c} Let $\alpha(x)=``10.25\le{}x\le10.5''$. We
want to find $P(\alpha(x))$. By definition, this is:

\[
\begin{array}{l}
P(\alpha(x))=\int\limits_{\alpha(x)}f(x)\d{x} \\
P(\alpha(x))=\int\limits_{-\infty}^\infty \lbrack\alpha(x)\rbrack f(x)\d{x} \\
P(\alpha(x))=\int\limits_{10.25}^{10.5} f(x)\d{x} \\
\end{array}
\]

We can expand this last expression into two integrals:

\[
P(\alpha(x))=\int\limits_{10}^{10.5} f(x)\d{x}-\int\limits_{10}^{10.25} f(x)\d{x}
\]

These integrals represent the values $F(10.5)$ and $F(10.25)$
respectively:

\[
P(\alpha(x))=F(10.5)-F(10.25)
\]

This is something we can now compute directly, to find:

\[
\subsolution{2.2}{11}{c}{P(\alpha(x))=0.283048}
\]





\end{document}
