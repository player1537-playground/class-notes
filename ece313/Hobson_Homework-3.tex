\documentclass{article}
%\usepackage[margin=0.75in]{geometry}
\usepackage{amsmath}
\usepackage{amssymb}
\usepackage{xintexpr}
\usepackage{multicol}
\usepackage{empheq}
\usepackage{forest}
\usepackage{booktabs}

\setlength{\columnsep}{0.5cm}
\setlength{\columnseprule}{1pt}

\newcommand{\T}{1}
\newcommand{\F}{0}
\newcommand{\TF}[1]{\if1#1\T\else\F\fi}
\newcommand{\xintTF}[1]{\xintifboolexpr{#1}{\T}{\F}}

\newcommand{\logicrule}[2]{
\begin{array}{l}
#1 \\
\midrule
\therefore #2 \\
\end{array}
}

\newcommand{\problem}[2]{$\boxed{\text{#1.#2}}$}
\newcommand{\subproblem}[3]{$\boxed{\text{(#3)}}$}
\newcommand{\solution}[3]{\boxed{#3\quad(\text{#1.#2})}}
\newcommand{\subsolution}[4]{\boxed{#4\quad(\text{#1.#2#3})}}

\newcommand{\inv}[1]{#1^{-1}}

\renewcommand{\d}[1]{\,\textnormal{d}#1}
\newcommand{\dd}[2]{\frac{\d{#1}}{\d{#2}}}
\newcommand{\ddd}[2]{\dfrac{\d{#1}}{\d{#2}}}

\setlength\parindent{0pt}
\setlength\parskip{0em}

\begin{document}

\section*{Homework 3}

%
\problem{2.1}{7}

Each event can be represented as a 2-tuple $(x,y)$, where $x$ is the
picture quality, and $y$ is the appearance quality. Each can take on
the values $x,y\in(D=\{p,g,s,f\})$, which stands for perfect, good,
satisfactory, and failed (respectively). Furthermore, we can determine
a score for each one: $C((x,y))$ for piCture quality, and $A((x,y))$
for appearance quality. These are defined as follows:

\[
C((x,y)):(D,D)\rightarrow\mathbb{R}
\] \[
C((x,y))=\begin{cases}4 & \text{if $x=p$} \\
2 & \text{if $x=g$} \\
1 & \text{if $x=s$} \\
0 & \text{if $x=f$}
\end{cases}
\]

and

\[
A((x,y)):(D,D)\rightarrow\mathbb{R}
\] \[
A((x,y))=\begin{cases}3 & \text{if $y=p$} \\
2 & \text{if $y=g$} \\
1 & \text{if $y=s$} \\
0 & \text{if $y=f$}
\end{cases}
\]

Finally, we can attach a total score to each event, $T(\sigma)$,
defined as:

\[
T(\sigma):(D,D)\rightarrow\mathbb{R}
\] \[
T(\sigma)=C(\sigma)\cdot A(\sigma)
\]

%
\subproblem{2.1}{7}{a} To construct the probability mass function, we
can create a table for all the event scores $(x,y)$, where the columns
are the $x$ values, and the rows are the $y$ values. Each entry is the
total score $T((x,y))$

\[
\begin{array}{c|cccc}
  & x=p & x=g & x=s & x=f \\
\midrule
y=p & 12 & 6 & 3 & 0 \\
y=g & 8 & 4 & 2 & 0 \\
y=s & 4 & 2 & 1 & 0 \\
y=f & 0 & 0 & 0 & 0 \\
\end{array}
\]

Then we can determine the range of the discrete random variable $T$,
which is:

\[
T(\sigma):(D,D)\rightarrow\{12,8,6,4,3,2,1,0\}
\]

Then we can construct the inverse image of the function as:

\[
\begin{array}{l|l}
e & \inv{T}\{e\} \\
\midrule
0 & \{(f,p),(f,g),(f,s),(p,f),(g,f),(s,f),(f,f)\} \\
1 & \{(s,s)\} \\
2 & \{(s,g),(g,s)\} \\
3 & \{(s,p)\} \\
4 & \{(g,g),(p,s)\} \\
6 & \{(g,p)\} \\
8 & \{(p,g)\} \\
12 & \{(p,p)\} \\
\end{array}
\]

And the PMF as:

\[
\subsolution{2.1}{7}{a}{\begin{array}{l|l}
e & P_T(e) \\
\midrule
0& 0.061 \\
1& 0.013 \\
2& 0.195 \\
3& 0.067 \\
4& 0.298 \\
6& 0.124 \\
8& 0.102 \\
12& 0.140 \\
\end{array}}
\]

%
\subproblem{2.1}{7}{b} We define the CDF, $F_T(e)$, as:

\[
F_T(e)=\sum\limits_{x\le e}P_T(e)
\]

Using this, we can construct the CDF as follows:

\[
\subsolution{2.1}{7}{b}{\begin{array}{l|l}
e & F_T(e) \\
\midrule
0& 0.061 \\
1& 0.074 \\
2& 0.269 \\
3& 0.336 \\
4& 0.634 \\
6& 0.758 \\
8& 0.860 \\
12& 1.000 \\
\end{array}}
\]

%
\problem{2.1}{9}

%
\problem{2.1}{10}

%
\problem{2.1}{11}

%
\problem{2.2}{6c}

%
\problem{2.2}{7b}

%
\problem{2.2}{8c}

%
\problem{2.2}{11}


\end{document}
