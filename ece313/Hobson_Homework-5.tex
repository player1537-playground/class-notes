\documentclass{article}
%\usepackage[margin=0.75in]{geometry}
\usepackage{amsmath}
\usepackage{amssymb}
\usepackage{xintexpr}
\usepackage{multicol}
\usepackage{empheq}
\usepackage{forest}
\usepackage{booktabs}
\usepackage{ifthen}
\usepackage{xfrac}

\setlength{\columnsep}{0.5cm}
\setlength{\columnseprule}{1pt}

\newcommand{\T}{1}
\newcommand{\F}{0}
\newcommand{\TF}[1]{\if1#1\T\else\F\fi}
\newcommand{\xintTF}[1]{\xintifboolexpr{#1}{\T}{\F}}

\newcommand{\logicrule}[2]{
\begin{array}{l}
#1 \\
\midrule
\therefore #2 \\
\end{array}
}

\newcommand{\problem}[2]{$\boxed{\text{#1.#2}}$}
\newcommand{\subproblem}[3]{$\boxed{\text{(#3)}}$}
\newcommand{\solution}[3]{\boxed{#3\quad(\text{#1.#2})}}
\newcommand{\subsolution}[4]{\boxed{#4\quad(\text{#1.#2#3})}}

\newcommand{\inv}[1]{#1^{-1}}

\renewcommand{\d}[1]{\,\textnormal{d}#1}
\newcommand{\dd}[2]{\frac{\d{#1}}{\d{#2}}}
\newcommand{\ddd}[2]{\dfrac{\d{#1}}{\d{#2}}}

\newcommand{\p}[1]{\,\partial #1}
\newcommand{\pp}[2]{\frac{\p{#1}}{\p{#2}}}
\newcommand{\ppp}[2]{\dfrac{\p{#1}}{\p{#2}}}

\DeclareMathOperator{\var}{Var}
\DeclareMathOperator{\E}{\mathcal{E}}

\setlength\parindent{0pt}
\setlength\parskip{0em}

\begin{document}

\section*{Homework 5}

%
\problem{2.6}{8}

The random variable $X$ has PDF:

\[
\rho_X(x)=\lbrack0\le x\le1\rbrack2x
\]

With a CDF of:

\[
F_X(x)=\lbrack0\le x\le1\rbrack x^2
\]

%
\subproblem{2.6}{8}{a} Suppose $Y=X^3$. The CDF for $Y$ is given by:

\[
\begin{array}{l}
F_Y(y)=P(Y\le y) \\
F_Y(y)=P(X^3\le y) \\
F_Y(y)=P(X\le \sqrt[3]{y}) \\
\end{array}
\]

But this value is just $F_X(\sqrt[3]{y})$, which we have a known value
for:

\[
F_Y(y)=F_X(\sqrt[3]{y})=y^{\sfrac{2}{3}}
\]

We can then take a derivative to get the PDF:

\[
\begin{array}{ll}
\rho_Y(y) &=\ddd{F_Y(y)}{y} \\
&=\ddd{}{y} y^{\sfrac{2}{3}} \\
\end{array}
\]

\[
\rho_Y(y)=\frac{2}{3}\cdot y^{-1/3}
\]

The expectation is then found via:

\[
\E Y=\int y \rho_Y(y)\d{y}
\] \[
\subsolution{2.6}{8}{a}{\begin{array}{l}
\rho_Y(y)=\frac{2}{3}\cdot y^{-1/3} \\
\E Y=\frac{2}{5} \\
\end{array}}
\]

%
\subproblem{2.6}{8}{b} Suppose $Y=\sqrt{X}$. The CDF is given by:

\[
\begin{array}{ll}
F_Y(y)&=P(Y\le y) \\
&=P(\sqrt{X}\le y) \\
&=P(X\le y^2) \\
&=F_X(y^2) \\
\end{array}
\]

We can then use the known value for $F_X(y^2)$:

\[
F_Y(y)=y^4
\]

The derivative is the PDF:

\[
\begin{array}{ll}
\rho_Y(y)&=\dd{}{y} y^4 \\
&=4y^3 \\
\end{array}
\]

Then the expectation is defined as:

\[
\begin{array}{ll}
\E Y&=\int y\rho_Y(y) \d{y} \\
\end{array}
\]

\[
\subsolution{2.6}{8}{b}{\begin{array}{l}
\rho_Y(y)=4y^3 \\
\E Y=\sfrac{4}{5}
\end{array}}
\]

%
\subproblem{2.6}{8}{c} Suppose $Y=1/(1+X)$. The CDF can be written as:

\[
\begin{array}{rl}
F_Y(y)&=P(Y\le y) \\
&=P\left(\frac{1}{1+X}\le y\right) \\
\end{array}
\]

Because $X$ is always positive, then $1+X$ will also always be
positive, so we can multiply both sides of the inequality by $1+X$
without needing to reverse the direction of the inequality sign.

\[
\begin{array}{rl}
F_Y(y)&=P(1\le y(1+X)) \\
&=P(1\le y+yX) \\
&=P(1-y\le yX) \\
\end{array}
\]

Now we want to divide by $y$ on each side, to get $X$ by
itself. However, we might run into an issue where $y$ is negative or
zero. We can look at this logic split into cases. In the case that $y$
is zero, the expression $\frac{1-y}{y}$ is undefined, so we can
exclude it from the other cases.

\[
F_Y(y)=\begin{cases}
P\left(\frac{1-y}{y}\le X\right) &\text{if $y>0$} \\
P\left(\frac{1-y}{y}\ge X\right) &\text{if $y<0$} \\
\end{cases}
\]

We can rearrange the expressions so that we can write this in terms of
$F_X(\text{something})$.

\[
F_Y(y)=\begin{cases}
1-F_X\left(\frac{1-y}{y}\right) &\text{if $y>0$} \\
F_X\left(\frac{1-y}{y}\right) &\text{if $y<0$} \\
\end{cases}
\]

Now, if we look at the original definition for $F_X$, we can rule out
some of the cases as being zero:

\[
F_Y(y)=\begin{cases}
1-\lbrack0\le \frac{1-y}{y}\le 1\rbrack \left(\frac{1-y}{y}\right)^2  &\text{if $y>0$} \\
\lbrack0\le \frac{1-y}{y}\le 1\rbrack \left(\frac{1-y}{y}\right)^2 &\text{if $y<0$} \\
\end{cases}
\]

We can simplify the Iverson bracket as:

\[
\begin{array}{rl}
\lbrack 0\le \frac{1-y}{y}\le 1\rbrack &= \begin{cases}
\lbrack 0\le 1-y \le y\rbrack &\text{if $y>0$} \\
\lbrack 0\ge 1-y \ge y\rbrack &\text{if $y<0$} \\
\end{cases} \\

&=\begin{cases}
\lbrack y\le 1 \le 2y\rbrack &\text{if $y>0$} \\
\lbrack y\ge 1 \ge 2y\rbrack &\text{if $y<0$} \\
\end{cases} \\
\end{array}
\]

The second case will lead the Iverson bracket to a value of zero,
because we assume $y<0$, but the statement checks that $y\ge1$, which
cannot both be true. In the first case, we can further split this into
two Iverson brackets:

\[
\begin{array}{rl}
\lbrack 0\le \frac{1-y}{y}\le 1\rbrack &= \begin{cases}
\lbrack y\le 1\rbrack \lbrack 1\le 2y\rbrack & \text{if $y>0$} \\
0 &\text{if $y<0$} \\
\end{cases} \\

&=\begin{cases}
\lbrack y\le 1\rbrack \lbrack \sfrac{1}{2}\le y\rbrack & \text{if $y>0$} \\
0 &\text{if $y<0$} \\
\end{cases} \\

&=\begin{cases}
\lbrack \sfrac{1}{2} \le y \le 1 \rbrack & \text{if $y>0$} \\
0 &\text{if $y<0$} \\
\end{cases} \\
\end{array}
\]

Substituting these expressions into the previously calculated value of
$F_Y$ gives us:

\[
F_Y(y)=\begin{cases}
1-\lbrack \sfrac{1}{2}\le y\le 1\rbrack \left(\frac{1-y}{y}\right)^2  &\text{if $y>0$} \\
0 &\text{if $y<0$} \\
\end{cases}
\]

We can rewrite this as:

\[
F_Y(y)=\lbrack y\ge 0\rbrack \left(1 -\lbrack\sfrac{1}{2}\le y\le1\rbrack \left(\frac{1-y}{y}\right)^2\right)
\]

Now we can differentiate this on the interval of
$\lbrack\sfrac{1}{2},1\rbrack$ to find the value of the PDF for $Y$:

\[
\begin{array}{rl}
\rho_Y(y)&=\dd{}{z} F_Y(z) \\
&=\lbrack \sfrac{1}{2}\le y\le1\rbrack \frac{2-2z}{z^3} \\
\end{array}
\]

And the expectation:

\[
\E Y=\int z \rho_Y(z)\d{z}
\]

\[
\subsolution{2.6}{8}{c}{
\begin{array}{l}
\rho_Y(y)=\lbrack \sfrac{1}{2}\le y\le1\rbrack \frac{2-2z}{z^3} \\
\E Y=0.61371 \\
\end{array}}
\]

%
\subproblem{2.6}{8}{d} Suppose $Y=2^X$.

%
\problem{2.6}{10}

We are given the PDF:

\[
\rho_X(x)=\lbrack0\le x\le L\rbrack Ax(L-x)
\]

%
\subproblem{2.6}{10}{a} The value for $A$ can be found by solving the
integral:

\[
1=\int \rho_X(z)\d{z}
\]

Which gives:

\[
\subsolution{2.6}{10}{a}{A=\frac{6}{L^3}}
\]

%
\subproblem{2.6}{10}{b} The random variable describing the difference
between the lengths of the two segments is given by:

\[
\begin{array}{rl}
D&=|X - (L - X)| \\
&=|2X - L| \\
\end{array}
\]

We can compute the density by observing that:

\[
\rho_D(x)=\pp{}{x} F_D(x) = \pp{}{x}P(D\le x) = \pp{}{x}P(|2X-L|\le x)
\]

We can solve the inequality for $X$ to get:

\[
\begin{array}{rl}
|2X-L|\le x &\equiv -x \le 2X-L\le x \\
&\equiv -x+L\le 2X\le x+L \\
&\equiv \frac{-x+L}{2}\le X\le \frac{x+L}{2} \\
\end{array}
\]

We can rewrite the expression for $\rho_D(x)$ using an integral:

\[
\begin{array}{rl}
\rho_D(x)&=\pp{}{x} \int_{\frac{-x+L}{2}}^{\frac{x+L}{2}} \rho_X(z)\d{z} \\
&=\pp{}{x} \int_{\frac{-x+L}{2}}^{\frac{x+L}{2}} \lbrack0\le z\le L\rbrack Az(L-z)\d{z} \\
\end{array}
\]

Using Leibniz integral rule, this is equivalent to:

\[
\rho_D(x)=\left(A\frac{x+L}{2}\left(L-\frac{x+L}{2}\right)\right)\cdot\frac{1}{2} - \left(A\frac{-x+L}{2}\left(L-\frac{-x+L}{2}\right)\right)\cdot\frac{-1}{2}
\]

Which simplifies to:

\[
\subsolution{2.6}{10}{b}{\rho_D(x)=\dfrac{3\left(L^2-x^2\right)}{2L^3}}
\]

%
\subproblem{2.6}{10}{c} The expectation can be calculated directly
from the definition of the random variable $D$:

\[
\begin{array}{rl}
\E D&=\E|2X-L| \\
&=A\int_0^L |2x-L| \rho_X(x)\d{x} \\
\end{array}
\]

We can also verify the answer against the value we found for the PDF
of $D$:

\[
\E D =\int\limits_0^L \rho_D(x)\d{x}
\]

\[
\subsolution{2.6}{10}{c}{\E D=\frac{3L}{8}}
\]

%
\problem{2.6}{11}

We are given the expectation and standard deviation of funds $A$ and
$B$:

\[
\begin{array}{cc}
\E A_x=0.1 x & \sqrt{\var(A_x)}=0.02x \\
\E B_x=0.1 x & \sqrt{\var(B_x)}=0.03x \\
\end{array}
\]

The total amount of money available to lend, $T$, is:

\[
T=1000
\]

%
\subproblem{2.6}{11}{a} Investing all money in $A$ gives an
expectation and standard deviation of:

\[
\subsolution{2.6}{11}{a}{\begin{array}{cc}
A_{1000} = 100 & \sqrt{\var(A_{1000})} = 20 \\
\end{array}}
\]

%
\subproblem{2.6}{11}{b} Investing all money in $B$ gives an
expectation and standard deviation of:

\[
\subsolution{2.6}{11}{b}{\begin{array}{cc}
B_{1000} = 100 & \sqrt{\var(B_{1000})} = 30 \\
\end{array}}
\]

%
\subproblem{2.6}{11}{c} Investing $50\%$ of the total money in each,
gives an expectation of:

\[
\begin{array}{rl}
\E(A_{500}+B_{500}) &= \E A_{500} + \E B_{500} \\
&= 50 + 50 \\
\end{array}
\]

And similarly for the standard deviations:

\[
\begin{array}{rl}
\sqrt{\var(A_{500}+B_{500})} &= \sqrt{\var(A_{500})+\var(B_{500})} \\
&=\sqrt{\sqrt{\var(A_{500})}^2+\sqrt{\var(B_{500})}^2} \\
&=\sqrt{10+15} \\
\end{array}
\]

Because $A$ and $B$ are independent.

\[
\subsolution{2.6}{11}{c}{\begin{array}{cc}
\E(A_{500}+B_{500})=100 & \sqrt{\var(A_{500}+B_{500})} = 5 \\
\end{array}}
\]

%
\subproblem{2.6}{11}{d} If $x$ dollars goes into fund $A$ and $T-x$
goes into fund $B$, then the variance is given by:

\[
\begin{array}{rl}
\var(A_x + B_{T-x}) &= \var(A_x) + \var(B_{T-x}) \\
&= \sqrt{\var(A_x)}^2 + \sqrt{\var(B_{T-x})}^2 \\
&= (0.02x)^2 + (0.03(T-x))^2 \\
\end{array}
\]

To find the minimum, we take a derivative:

\[
\dd{}{x}\var(A_x+B_{T-x}) = 0.0026 x - 0.0018 T
\]

Then we set it to $0$ and solve for $x$:

\[
x=\frac{9}{13} T
\] \[
\subsolution{2.6}{11}{d}{x=692.31}
\]

%
\problem{2.6}{12} We previously found, in problem 2.3.17, that the
expected value of each resistor is given by:

\[
\E X=10.418
\]

And in problem 2.2.11, we found that the PDF was:

\[
\rho_X(x)=\lbrack 10\le x\le 11\rbrack \frac{4x(130-x^2)}{819}
\]

%
\subproblem{2.6}{12}{a} To find the expected resistance of the 5
resistors in series, $X_1,X_2,\cdots,X_5$, we note that:

\[
\E(X_1+X_2+\cdots+X_5)=\E X_1 + \E X_2 + \cdots + \E X_5
\]

But because each $X_i=X$, this is just:

\[
\subsolution{2.6}{12}{a}{5 \E X =52.090}
\]

%
\subproblem{2.6}{12}{b} To find the variance of the total set of
resistors, we first want to calculate the variance for one resistor:

\[
\var(X)=\E(X^2)-\E(X)^2
\]

where

\[
\begin{array}{rl}
\E(X^2) &=\int x^2 \rho_X(x)\d{x} \\
&=108.62 \\
\end{array}
\]

Given a variance of:

\[
\begin{array}{rl}
\var(X)&=108.62 - 10.418^2 \\
&=0.085276 \\
\end{array}
\]

Then we note that, for independent variables $X_i$, then:

\[
\var\left(\sum\limits_i X_i\right) = \sum\limits_i \var(X_i)
\]

And because each $X_i=X$, then

\[
\var\left(\sum\limits_i X_i\right)=5\var(X)
\]

\[
\subsolution{2.6}{12}{b}{0.426380}
\]

%
\problem{3.1}{4}

Let $X$ be a random variable for how many arrows hit a bull's-eye out
of 9 total arrows, described by

\[
X\sim B(9, 0.09)
\]

%
\subproblem{3.1}{4}{a} We want to find the probability that two arrows
hit a bull's-eye, or in other words, the value of $P(X=2)$. Applying
the definition of the PMF for a binomially distributed random
variable, we get:

\[
P(X=2)=\binom{9}{2} 0.09^2(1-0.09)^{9-2}
\]

\[
\subsolution{3.1}{4}{a}{P(X=2)=0.1506875}
\]

%
\subproblem{3.1}{4}{b} To find the probability that greater than 2
bull's-eye are hit, we are looking for the value of
$P(X\ge2)$. However, we can simplify the expression if we look at the
value of the complement of this probability:

\[
P(X\ge2) = 1 - P(X<2)
\]

The latter side can be written as:

\[
\begin{array}{rl}
P(X\ge2) &= 1 - \sum\limits_{i=0}^1 P(X=i) \\
&= 1 - \left(P(X=0)+P(X=1)\right) \\
&= 1 - \left(\binom{9}{0} 0.09^0 (1-0.09)^{9-0} + \binom{9}{1} 0.09^1 (1-0.09)^{9-1}\right) \\
\end{array}
\]

\[
\subsolution{3.1}{4}{b}{P(X\ge2) =0.191166}
\]

%
\subproblem{3.1}{4}{c} The expected value can be computed directly
from the definition:

\[
\E X=9\cdot0.09
\] \[
\subsolution{3.1}{4}{c}{\E X=0.81}
\]

%
\problem{3.1}{6}

The individual probability that a randomly chosen answer from a pool
of $k$ answers is:

\[
p_k = \frac{1}{k}
\]

Let the variable $X_k$ be a binomial distribution:

\[
X_k\sim B(10, p_k)
\]

%
\subproblem{3.1}{6}{a} We want to find the probability that the
student passes the test with 5 possible answers per question,
i.e. $P(X_5\ge7)$. The definition states that this value is:

\[
P(X_5\ge7)=\sum\limits_{i=7}^{10} \binom{10}{i} \left(\frac{1}{5}\right)^i\left(1-\frac{1}{5}\right)^{10-i}
\]

\[
\subsolution{3.1}{6}{a}{P(X_5\ge7)=0.0008644}
\]

%
\subproblem{3.1}{6}{b} This time, we want to find the value of
$P(X_2\ge7)$:

\[
P(X_2\ge7)=\sum\limits_{i=7}^{10} \binom{10}{i} \left(\frac{1}{2}\right)^i\left(1-\frac{1}{2}\right)^{10-i}
\]

\[
\subsolution{3.1}{6}{b}{P(X_2\ge7)=0.171875}
\]

%
\problem{3.1}{7}

Let $X$ be a random variable for the number of people who need to take
sick leave, defined by:

\[
X\sim B(180, 0.35)
\]

Let $Y$ be a random variable for the proportion of people who need
sick leave:

\[
Y=\frac{1}{180} X
\]

%
\subproblem{3.1}{7}{a} The expectation of $X$ is defined to be:

\[
\E X=180\cdot 0.35
\]

Likewise, the expectation for $Y$ is:

\[
\begin{array}{rl}
\E Y &= \E\left(\frac{1}{180} X\right) \\
&= \frac{1}{180}\E X \\
\end{array}
\]

The variance for $X$ is:

\[
\var(X)=180\cdot0.35\cdot(1-0.35)
\]

And the variance for $Y$:

\[
\begin{array}{rl}
\var(Y) &= \var\left(\frac{1}{180} X\right) \\
&= \left(\frac{1}{180}\right)^2\var(X) \\
&= \frac{0.35\cdot(1-0.35)}{180} \\
\end{array}
\]

\[
\subsolution{3.1}{7}{a}{\begin{array}{cc}
\E Y=0.35 & \var(Y)=0.00126389 \\
\end{array}}
\]

%
\subproblem{3.1}{7}{b} For arbitrary sick rate $p$, we can define the
random variable $X$ as:

\[
X\sim B(180, p)
\]

with the same $Y$ from earlier. We can calculate the variance of $Y$:

\[
\begin{array}{rl}
\var(Y) &= \var\left(\frac{1}{180} X\right) \\
&= \left(\frac{1}{180}\right)^2 \var(X) \\
&= \frac{1}{180^2} 180\cdot p\cdot (1-p) \\
&= \frac{p-p^2}{180} \\
\end{array}
\]

To find the maximum variance, we set the derivative to $0$:

\[
\dd{}{p}\var(Y) =\frac{1}{180}\left(2p-1\right)
\] \[
\frac{1}{180}\left(2p-1\right)=0
\] \[
2p-1=0
\]

The individual probability value which leads to the largest variance
is:

\[
\subsolution{3.1}{7}{b}{p=0.5}
\]

%
\problem{3.1}{12}

Let $X$ be a random variable for the number of visitors bounce a
website, defined by:

\[
X\sim B(12, 0.93)
\]

We want to find the probability that at least 10 of the 12 visitors
will bounce, i.e. $P(X\ge10)$. From the definition, this is:

\[
P(X\ge10)=\sum\limits_{i=10}^{12} \binom{12}{i} 0.93^i (1-0.93)^{12-i}
\]

\[
\solution{3.1}{12}{P(X\ge10)=0.953203}
\]

\end{document}
