\documentclass{article}
%\usepackage[margin=0.75in]{geometry}
\usepackage{amsmath}
\usepackage{amssymb}
\usepackage{xintexpr}
\usepackage{multicol}
\usepackage{empheq}
\usepackage{forest}
\usepackage{booktabs}
\usepackage{ifthen}
\usepackage{xfrac}

\setlength{\columnsep}{0.5cm}
\setlength{\columnseprule}{1pt}

\newcommand{\T}{1}
\newcommand{\F}{0}
\newcommand{\TF}[1]{\if1#1\T\else\F\fi}
\newcommand{\xintTF}[1]{\xintifboolexpr{#1}{\T}{\F}}

\newcommand{\logicrule}[2]{
\begin{array}{l}
#1 \\
\midrule
\therefore #2 \\
\end{array}
}

\newcommand{\problem}[2]{$\boxed{\text{#1.#2}}$}
\newcommand{\subproblem}[3]{$\boxed{\text{(#3)}}$}
\newcommand{\solution}[3]{\boxed{#3\quad(\text{#1.#2})}}
\newcommand{\subsolution}[4]{\boxed{#4\quad(\text{#1.#2#3})}}

\newcommand{\inv}[1]{#1^{-1}}

\renewcommand{\d}[1]{\,\textnormal{d}#1}
\newcommand{\dd}[2]{\frac{\d{#1}}{\d{#2}}}
\newcommand{\ddd}[2]{\dfrac{\d{#1}}{\d{#2}}}

\DeclareMathOperator{\var}{Var}
\DeclareMathOperator{\E}{\mathcal{E}}

\setlength\parindent{0pt}
\setlength\parskip{0em}

\begin{document}

\section*{Homework 4}

%
\problem{2.6}{8}

The random variable $X$ has PDF:

\[
\rho_X(x)=\lbrack0\le x\le1\rbrack2x
\]

With a CDF of:

\[
F_X(x)=\lbrack0\le x\le1\rbrack x^2
\]

%
\subproblem{2.6}{8}{a} Suppose $Y=X^3$. The CDF for $Y$ is given by:

\[
\begin{array}{l}
F_Y(y)=P(Y\le y) \\
F_Y(y)=P(X^3\le y) \\
F_Y(y)=P(X\le \sqrt[3]{y}) \\
\end{array}
\]

But this value is just $F_X(\sqrt[3]{y})$, which we have a known value
for:

\[
F_Y(y)=F_X(\sqrt[3]{y})=y^{\sfrac{2}{3}}
\]

We can then take a derivative to get the PDF:

\[
\begin{array}{ll}
\rho_Y(y) &=\ddd{F_Y(y)}{y} \\
&=\ddd{}{y} y^{\sfrac{2}{3}} \\
\end{array}
\]

\[
\rho_Y(y)=\frac{2}{3}\cdot y^{-1/3}
\]

The expectation is then found via:

\[
\E Y=\int y \rho_Y(y)\d{y}
\] \[
\subsolution{2.6}{8}{a}{\begin{array}{l}
\rho_Y(y)=\frac{2}{3}\cdot y^{-1/3} \\
\E Y=\frac{2}{5} \\
\end{array}}
\]

%
\subproblem{2.6}{8}{b} Suppose $Y=\sqrt{X}$. The CDF is given by:

\[
\begin{array}{ll}
F_Y(y)&=P(Y\le y) \\
&=P(\sqrt{X}\le y) \\
&=P(X\le y^2) \\
&=F_X(y^2) \\
\end{array}
\]

We can then use the known value for $F_X(y^2)$:

\[
F_Y(y)=y^4
\]

The derivative is the PDF:

\[
\begin{array}{ll}
\rho_Y(y)&=\dd{}{y} y^4 \\
&=4y^3 \\
\end{array}
\]

Then the expectation is defined as:

\[
\begin{array}{ll}
\E Y&=\int y\rho_Y(y) \d{y} \\
\end{array}
\]

\[
\subsolution{2.6}{8}{b}{\begin{array}{l}
\rho_Y(y)=4y^3 \\
\E Y=\sfrac{4}{5}
\end{array}}
\]

%
\subproblem{2.6}{8}{c} Suppose $Y=1/(1+X)$. The CDF can be written as:

\[
\begin{array}{rl}
F_Y(y)&=P(Y\le y) \\
&=P\left(\frac{1}{1+X}\le y\right) \\
\end{array}
\]

Because $X$ is always positive, then $1+X$ will also always be
positive, so we can multiply both sides of the inequality by $1+X$ without

%
\subproblem{2.6}{8}{d} Suppose $Y=2^X$.

%
\problem{2.6}{10}

%
\problem{2.6}{11}

%
\problem{2.6}{12}

%
\problem{3.1}{4}

%
\problem{3.1}{6}

%
\problem{3.1}{7}

%
\problem{3.1}{12}

\end{document}
