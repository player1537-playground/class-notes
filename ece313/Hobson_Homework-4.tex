\documentclass{article}
%\usepackage[margin=0.75in]{geometry}
\usepackage{amsmath}
\usepackage{amssymb}
\usepackage{xintexpr}
\usepackage{multicol}
\usepackage{empheq}
\usepackage{forest}
\usepackage{booktabs}
\usepackage{ifthen}
\usepackage{xfrac}

\setlength{\columnsep}{0.5cm}
\setlength{\columnseprule}{1pt}

\newcommand{\T}{1}
\newcommand{\F}{0}
\newcommand{\TF}[1]{\if1#1\T\else\F\fi}
\newcommand{\xintTF}[1]{\xintifboolexpr{#1}{\T}{\F}}

\newcommand{\logicrule}[2]{
\begin{array}{l}
#1 \\
\midrule
\therefore #2 \\
\end{array}
}

\newcommand{\problem}[2]{$\boxed{\text{#1.#2}}$}
\newcommand{\subproblem}[3]{$\boxed{\text{(#3)}}$}
\newcommand{\solution}[3]{\boxed{#3\quad(\text{#1.#2})}}
\newcommand{\subsolution}[4]{\boxed{#4\quad(\text{#1.#2#3})}}

\newcommand{\inv}[1]{#1^{-1}}

\renewcommand{\d}[1]{\,\textnormal{d}#1}
\newcommand{\dd}[2]{\frac{\d{#1}}{\d{#2}}}
\newcommand{\ddd}[2]{\dfrac{\d{#1}}{\d{#2}}}

\setlength\parindent{0pt}
\setlength\parskip{0em}

\begin{document}

\section*{Homework 4}

%
\problem{2.3}{4}

We can reuse the solution to problem 2.1.9 which was:

\[
\begin{array}{l|l|l}
i & P_X(i) & F_X(i) \\
\midrule
0 & 0 & 0 \\
1 & \sfrac{2}{5} & \sfrac{2}{5} \\
2 & \sfrac{3}{10} & \sfrac{7}{10} \\
3 & \sfrac{1}{5} & \sfrac{9}{10} \\
4 & \sfrac{1}{10} & 1 \\
5 & 0 & 1 \\
\end{array}
\]

And then apply the definition for the expected value:

\[
E(X)=\sum\limits_i p_i x_i
\]

Where $p_i$ is the value from the table, $P_X(i)$, and $x_i$ is just
$i$. With this knowledge, the summation becomes:

\[
E(X)=\sum\limits_{i=1}^4 iP_X(i)
\] \[
E(X)=1\cdot\frac{2}{5}+2\cdot\frac{3}{10}+3\cdot\frac{1}{5}+4\cdot\frac{1}{10}
\] \[
\solution{2.3}{4}{E(X)=2}
\]

%
\problem{2.3}{12}

Using the result from problem 2.2.6, we know that the probability
density function is:

\[
f(x)=\lbrack0.125\le x\le0.5\rbrack 5.50538\left(0.5-(x-0.25)^2\right)
\]

%
\subproblem{2.3}{12}{a} We can apply the definition for the expected value
based on the PDF as:

\[
E(X)=\int\limits_{-\infty}^\infty xf(x)\d{x}
\]

The limits can be simplifed by the definition of $f(x)$, so we have:

\[
E(X)=\int\limits_{0.125}^{0.5} xf(x)\d{x}
\]

We can directly evaluate this integral to get:

\[
\subsolution{2.3}{12}{a}{E(X)=0.309476}
\]

%
\subproblem{2.3}{12}{b} In solving the median paint thickness, we want
to solve the following equation for $x$:

\[
F_X(\tilde{x})=0.5
\]

We can start by determining the value for $F_X(x)$, which is defined
as:

\[
F_X(x)=\int\limits_{-\infty}^x f(z)\d{z}
\]

Because of the limitation of the domain on $f(x)$, we can simplify the
limits of the integral:

\[
F_X(x)=\int\limits_{0.125}^x f(z)\d{z}
\]

Now the equation we want to solve is:

\[
0.5=\int\limits_{0.125}^{\tilde{x}} f(z)\d{z}
\]

We can solve this to find that $\tilde{x}$, which is the median, must
be:

\[
\subsolution{2.3}{12}{b}{\tilde{x}=0.308073}
\]

%
\problem{2.3}{13}

In problem 2.2.8, we found that the PDF was defined as:

\[
f(\theta)=\lbrack0\le\theta\le10\rbrack \frac{1}{e^{10}-11}\left(e^{10-\theta}-1\right)
\]

%
\subproblem{2.3}{13}{a} The expected value can be found by:

\[
E(\Theta)=\int\limits_{-\infty}^\infty \theta f(\theta)\d{\theta}
\]

We can simplify the range:

\[
E(\Theta)=\int\limits_{0}^{10} \theta f(\theta)\d{\theta}
\]

Which we can calculate as:

\[
\subsolution{2.3}{13}{a}{E(\Theta)=0.99773}
\]

%
\subproblem{2.3}{13}{b} To find the medium value, we solve the
following for $\tilde{\theta}$:

\[
0.5=F_\Theta(\tilde{\theta})
\]

Where $F_\Theta(\theta)$ is defined as:

\[
F_\Theta(\theta)=\int\limits_{-\infty}^\theta f(z)\d{z}
\] \[
F_\Theta(\theta)=\int\limits_{0}^\theta f(z)\d{z}
\]

Now the problem is to solve for $\tilde{\theta}$ of:

\[
0.5=F_\Theta(\tilde{\theta})=\int\limits_{0}^{\tilde{\theta}} f(z)\d{z}
\] \[
\subsolution{2.3}{13}{b}{\tilde{\theta}=0.692711}
\]

%
\problem{2.3}{16}

Previously, we found that the PMF and CDF for problem 2.1.11 were:

\[
\begin{array}{l|l|l}
k & P_f(k) & F_f(k) \\
\midrule
1 & 0 & 0 \\
2 & 0 & 0 \\
3 & \sfrac{1}{20} & \sfrac{1}{20} \\
4 & \sfrac{3}{20} & \sfrac{4}{20} \\
5 & \sfrac{6}{20} & \sfrac{10}{20} \\
6 & \sfrac{10}{20} & 1 \\
7 & 0 & 1 \\
\end{array}
\]

The expected value is defined as:

\[
E(f)=\sum\limits_k p_k x_k
\]

Where each $p_k=P_f(k)$ and $x_k=k$. This simplifies to:

\[
E(f)=\sum\limits{k=3}^6 k P_f(k)
\] \[
E(f)=3\cdot\frac{1}{20}+4\cdot\frac{3}{20}+5\cdot\frac{6}{20}+6\cdot\frac{10}{20}
\] \[
\solution{2.3}{16}{E(f)=5.25}
\]

%
\problem{2.3}{17}

Previously, we found in problem 2.2.11 that the PDF was:

\[
f(x)=\lbrack10\le x\le11\rbrack \frac{4x(130-x^2)}{819}
\]

and the CDF was:

\[
F(x)=\lbrack10\le x\le11\rbrack \frac{4}{819}\int\limits_{10}^x z(130-z^2)\d{z}
\]

%
\subproblem{2.3}{17}{a} The expected value of the resistance is
defined as:

\[
E(X)=\int\limits_{-\infty}^\infty x f(x)\d{x}
\]

The limits can be simplified to:

\[
E(X)=\int\limits_{10}^{11} x f(x)\d{x}
\]

This expression can be directly computed:

\[
\subsolution{2.3}{17}{a}{E(X)=10.418}
\]

%
\subproblem{2.3}{17}{b} The median value can be found by solving the
following for $\tilde{x}$:

\[
0.5=F_X(\tilde{x})
\]

We already know the value for $F_X(x)$, so we can directly solve this
to find:

\[
\subsolution{2.3}{17}{b}{\tilde{x}=10.385}
\]

%
\problem{2.4}{7}

%
\problem{2.4}{8}

%
\problem{2.4}{10}

%
\problem{2.4}{12}

%
\problem{2.4}{13}

\end{document}
