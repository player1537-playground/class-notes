\documentclass{article}
\usepackage{amsmath}
\usepackage{amssymb}
\usepackage{booktabs}
\usepackage{xintexpr}

\newcommand{\T}{1}
\newcommand{\F}{0}
\newcommand{\TF}[1]{\if1#1\T\else\F\fi}
\newcommand{\xintTF}[1]{\xintifboolexpr{#1}{\T}{\F}}

\newcommand{\logicrule}[2]{
\begin{array}{l}
#1 \\
\midrule
\therefore #2 \\
\end{array}
}

\newcommand{\inv}[1]{#1^{-1}}

\renewcommand{\d}[1]{\,\textnormal{d}#1}
\newcommand{\dd}[2]{\frac{\d{#1}}{\d{#2}}}
\newcommand{\ddd}[2]{\dfrac{\d{#1}}{\d{#2}}}

\setlength\parindent{0pt}
\setlength\parskip{1em}

\begin{document}

\section*{Differences between book and class}

We use the notation of a discrete random varaible
$f:\Omega\rightarrow\mathbb{R}$, then:

\[
P(\alpha(f))=\sum\limits_{\alpha(x)}P_f(x)
\]

The book likes to say $P(f=x)$, but that is equivalent to:

\[
P(f=x)\equiv P(f\in\{x\})
\] \[
P(\inv{f}\{x\})
\] \[
P_f(x)
\]

Remembering that:

\[
\inv{f}S=\{x|f(x)\in S\}
\]

We like our notation, because it's easier to remember the analog
between continuous and discrete variables. In the continuous case, we
have:

\[
P(\alpha(f))=\int\limits_{\alpha(x)}\rho_f(x)\d{x}
\]

\section*{Example 2.1.7}

Suppose we have pictures or something. Pictures can be perfect (P),
good (G), satisfactory (S), or failed (F).

Each event is where we take two pictures, represented as a 2-tuple
$(x,y)$, where $x,y\in\{P,G,S,F\}$.

We are told that there are different scores for each type of picture:

\[
\begin{array}{c|c}
x & S(x) \\
\midrule
P & 4 \\
G & 2 \\
S & 1 \\
F & 0 \\
\end{array}
\]

We are told some probabilities of events:

\[
\begin{array}{cccc}
\underset{0.14}{(P,P)} & \underset{0.102}{(P,G)} & \underset{0.157}{(P,S)} & \underset{0.007}{(P,F)} \\
(G,P) & (G,G) & (G,S) & (G,F) \\
(S,P) & (S,G) & (S,S) & (S,F) \\
(F,P) & (F,G) & (F,S) & (F,F) \\
\end{array}
\]

We are told that the total score of each event $(x,y)$ is
$S(x)\cdot{}S(y)$.

Now we want to know the probability that the total score is $1$. In
other words, we want to know:

\[
P_f(4) = P(\inv{f}\{4\})
\] \[
P(\{(P,S),(S,P),(G,G)\})
\]

And then we can just sum those probabilities. Cool beans.

\subsection*{Probability Mass Function}

To construct a PMF, we would just make a table (probably):

\[
\begin{array}{c|c}
x & P_f(x) \\
\midrule
4 & \cdots \\
1 & \cdots \\
2 & \cdots \\
\end{array}
\]

Where, for each row, we compute like we did earlier.

\section*{Cumulative Distribution Function}

We like to use special notation, because fuck it, why not.

For the random variable, $f$, the CDF is defined as:

\[
F_f(x)=P(f\le x)
\]

If we let $\alpha$ be:

\[
\alpha=\text{``$f\le x$''}
\]

Then we know its value. It's:

\[
\sum\limits_{\alpha(y)}P_f(y)
\]

Look. We get it. Variables are different. Dummy variables take up our
namespace, so use a different one. Every day, it's a new variation of
this.

Anyways, we end up with:

\[
F_f(x) = \sum\limits_{y\le x}P_f(y)
\]

\section*{Example 2.2.7}

A random variable $x$ is the time taken for a garage to service a
car. These times are within $\lbrack 0,10\rbrack$. The times are
distributed according to:

\[
F_x(t)=A+B\log(3t+2)\text{ for }0\le t\le 10
\]

How do we understand the domain of $t$? We know that, for $t<0$, then
$F_x(t)$ is $0$. And for $t>10$, then $F_x(t)$ is $1$. This should
make sense if we consider that it's the sum of probabilities.

So now we have $F_x(t)$, how do we get $\rho_x(t)$? We take a derivative!

\[
\rho_x(t)=B\frac{3}{3x+2}\text{ for }0\le t\le 10
\]

How do we understand the domain here? Well, $F_x(t)$ is $0$ or $1$
outside of the range, and the derivative of a constant is $0$, so the
probability density function is $0$ outside of the range.

\subsection*{Diversion: Evaluating weird integrals}

Suppose we want to evaluate:

\[
\int\limits\alpha(x) A(x)\d{x}
\]

For example, with a weird predicate like:

\[
\alpha(v)=\begin{cases}\text{T} & \text{if the closest prime to $v$ is
  the $k$th prime $P_k$ and $v\in(P_k-\frac{1}{k},P_k+\frac{1}{k})$} \\
\text{F} & {otherwise}
\end{cases}
\]

Then we just evaluate using Iverson brackets:

\[
\int\limits_{-\infty}{\infty}\lbrack \alpha(x)\rbrack A(x)\d{x}
\]

Yay.

\subsection*{Back to the problem}

Wouldn't it be nice if we could make the integral a known value? Well,
suppose we chose our predicate to be:

\[
\alpha(x)=\text{``$x\in\mathbb{R}$''}
\]

Then we'd have:

\[
P(\alpha(f)) = P(f\in\{x|\alpha(x)\})
\] \[
P(\inv{f}\{x|\alpha(x)\})
\] \[
P(\inv{f}\mathbb{R})
\] \[
P(\Omega)
\] \[
1
\]

So now we look at the density function, since we know the integral of
it should be the same as the CDF.

\[
P(\alpha(f)) = \int\limits_{\alpha(x)}\rho_f(x)\d{x}
\] \[
\int\limits{-\infty}^{\infty}\lbrack\alpha(x)\rbrack\frac{3B}{3x+2}\d{x}
\] \[
\int\limits_{0}^{10}\lbrack \alpha(x)\rbrack\frac{3B}{3x+2}\d{x}
\] \[
B\int\limits_{0}^{10}\frac{3}{3x+2}\d{x}
\]

Which is equal to $1$, so we can solve for $B$. We were able to limit
the domain of the integral because the CDF is only defined between $0$
and $10$.

Then we could figure out what $A$ is, based on our new value of $B$.

\end{document}
