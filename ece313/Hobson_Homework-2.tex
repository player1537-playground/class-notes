\documentclass{article}
%\usepackage[margin=0.75in]{geometry}
\usepackage{amsmath}
\usepackage{amssymb}
\usepackage{xintexpr}
\usepackage{multicol}
\usepackage{empheq}
\usepackage{forest}
\usepackage{booktabs}

\setlength{\columnsep}{0.5cm}
\setlength{\columnseprule}{1pt}

\newcommand{\T}{1}
\newcommand{\F}{0}
\newcommand{\TF}[1]{\if1#1\T\else\F\fi}
\newcommand{\xintTF}[1]{\xintifboolexpr{#1}{\T}{\F}}

\newcommand{\logicrule}[2]{
\begin{array}{l}
#1 \\
\midrule
\therefore #2 \\
\end{array}
}

\newcommand{\problem}[2]{$\boxed{\text{#1.#2}}$}
\newcommand{\subproblem}[3]{$\boxed{\text{(#3)}}$}
\newcommand{\solution}[3]{\boxed{#3\quad(\text{#1.#2})}}
\newcommand{\subsolution}[4]{\boxed{#4\quad(\text{#1.#2#3})}}

\setlength\parindent{0pt}
\setlength\parskip{0em}

\begin{document}

\section*{Homework 2}

%
\problem{1.4}{2}

This scenario can be described by two events:

\[
A:\text{``order is over \$1000''}
\] \[
B:\text{``person pays with credit card''}
\]

We are given the following information:

\[
P(A)=0.42
\] \[
P(B|A)=0.63
\]

%
\subproblem{1.4}{2}{a} We can further extend our previous events with
$A_i:\text{``order $i$ is over \$1000''}$ for $i\in\{1,2,3\}$, and
probabilities $P(A_i)=P(A)$. We are then looking for the value of
$P(A_1A_2A_3)$.

Furthermore, we are told that events $A_1$, $A_2$, and $A_3$ are
indepedent, which means that we can use the following equivalence:

\[
P(A_1A_2A_3)=P(A_1)P(A_2)P(A_3)
\]

And then substitute the known probabilities to get:

\[
\subsolution{1.4}{2}{a}{P(A_1A_2A_3)=0.07}
\]

%
\subproblem{1.4}{2}{b} For the next part, we are interested in the value of
$P(AB')$. We also know that for any events $P(X|Y)$, we have
$P(X'|Y)=1-P(X|Y)$, so we can use this to rewrite our second given as:

\[
P(B|A)=1-P(B'|A)
\]

And then using the definition of conditional probability, we can
rewrite this as

\[
P(B|A)=1 - \dfrac{P(B'A)}{P(A)}
\]

And we can then solve for $P(B'A)$ (which is equal to $P(AB')$):

\[
\dfrac{P(B'A)}{P(A)}=1-P(B|A)
\] \[
P(B'A)=P(A)-P(A)P(B|A)
\] \[
\subsolution{1.4}{2}{b}{P(B'A)=0.1554}
\]

%
\problem{1.4}{10}

We are given that there are two assembly lines, each with an event
that the line is shut down, running at partial capacity, or running at
full capacity. We can organize this as a 2-tuple $(X,Y)$ where
$X,Y\in\{S,P,F\}$. We are then told the probabilities, as follows:

\[
\boxed{\begin{array}{ccc}
\underset{0.02}{(S, S)} & \underset{0.06}{(S, P)} & \underset{0.05}{(S, F)} \\
\underset{0.07}{(P, S)} & \underset{0.14}{(P, P)} & \underset{0.20}{(P, F)} \\
\underset{0.06}{(F, S)} & \underset{0.21}{(F, P)} & \underset{0.19}{(F, F)} \\
\end{array}}
\]

%
\subproblem{1.4}{10}{a} For this problem, we're interested in the event that
both lines are at full capacity:

\[
E_{full}=\{(F,F)\}
\]

given that neither line is shut down:

\[
E_{working}=\{(P,P),(P,F),(F,P),(F,F)\}
\]

or in other words, $P(E_{full}|E_{working})$, which by application of
the definition of conditional probability is:

\[
P(E_{full}|E_{working})=
\dfrac{P(E_{full}E_{working})}{P(E_{working})}
\]

We know that each of the elementary events that make up $E_{working}$
are mutually exclusive, so we can apply the union of mutually
exclusive events definition to get:

\[
P(E_{full}|E_{working})=
\dfrac{P(E_{full}E_{working})}{P((P,P))+P((P,F))+P((F,P))+P((F,F))}
\]

And we can also trivially find the value of $E_{full}E_{working}$, to get:

\[
P(E_{full}|E_{working})=
\dfrac{P((F,F))}{P((P,P))+P((P,F))+P((F,P))+P((F,F))}
\]

And then substituing known values, we get the final answer:

\[
\subsolution{1.4}{10}{a}{P(E_{full}|E_{working})=0.25675}
\]

%
\subproblem{1.4}{10}{b} In addition to the previous events, we now
have the event that at least one line is at full capacity:

\[
E_{1+ full}=\{(S,F),(P,F),(F,P),(F,S),(F,F)\}
\]

and we are interested in the value of $P(E_{1+ full}|E_{working})$.
Through the same steps as before, we get:

\[
P(E_{1+ full}|E_{working})=
\dfrac{P(E_{1+full}E_{working})}{P(E_{working})}
\] \[
P(E_{1+full}|E_{working})=
\dfrac{P(E_{1+full}E_{working})}{P((P,P))+P((P,F))+P((F,P))+P((F,F))}
\] \[
P(E_{1+full}|E_{working})=
\dfrac{P(\{(P,F),(F,P),(F,F)\})}
{P((P,P))+P((P,F))+P((F,P))+P((F,F))}
\] \[
P(E_{1+full}|E_{working})=
\dfrac{P((P,F))+P((F,P))+P((F,F))}
{P((P,P))+P((P,F))+P((F,P))+P((F,F))}
\] \[
\subsolution{1.4}{10}{b}{P(E_{1+full}|E_{working})=0.81081}
\]

%
\subproblem{1.4}{10}{c} We introduce a new event for when exactly one
line is shut down:

\[
E_{1shutdown} = \{(S,P),(S,F),(P,S),(F,S))\}
\]

and where one line is at full capacity:

\[
E_{1full}=\{(S,F),(P,F),(F,S),(F,P)\}
\]

and we are interested in the value of
$P(E_{1full}|E_{1shutdown})$. The steps are the same as the previous
problems:

\[
P(E_{1full}|E_{1shutdown})=
\dfrac{P(E_{1full}E_{1shutdown})}{P(E_{1shutdown})}
\] \[
P(E_{1full}|E_{1shutdown})=
\dfrac{P(E_{1full}E_{1shutdown})}
{P((S,P))+P((S,F))+P((P,S))+P((F,S))}
\] \[
P(E_{1full}|E_{1shutdown})=
\dfrac{P(\{(S,F),(F,S)\})}
{P((S,P))+P((S,F))+P((P,S))+P((F,S))}
\] \[
P(E_{1full}|E_{1shutdown})=
\dfrac{P((S,F))+P((F,S))}
{P((S,P))+P((S,F))+P((P,S))+P((F,S))}
\] \[
\subsolution{1.4}{10}{c}{P(E_{1full}|E_{1shutdown})=0.45833}
\]

%
\subproblem{1.4}{10}{d} For the last one, we introduce two new events:
neither line is at full capacity:

\[
E_{0full}=E_{1+full}'=\{(S,S),(S,P),(P,S),(P,P)\}
\]

and where at least one line is at partial capacity:

\[
E_{1+partial}=\{(S,P),(P,P),(F,P),(P,S),(P,F)\}
\]

and we are interested in the value of $P(E_{0full}|E_{1+partial})$. The steps we follow are:

\[
P(E_{0full}|E_{1+partial})=
\dfrac{P(E_{0full}E_{1+partial})}
{P(E_{1+partial})}
\] \[
P(E_{0full}|E_{1+partial})=
\dfrac{P(E_{0full}E_{1+partial})}
{P((S,P))+P((P,P))+P((F,P))+P((P,S))+P((P,F))}
\] \[
P(E_{0full}|E_{1+partial})=
\dfrac{P(\{(S,P),(P,S),(P,P)\})}
{P((S,P))+P((P,P))+P((F,P))+P((P,S))+P((P,F))}
\] \[
P(E_{0full}|E_{1+partial})=
\dfrac{P((S,P))+P((P,S))+P((P,P))}
{P((S,P))+P((P,P))+P((F,P))+P((P,S))+P((P,F))}
\] \[
\subsolution{1.4}{10}{d}{P(E_{0full}|E_{1+partial})=0.39705}
\]

%
\problem{1.4}{11}

We have 3 events that describe the probability space:

\[
W:\text{``width is within tolerance''}
\] \[
H:\text{``height is within tolerance''}
\] \[
L:\text{``length is within tolerance''}
\]

We are given the following information:

\[
P(W)=0.86
\] \[
P(WHL)=0.86
\] \[
P(WH'L)=0.02
\] \[
P(WHL')=0.03
\] \[
P(W+H)=0.92
\]

%
\subproblem{1.4}{11}{a} We are interested in the value of
$P(W|H)$. Our first step is to expand this based on the definition of
conditional probability:

\[
P(W|H)=\dfrac{P(WH)}{P(H)}
\]

We can then take our last given value and rewrite it and solve for
$P(H)$ to get a quantity for the denominator:

\[
P(W+H)=P(W)+P(H)-P(WH)
\] \[
P(H)=P(W+H)+P(WH)-P(W)
\]

Then to get the numerator $P(WH)$, we can look at the collection of
events:

\[
\zeta=\{WHL, WHL'\}
\]

We can show that these are mutually exclusive:

\[
(WHL)(WHL')
\] \[
WHLWHL'
\] \[
(WW)(HH)(LL')
\] \[
(WW)(HH)(\varnothing)
\] \[
\varnothing
\]

Therefore we can use the definition for the union of mutually
exclusive events:

\[
P\left(\bigcup\limits_{E\in\zeta}E\right)=\sum\limits_{E\in\zeta}P(E)
\]

We then need to determine the union of the inside of the left side:

\[
WHL+WHL'
\] \[
WH(L+L')
\] \[
WH(\Omega)
\] \[
WH
\]

So now we have:

\[
P(WH)=P(WHL)+P(WHL')
\]

and we can substitute this into the earlier expression:

\[
P(W|H)=\dfrac{P(WHL)+P(WHL')}{P(W+H)+P(WHL)+P(WHL')-P(W)}
\] \[
\subsolution{1.4}{11}{a}{P(W|H)=0.93684}
\]

%
\subproblem{1.4}{11}{b} For this problem, we want to find the value of
$P(WHL|WL)$. By applying the definition of conditional probability, we
get:

\[
P(WHL|WL)=\dfrac{P(WHLWL)}{P(WL)}
\] \[
P(WHL|WL)=\dfrac{P(WHL)}{P(WL)}
\]

We already know the value of the numerator, so we only need to compute
the denominator. We can again select a set of mutually exclusive
events:

\[
\zeta=\{WH'L, WHL\}
\]

Which we could show to be mutually exclusive in the same way as the
previous problem. We can then apply the definition of the union of
mutually exclusive events:

\[
P\left(\bigcup\limits_{E\in\zeta}E\right)=\sum\limits_{E\in\zeta}P(E)
\]

The union on the left will evaluate to $WL$ by the same logic as the
previous problem. Now we have:

\[
P(WL)=P(WHL)+P(WH'L)
\]

and we can substitute this into our previous expression:

\[
P(WHL|WL)=\dfrac{P(WHL)}{P(WHL)+P(WH'L)}
\] \[
\subsolution{1.4}{11}{b}{P(WHL|WL)=0.97727}
\]

%
\problem{1.4}{13}

In this problem, we have 3 events:

\[
R:\text{``performance passes''}
\] \[
A:\text{``appearance passes''}
\] \[
C:\text{``cost passes''}
\]

and the following givens:

\[
P(AC)=0.40
\] \[
P(RAC)=0.31
\] \[
P(R)=0.64
\] \[
P(R'A'C')=0.19
\] \[
P(R'AC')=0.06
\]

%
\subproblem{1.4}{13}{a\dag} We are interested in the value of
$P(R'A'C)$. Equivalently, we can look for the value of:

\[
P(R'A'C)=1-P((R'A'C)')
\] \[
P(R'A'C)=1-P(R+A+C')
\]

By applying the definition for the union of events, we have:

\[
P(R+A+C')=P(R)+P(A)+P(C')-P(RA)-P(AC')-P(RC')+P(RAC')
\]

%
\subproblem{1.4}{13}{b} We are looking for the value of the expression
$P(RAC|AC)$. We can apply the definition of conditional probability to
get:

\[
P(RAC|AC)=\dfrac{P(RACAC)}{P(AC)}
\] \[
P(RAC|AC)=\dfrac{P(RAC)}{P(AC)}
\]

Luckily, we know all of these, so we can just substitute the known
values and we're done.

\[
\subsolution{1.4}{13}{b}{P(RAC|AC)=0.775}
\]

%
\problem{1.4}{16}

There are two events:

\[
T:\text{``chip does survive 500 cycles''}
\] \[
A:\text{``chip is from company A''}
\]

We have the following givens:

\[
P(T)=0.42
\] \[
P(A|T')=0.73
\]

We are looking for the probability $P(T'A')$. We can apply the
following relation between probabilities and their complements:

\[
P(A'|T')=1-P(A|T')
\] \[
P(T')=1-P(T)
\]

and then, by applying the definition for conditional probability, we
get:

\[
P(A'|T')=\dfrac{P(A'T')}{P(T')}
\]

Where the numerator is the value we're solving for, and the
denominator is already known. We can substitute known values and solve
for $P(A'T')$ to get the solution:

\[
1-P(A|T')=\dfrac{P(A'T')}{1-P(T)}
\] \[
P(A'T')=(1-P(T))(1-P(A|T'))
\] \[
\solution{1.4}{16}{P(A'T')=0.1566}
\]

%
\problem{1.5}{6}

Let $A$ and $B$ be indepedent events. Therefore:

\[
P(AB)=P(A)P(B)
\]

and similarly, we have:

\[
P(A|B)=P(A)
\] \[
P(B|A)=P(B)
\]

%
\subproblem{1.5}{6}{a} We want to see whether $A$ and $B'$ are
independent. In other words:

\[
P(AB')\overset{?}{=}P(A)P(B')
\]

We know based on the complement of conditional probabilities that:

\[
P(B'|A)=1-P(B|A)
\]

But we can apply what we found earlier, that $P(B|A)=P(B)$ to
simplify the expression and get:

\[
P(B'|A)=1-P(B)
\]

But $1-P(B)$ is the same as $P(B')$, therefore:

\[
P(B'|A)=P(B')
\]

which means that $A$ and $B'$ are indepedent events.

%
\subproblem{1.5}{6}{b} We can apply the exact same reasoning as the
previous problem, swapping every $A$ for $B$ and vice versa.

%
\subproblem{1.5}{6}{c} Working from the knowledge that $A$ and $B$ are
indepedent, and $A$ and $B'$ are indepedent, we can start with the
true statement:

\[
P(A|B')=P(A)
\]

And apply what we know about the complement of conditional
probabilities, to get:

\[
P(A'|B')=1-P(A|B')
\]

But we already know what $P(A|B')$ is:

\[
P(A'|B')=1-P(A)
\]

We know that $1-P(A)$ is the same thing as $P(A')$, so we can write:

\[
P(A'|B')=P(A')
\]

which means that $A'$ and $B'$ are indepedent.

%
\problem{1.5}{7}

There are 3 switches which operate indepedently of each other, with
appropriate events:

\[
S_i:\text{``switch $i$ allows message through''}
\]

Each switch has an associated probability:

\[
P(S_1)=0.88
\] \[
P(S_2) = 0.92
\] \[
P(S_3) = 0.90
\]

We also have aggregate events for if the switch makes it through the
top portion of the network (which has $S_1$ going to $S_2$), and the
bottom portion is just $S_3$.

\[
T=S_1S_2
\] \[
B=S_3
\]

We want to find the probability $P(T+B)$. We can apply the definition
for the union of events to get:

\[
P(T+B)=P(T)+P(B)-P(TB)
\]

We can then substitute the known values for $T$ and $B$ to get:

\[
P(T+B)=P(S_1S_2)+P(S_3)-P(S_1S_2S_3)
\]

Because we know that these are indepedent events, we can simply
multiply the probabilities of each part of the intersection, to get:

\[
P(T+B)=P(S_1)P(S_2)+P(S_3)-P(S_1)P(S_2)P(S_3)
\]

And then by substituting known values for each of these probabilities,
we get the final answer:

\[
\solution{1.5}{7}{P(T+B)=0.98096}
\]

%
\problem{1.5}{9}

The original probability that a broken lightbulb is chosen is:

\[
P(B)=\frac{17}{100}
\]

The probability tree follows. Selecting a broken light bulb is the
path to the left, whereas selecting a non-broken light bulb is to the
right.

\begin{center}
\begin{forest}
  [\text{.}
    [$\dfrac{17}{100}$
      [$\dfrac{16}{99}$
        [$\dfrac{15}{98}$]
        [$\dfrac{83}{98}$]
      ]
      [$\dfrac{83}{99}$
        [$\dfrac{16}{98}$]
        [$\dfrac{82}{98}$]
      ]
    ]
    [$\dfrac{83}{100}$
      [$\dfrac{17}{99}$
        [$\dfrac{16}{98}$]
        [$\dfrac{82}{98}$]
      ]
      [$\dfrac{82}{99}$
        [$\dfrac{17}{98}$]
        [$\dfrac{81}{98}$]
      ]
    ]
  ]
\end{forest}
\end{center}

\subproblem{1.5}{9}{a} To find the probability that there are no
broken light bulbs chosen, we simply multiply the fractions we get by
following the path to the right each time. In other words, we do:

\[
P(\text{no broken bulbs})=\dfrac{83}{100}\cdot\dfrac{82}{99}\cdot\dfrac{81}{98}
\] \[
\subsolution{1.5}{9}{a}{P(\text{no broken bulbs})=0.56821}
\]

\subproblem{1.5}{9}{b} To find this probability, we follow these
paths: LRR, RLR, RRL, RRR. We multiply across each path and then sum
up the products to find the final probability:

\[
P(E)=
\left(\dfrac{17}{100}\cdot
\dfrac{83}{99}\cdot
\dfrac{82}{98}\right)
+
\left(\dfrac{83}{100}\cdot
\dfrac{17}{99}\cdot
\dfrac{82}{98}\right)
+
\left(\dfrac{83}{100}\cdot
\dfrac{82}{99}\cdot
\dfrac{17}{98}\right)
+
\left(\dfrac{83}{100}\cdot
\dfrac{82}{99}\cdot
\dfrac{81}{98}\right)
\]

\[
\subsolution{1.5}{9}{b}{P(E)=0.92599}
\]

%
\problem{1.5}{16}

Let there be $n$ components and associated indepedent events, where
each $E_i$ is defined as:

\[
E_i:\text{``$i$th component operates correctly''}
\]

And the probability of each event defined as:

\[
\forall i\in\{1,2,\cdots,n\}(P(E_i)=\rho=0.90)
\]

We are looking for the value of $n$ where:

\[
P\left(\bigcup\limits_{i=1}^n E_i\right)\ge 0.995
\]

We can look at the value of the complement, defined as:

\[
P\left(\bigcup\limits_{i=1}^n E_i\right)=
1 - P\left(\left(\bigcup\limits_{i=1}^nE_i\right)'\right)
\]

By applying DeMorgan's Law, we have:

\[
P\left(\bigcup\limits_{i=1}^n E_i\right)=
1 - P\left(\bigcap\limits_{i=1}^nE_i'\right)
\]

Because each $E_i$ is independent, so is each $E_i'$. Therefore, we
can take the intersection and rewrite it as:

\[
P\left(\bigcup\limits_{i=1}^n E_i\right)=
1 - \prod\limits_{i=1}^nP(E_i')
\]

Then we can rewrite $P(E_i')$ to get:

\[
P\left(\bigcup\limits_{i=1}^n E_i\right)=
1 - \prod\limits_{i=1}^n(1-P(E_i))
\]

Finally, since every event has the same probability, we can simplify
the expression:

\[
P\left(\bigcup\limits_{i=1}^n E_i\right)=
1 - (1-\rho)^n
\]

Now we just need to solve the given problem, which is:

\[
P\left(\bigcup\limits_{i=1}^n E_i\right)\ge 0.995
\] \[
1-(1-\rho)^n\ge 0.995
\] \[
\solution{1.5}{16}{\lceil n\rceil=3}
\]

%
\problem{1.6}{1}

The events used in this problem are:

\[
A:\text{``People who actually have disease''}
\] \[
T:\text{``Test says person has disease''}
\]

And associated probabilities:

\[
P(A)=0.01
\] \[
P(T|A)=0.97
\] \[
P(T|A')=0.06
\]

%
\subproblem{1.6}{1}{a} We are looking for the value of $P(T)$. We
start with the definition of conditional probability to get:

\[
P(T|A)=\dfrac{P(TA)}{P(A)}
\]

and

\[
P(T|A')=\dfrac{P(TA')}{P(A')}
\]

We can solve these for $P(TA)$ and $P(TA')$ respectively:

\[
P(TA)=P(T|A)P(A)
\] \[
P(TA')=P(T|A')P(A')
\]

If we select a collection of events as:

\[
\zeta=\{TA,TA'\}
\]

We can see that they are mutually exclusive, therefore we can apply
the definition for the union of mutually exclusive events, which is:

\[
P\left(\bigcup\limits_{E\in\zeta}E\right)=\sum\limits_{E\in\zeta}P(E)
\]

The union on the left evaluates to:

\[
\bigcup\limits_{E\in\zeta}E
\] \[
TA+TA'
\] \[
T(A+A')
\] \[
T\Omega
\] \[
T
\]

And the sum on the right is:

\[
P(TA)+P(TA')
\]

So we now have:

\[
P(T)=P(TA)+P(TA')
\]

And can substitute the equivalences we found earlier and solve:

\[
P(T)=P(T|A)P(A)+P(T|A')P(A')
\] \[
P(T)=P(T|A)P(A)+P(T|A')(1-P(A))
\] \[
\subsolution{1.6}{1}{a}{P(T)=0.0691}
\]

%
\subproblem{1.6}{1}{b} We are looking for the value of $P(A|T)$. We
can arrive at this value via Bayes' Theorem:

\[
P(A|T)=\dfrac{P(T|A)P(A)}{P(T)}
\] \[
\subsolution{1.6}{1}{b}{P(A|T)=0.14038}
\]

%
\subproblem{1.6}{1}{c} We are looking for the value of
$P(A'|T')$. Again, we can use Bayes' Theorem:

\[
P(A'|T')=\dfrac{P(T'|A')P(A')}{P(T')}
\] \[
P(A'|T')=\dfrac{(1-P(T|A'))(1-P(A))}{1-P(T)}
\] \[
\subsolution{1.6}{1}{c}{P(A'|T')=0.99968}
\]

%
\problem{1.6}{5}

We have five events that describe the probability space:

\[
A:\text{``Circuit got an A''}
\] \[
B:\text{``Circuit got a B''}
\] \[
C:\text{``Circuit got a C''}
\] \[
D:\text{``Circuit got a D''}
\] \[
F:\text{``Circuit eventually failed''}
\]

And we know the following probabilities:

\[
\begin{array}{c|c|c}
E & P(E) & P(F|E) \\
\midrule
A & 0.77 & 0.02 \\
B & 0.11 & 0.10 \\
C & 0.07 & 0.14 \\
D & 0.05 & 0.25 \\
\end{array}
\]

%
\subproblem{1.6}{5}{a} We are looking for the value of $P(C+D|F)$. By
applying Bayes' Theorem, we have:

\[
P(C+D|F)=\dfrac{P(F|C+D)P(C+D)}{P(F)}
\]

We can expand this using the definition of conditional probability:

\[
P(C+D|F)=\dfrac{\dfrac{P(F(C+D))}{P(C+D)}P(C+D)}{P(F)}
\] \[
P(C+D|F)=\dfrac{P(F(C+D))}{P(F)}
\] \[
P(C+D|F)=\dfrac{P(FC+FD)}{P(F)}
\]

Because $FC$ and $FD$ are mutually exclusive (a circuit can't have
both a C rating and a D rating), we can expand $P(FC+FD)$:

\[
P(C+D|F)=\dfrac{P(FC)+P(FD)}{P(F)}
\]

If we then look at the definition of conditional probability applied
to some of our givens, we have:

\[
P(F|C)=\dfrac{P(FC)}{P(C)}
\] \[
P(F|D)=\dfrac{P(FD)}{P(D)}
\]

And then solving for $P(FC)$ and $P(FD)$ respectively:

\[
P(FC)=P(F|C)P(C)
\] \[
P(FD)=P(F|D)P(D)
\]

Which we can substitute into our earlier equation:

\[
P(C+D|F)=\dfrac{P(F|C)P(C)+P(F|D)P(D)}{P(F)}
\]

To calculate the denominator, we use the law of total probability,
with the collection of events as:

\[
\zeta=\{A,B,C,D\}
\]

Which are mutually exclusive and a superset of the event we're solving
for $F$.

\[
P(F)=\sum\limits_{E\in\zeta}P(F|E)P(E)
\] \[
P(F)=P(F|A)P(A)+P(F|B)P(B)+P(F|C)P(C)+P(F|D)P(D)
\]

Which we can substitute into our equation and solve for $P(C+D|F)$.

\[
P(C+D|F)=\dfrac{P(F|C)P(C)+P(F|D)P(D)}{P(F|A)P(A)+P(F|B)P(B)+P(F|C)P(C)+P(F|D)P(D)}
\] \[
\subsolution{1.6}{5}{a}{P(C+D|F)=0.45791}
\]

%
\subproblem{1.6}{5}{b} We are looking for the value of
$P(A|F')$. Through basic application of Bayes' Theorem, we have:

\[
P(A|F')=\dfrac{P(F'|A)P(A)}{P(F')}
\]

We can then look at the complements of these probabilities to make the
computation easier.

\[
P(A|F')=\dfrac{(1-P(F|A))P(A)}{1-P(F)}
\]

We can then apply what we learned earlier about the value of $P(F)$:

\[
P(A|F')=\dfrac{(1-P(F|A))P(A)}{1-(P(F|A)P(A)+P(F|B)P(B)+P(F|C)P(C)+P(F|D)P(D))}
\] \[
\subsolution{1.6}{5}{b}{P(A|F')=0.79323}
\]

%
\problem{1.6}{7}

We have the following events:

\[
C:\text{``Valve used at cold temperature''}
\] \[
M:\text{``Valve used at medium temperature''}
\] \[
W:\text{``Valve used at warm temperature''}
\] \[
H:\text{``Valve used at hot temperature''}
\] \[
L:\text{``Valve leaks''}
\]

And the following probabilities:

\[
\begin{array}{c|c|c}
E & P(E) & P(L|E) \\
\midrule
C & 0.12 & 0.003 \\
M & 0.55 & 0.009 \\
W & 0.20 & 0.014 \\
H & 0.13 & 0.018 \\
\end{array}
\]

%
\subproblem{1.6}{7}{a} We want to find $P(H|L)$. We can first apply
Bayes' Theorem:

\[
P(H|L)=\dfrac{P(L|H)P(H)}{P(L)}
\]

To find $P(L)$, we apply the Law of Total Probability, using the
collection of events:

\[
\zeta=\{C,M,W,H\}
\]

which is a superset of our desired event ($L$), so now we have:

\[
P(L)=P(L|C)P(C)+P(L|M)P(M)+P(L|W)P(W)+P(L|H)P(H)
\]

And substitute and solve:

\[
P(H|L)=\dfrac{P(L|H)P(H)}{P(L|C)P(C)+P(L|M)P(M)+P(L|W)P(W)+P(L|H)P(H)}
\] \[
\subsolution{1.6}{7}{a}{P(H|L)=0.22392}
\]

%
\subproblem{1.6}{7}{b} We want to find $P(M|L')$. Bayes' Theorem:

\[
P(M|L')=\dfrac{P(L'|M)P(M)}{P(L')}
\]

Complement:

\[
P(M|L')=\dfrac{(1-P(L|M))P(M)}{1-P(L)}
\]

Then we can use the previous value of $P(L)$ and solve:

\[
P(M|L')=\dfrac{(1-P(L|M))P(M)}{1-(P(L|C)P(C)+P(L|M)P(M)+P(L|W)P(W)+P(L|H)P(H))}
\] \[
\subsolution{1.6}{7}{b}{P(M|L')=0.55081}
\]

\end{document}
