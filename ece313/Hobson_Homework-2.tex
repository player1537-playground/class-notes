\documentclass{article}
\usepackage{amsmath}
\usepackage{pgfplots}
\usepackage{empheq}

\setlength\parindent{0pt}
%\setlength\parskip{1em}
\pgfkeys{/pgfplots/MyAxisStyle/.style={xmin=-10,xmax=10, ymin=-10,ymax=10,height=6cm,width=6cm}}

\begin{document}

\section*{Homework 2}


\subsection*{1.4.2 (a)}

Givens:

\[
\begin{cases}
P(\text{over \$1000}) = 0.42 \\
P\left(\text{paid with credit card}\mid\text{over \$1000}\right) = 0.63
\end{cases}
\]

\[
(P(\text{over \$1000}))^3 = (0.42)^3
\]

\[
\boxed{(P(\text{over \$1000}))^3 = 0.07}
\]

\subsection*{1.4.2 (b)}

\[
P\left(\text{not paid with credit card}\mid\text{over \$1000}\right) =
1 - P\left(\text{paid with credit card}\mid\text{over \$1000}\right)
\]

\[
\boxed{P\left(\text{not paid with credit card}\mid\text{over \$1000}\right) = 0.37}
\]

\subsection*{1.4.10}

\[
\begin{array}{ccc}
\underset{0.02}{(S, S)} & \underset{0.06}{(S, P)} & \underset{0.05}{(S, F)} \\
\underset{0.07}{(P, S)} & \underset{0.14}{(P, P)} & \underset{0.20}{(P, F)} \\
\underset{0.06}{(F, S)} & \underset{0.21}{(F, P)} & \underset{0.19}{(F, F)}
\end{array}
\]

\subsection*{1.4.10 (a)}

\[
\text{full capacity } = E_{full} = \{(F, F)\}
\]

\[
\text{neither shut down } = E_{working} = \{(P, P), (P, F), (F, P), (F, F)\}
\]

\[
P\left(E_{full}\mid E_{working}\right) =
\frac{P(E_{full}\cap E_{working})}{P(E_{working})}
\]

\[
P\left(E_{full}\mid E_{working}\right) =
\frac
    {P((F, F))}
    {P((P, P)) + P((P, F)) + P((F, P)) + P((F, F))}
\]

\[
P\left(E_{full}\mid E_{working}\right) =
\frac
    {0.19}
    {0.14 + 0.20 + 0.21 + 0.19}
\]

\[
\boxed{P\left(E_{full}\mid E_{working}\right) = 0.25675}
\]

\subsection*{1.4.10 (b)}

\[
\text{at least one at full capacity } = E_{1+ full} = \{(P, F), (F, P), (F, F)\}
\]

\[
P\left(E_{1+ full}\mid E_{working}\right) =
\frac
    {P(E_{1+ full}\cap E_{working}}
    {P(E_{working})}
\]

\[
P\left(E_{1+ full}\mid E_{working}\right) =
\frac
    {P((P, F)) + P((F, P)) + P((F, F))}
    {P((P, P)) + P((P, F)) + P((F, P)) + P((F, F))}
\]

\[
P\left(E_{1+ full}\mid E_{working}\right) =
\frac
    {0.20 + 0.21 + 0.19}
    {0.14 + 0.20 + 0.21 + 0.19}
\]

\[
\boxed{P\left(E_{1+ full}\mid E_{working}\right) = 0.81081}
\]

\subsection*{1.4.11}

Let the following events represent the english equivalents:

\[
\begin{cases}
L: \text{length within tolerances} \\
W: \text{width within tolerances} \\
H: \text{height within tolerances} \\
\end{cases}
\]

The given statements are:

\[
\begin{cases}
P(L) = 0.86 \\
P(L\cap W\cap H) = 0.80 \\
P(L\cap W\cap H') = 0.02 \\
P(L'\cap W\cap H) = 0.03 \\
P(W\cup H) = 0.92 \\
\end{cases}
\]

Using these we can also compute the opposite probabilities:

\[
\begin{cases}
P(L') = 1 - P(L) = 0.14 \\
P(L'\cup W'\cup H') = 1 - P(L\cap W\cap H) = 0.20 \\
P(L'\cup W'\cup H) = 1 - P(L\cap W\cap H') = 0.98 \\
P(L\cup W'\cup H') = 1 - P(L'\cap W\cap H) = 0.97 \\
P(W'\cap H') = 1 - P(W\cup H) = 0.08 \\
\end{cases}
\]

\subsection*{1.4.11 (a)}

\textit{If a part is within the specified tolerance limits for height}
(event = $H$)
\textit{what is the probability that it will also be within the specified tolerance limits for width?}
(what is $P(W|H)$?)

From the basic definition of conditional probability, we can rewrite
this probability as:

\[
P(W|H) = \frac{P(W\cap H)}{P(H)}
\]

Then we can use the following equivalences:

\[
\begin{cases}
P(W\cap H) = P(L\cap W\cap H) + P(L'\cap W\cap H) \\
P(H) = P(L\cap W\cap H) + P(L\cap W'\cap H) + P(L'\cap W\cap H) + P(L'\cap W'\cap H)
\end{cases}
\]





\subsection*{1.4.13}



\subsection*{1.4.16}

\end{document}
