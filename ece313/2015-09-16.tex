\documentclass{article}
\usepackage{amsmath}
\usepackage{amssymb}
\usepackage{booktabs}
\usepackage{xintexpr}

\newcommand{\T}{1}
\newcommand{\F}{0}
\newcommand{\TF}[1]{\if1#1\T\else\F\fi}
\newcommand{\xintTF}[1]{\xintifboolexpr{#1}{\T}{\F}}

\newcommand{\logicrule}[2]{
\begin{array}{l}
#1 \\
\midrule
\therefore #2 \\
\end{array}
}

\setlength\parindent{0pt}
\setlength\parskip{1em}

\begin{document}

\section*{Example 1.5.7}

Suppose we have a network split in two halves, top and bottom, and has
three switches $\{S_1,S_2,S_3\}$. $S_1$ connects to $S_2$ and makes up
the top half. $S_3$ makes up the bottom half.

Each switch has an associated probability of success for the message
going through, $P(S_i)=p_i$. Each switch operates independently of
each other.

\[
\begin{array}{cc}
i & p_i \\
\midrule
1 & 0.88 \\
2 & 0.92 \\
3 & 0.90 \\
\end{array}
\]

We are interested in what the probability that the message gets
through at all. In other words, we want to know:

\[
P(T\cup B)
\]

Where \[
T=S_1\cap S_2
\] \[
B=S_3
\]

We can apply the basic rule of unions of events, to get:

\[
P(T\cup B)=P(T)+P(B)-P(T\cap B)
\]

By substituting $T$ and $B$, we have:

\[
P(T\cup B)=\underbrace{P(S_1\cap S_2)}_{P(T)}+\underbrace{P(S_3)}_{P(B)}-\underbrace{P((S_1\cap S_2)\cap S_3)}_{P(T\cap B)}
\]

Because we were told that the switches operate independently, we can
just multiply each of the intersected events, to get:

\[
P(T\cup B)=
\underbrace{P(S_1)P(S_2)}_{P(S_1\cap S_2)}
+
P(S_3)
-
\underbrace{P(S_1)P(S_2)P(S_3)}_{P((S_1\cap S_2)\cap S_3)}
\]

We know all of these, so we're good.

\subsection*{Alternative Way}

Now suppose we want to look directly at

\[
P(T\cup B)=P((S_1\cap S_2)\cup S_3)
\]

From there, we might go immediately to:

\[
P(T\cup B)=P(S_1)P(S_2)+P(S_3)
\]

But that's kind of sneaky. It'd be better if we could depend on the
formulas to do these calculations. In this way, we'd want to go from:

\[
P((S_1\cap S_2)\cup S_3)
\] \[
P(S_1\cap S_2)+P(S_3)
\] \[
P(S_1)P(S_2)+P(S_3)
\]

But who knows if that's correct. Let's derive!

\subsection*{Union of Independent Events}

Assume $A$ and $B$ are independent. We want to know if:

\[
P(A\cup B)\overset{?}{=}P(A)+P(B)
\]

We always know that:

\[
P(A\cup B)=P(A)+P(B)-P(A\cap B)
\]

The question is whether, for independent events:

\[
P(A\cap B)\overset{?}{=} 0
\]

To illustrate this, we can look at rolling a die, and the events that
we roll a prime and rolling a large number. More directly, we have:

\[
\Omega=\{1,2,3,4,5,6\}
\] \[
\rho=\{2,3,5\}
\] \[
L=\{5,6\}
\]

We want to know whether these are independent, i.e.:

\[
P(\rho\cap L)\overset{?}{=}P(\rho)P(L)
\]

\[
\underbrace{\frac{1}{6}}_{P(\rho\cap L)}=\underbrace{\frac{1}{2}}_{P(\rho)}\underbrace{\frac{1}{3}}_{P(L)}
\]

Therefore, these are independent events. We can try applying our
desired equality:

\[
P(\rho\cup L)
\] \[
P(\{2,3,5,6\})
\] \[
\frac{2}{3} \ne 0
\]

Counter-example!

\section*{Example 1.4.41}

Suppose we have a factory which creates parts that must be within
specified tolerances for height, length, and width. The events
describing these are $T_h$, $T_l$, and $T_w$.

We are told that 86\% of parts are within the tolerances for width, so:

\[
P(T_w)=0.86
\]

Similarly, 80\% are within all three tolerances:

\[
P(T_hT_lT_w)=0.80
\]

\textbf{Note}: We are using a shorthand where, for events $A$ and $B$,
$AB$ is the same as $A\cap{}B$.

We are also told that 2\% are within tolerances for width and length,
but not height:

\[
P(T_wT_lT_h')=0.02
\]

And 3\% within tolerances for width and height, but not length:

\[
P(T_wT_hT_l')=0.03
\]

Similarly, we have that 92\% are within tolerances for width or height:

\[
P(T_w+T_h)=0.92
\]

\textbf{Note}: We are using a notation that for events $A$ and $B$,
$A+B$ is the same as $A\cup{}B$.

Now suppose we are asked what the probability that a part is within
tolerances for width, given that it's within tolerances for height. In
other words, we want to know:

\[
P(T_w|T_h)=?
\]

We also want to know what the probability that a part is within
tolerances for height, width, and length, given that it's within
tolerances for length and width. In other words, we want to know:

\[
P(T_lT_wT_h|T_lT_w)=
?
\]

Our first step will be to use definitions to get:

\[
P(T_w|T_h)=
\dfrac{P(T_wT_h)}{P(T_h)}
\]

And similarly:

\[
P(T_lT_wT_h|T_lT_w)=
\dfrac{P(T_lT_wT_hT_lT_w)}{P(T_lT_w)}
\]

Now we want to figure out what $T_lT_w$ is, so we can figure out the
denominator of the second question. Our process is:

\[
T_lT_w
\] \[
T_lT_w\Omega
\] \[
T_lT_w(T_h+T_h')
\] \[
T_lT_wT_h+T_lT_wT_h
\]

\textbf{Note}: This new notation makes it simple to distribute across
parentheses. Also note that $+$ now distributes as well.

Now suppose we have a collection of mutually exclusive events $\zeta$:

\[
\zeta=\{T_lT_wT_h,T_lT_wT_h'\}
\]

Using the union of mutually exclusive events formula, we have:

\[
P(T_lT_w)=P(T_lT_wT_h)+P(T_lT_wT_h')
\]

So now we have the numerator of the second problem. Now to look at the
numerator.

Because the intersection of a set with itself is that same set, we can
cancel out the remaining intersections, to get:

\[
P(T_lT_wT_hT_lT_w)=P(T_lT_wT_h)
\]

Which we know. Therefore, the final probability for problem 2 is:

\[
P(T_lT_wT_h|T_lT_w)=\dfrac{P(T_lT_wT_h)}{P(T_lT_wT_h)+P(T_lT_wT_h')}
\]

My fingers hurt. I think I'm not going to type the rest of the first
problem.

\end{document}
