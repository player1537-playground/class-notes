\documentclass{article}
\usepackage{amsmath}
\usepackage{amssymb}
\usepackage{booktabs}
\usepackage{xintexpr}

\newcommand{\T}{1}
\newcommand{\F}{0}
\newcommand{\TF}[1]{\if1#1\T\else\F\fi}
\newcommand{\xintTF}[1]{\xintifboolexpr{#1}{\T}{\F}}

\newcommand{\logicrule}[2]{
\begin{array}{l}
#1 \\
\midrule
\therefore #2 \\
\end{array}
}

\newcommand{\inv}[1]{#1^{-1}}

\renewcommand{\d}[1]{\,\textnormal{d}#1}
\newcommand{\dd}[2]{\frac{\d{#1}}{\d{#2}}}
\newcommand{\ddd}[2]{\dfrac{\d{#1}}{\d{#2}}}

\DeclareMathOperator{\var}{Var}
\DeclareMathOperator{\E}{\mathcal{E}}

\setlength\parindent{0pt}
\setlength\parskip{1em}

\begin{document}

\section*{Independent Random Variables}

Remember that, for events, we define independence as:

\[
P(AB)=P(A)P(B)
\]

Similarly, we can define independent random variables.

$\eta_1$ and $\eta_2$ are independent random variables if and only if
the following events are independent:

\[
\eta_1\le x_1, \eta_2\le x_2
\]

For all real numbers $x_1,x_2$.

Recall that these events are related to events by:

\[
\eta_1\le x_1\equiv \inv{\eta_1}\{z|z\le x_1\}
\] \[
\eta_2\le x_2\equiv \inv{\eta_2}\{z|z\le x_2\}
\]

Rants again about what it really means to be independent without
cheating. Warm and fuzzy feelings, yo.

\section*{What does independence buy us?}

We know that, for any $\eta_1,\eta_2$, then:

\[
\E(\eta_1+\eta_2)\equiv \E(\eta_1)+\E(\eta_2)
\]

However, if the variables are independent, then we also have:

\[
\E(\eta_1\cdot\eta_2)=\E(\eta_1)\cdot\E(\eta_2),\quad \text{$\eta_1$ and $\eta_2$ are independent}
\]

Additionally, if the variables are independent, then we also have:

\[
\var(\eta_1+\eta_2)=\var(\eta_1)=\var(\eta_2),\quad \text{$\eta_1$ and $\eta_2$ are independent}
\]

\end{document}
