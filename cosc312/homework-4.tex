\documentclass{article}
%\usepackage[margin=0.9in]{geometry}
\usepackage{amsmath}
\usepackage{amssymb}
\usepackage{booktabs}
\usepackage{xintexpr}
\usepackage{tikz}
\usetikzlibrary{arrows,automata}
\usepackage{subcaption}
\usepackage{float}
\usepackage{fancyhdr}

\pagestyle{fancy}
\lhead{Tanner Hobson}
\rhead{thobson2}

\newcommand{\T}{1}
\newcommand{\F}{0}
\newcommand{\TF}[1]{\if1#1\T\else\F\fi}
\newcommand{\xintTF}[1]{\xintifboolexpr{#1}{\T}{\F}}

\newcommand{\logicrule}[2]{
\begin{array}{l}
#1 \\
\midrule
\therefore #2 \\
\end{array}
}

\newcommand{\inv}[1]{#1^{-1}}

\renewcommand{\d}[1]{\,\textnormal{d}#1}
\newcommand{\dd}[2]{\frac{\d{#1}}{\d{#2}}}
\newcommand{\ddd}[2]{\dfrac{\d{#1}}{\d{#2}}}

\DeclareMathOperator{\var}{Var}
\DeclareMathOperator{\E}{\mathcal{E}}

\newcommand{\multistep}[1]{\begin{array}{rl} #1 \end{array}}
\newcommand{\subeq}{\subseteq}
\newcommand{\sub}{\subset}

\newcommand{\conj}[1]{\overline{#1}}

\newcommand{\problem}[1]{$\boxed{\textbf{#1}}$}

\setlength\parindent{0pt}
\setlength\parskip{0em}
\newenvironment{blockquote}{%
  \par%
  \medskip
  \leftskip=4em\rightskip=2em%
  \noindent\ignorespaces}{%
  \par\medskip}
\begin{document}

\begin{minipage}{\textwidth}
\problem{4.1}

Let $C$ be the language $\{w|\text{$w$ has an equal number of 0s and 1s}\}$. We aim to prove that $C$ is irregular.

\textbf{Proof}:

Assume $C$ is regular and that $p$ is the pumping length of $C$. If we
let $s=0^p1^p$, then we know that $s\in{}C$ and that $|s|\ge{}p$. By
the Pumping Lemma, we can write $s=xyz$. The PL states that we must
satisfy $|y|\ge1$, meaning that we must have $p\ge1$. Similarly, the
PL states that $|xy|\le{}p$. Because the first $p$ characters of $s$
are all 0s, we know that $y$ must also consist solely of 0s. Finally,
the PL states that we must have, for all $i\ge0$,
$xy^iz\in{}C$. However, we have reached a contradiction, because there
would be more 0s than 1s in the string $xy^iz$ where $i\ge1$. Because
we reached a contradiction, we know that our earlier premise that $C$
is regular must be false, and can conclude that $C$ is irregular.

\end{minipage}

\begin{minipage}{\textwidth}
\problem{4.2}

Let $F$ be the language $\{ww|\text{$w$ is a string from $\{0,1\}^*$}\}$. We aim to prove that $F$ is irregular.

\textbf{Proof}:

Assume that $F$ is regular and that $p$ is the pumping the length of
$F$. Consider the string $s=0^p10^p1$ where $|s|\ge{}p$. To use the
PL, we must introduce $s=xyz$ for some $|x|\ge0$, $|z|\ge0$, and
$1\le|y|\le{}p$. We know that the first $p$ characters of $s$ are all
0s, and by the PL, we know that $|xy|\le{}p$. From this, we can
conclude that $y$ must consist solely of 0s. Finally, the PL states
that for all $i\ge0$, $xy^iz\in{}F$. If we consider the case where
$i=1$, then we can see that $xyyz\not\in{}F$ because there will be
more 0s on the left half of the string than on the right. We have
reached a contradiction so we know that our premise must be wrong, and
can conclude that $F$ is irregular.

\end{minipage}

\begin{minipage}{\textwidth}
\problem{4.3}

Let $A$ be the language $\{www|\text{$w$ is a string from $\{a,b\}^*$}\}$. We aim to prove that $A$ is irregular.

\textbf{Proof}:

Assume that $A$ is regular and that $p$ is the pumping the length of
$A$. Consider the string $s=a^pba^pba^pb$ where $|s|\ge{}p$. To use
the PL, we must introduce $s=xyz$ for some $|x|\ge0$, $|z|\ge0$, and
$1\le|y|\le{}p$. We know that the first $p$ characters of $s$ are all
$a$s, and by the PL, we know that $|xy|\le{}p$. From this, we can
conclude that $y$ must consist solely of $a$s. Finally, the PL states
that for all $i\ge0$, $xy^iz\in{}A$. If we consider the case where
$i=1$, then we can see that $xyyz\not\in{}A$ because there will be
more $a$s on the left third of the string than on the rest. We have
reached a contradiction so we know that our premise must be wrong, and
can conclude that $A$ is irregular.


\end{minipage}

\begin{minipage}{\textwidth}
\problem{4.4}

We aim to find the problem with the following proof that $0^*1^*$ is
irregular.

\begin{blockquote}
  Assume that $0^*1^*$ is regular.  Let $p$ be the pumping length for
  $0^*1^*$ given by the Pumping Lemma. Choose $s$ to be the string
  $0^p1^p$.  We know that $s$ is a member of $0^*1^*$ but we showed
  that s cannot be pumped from the lecture example with the language
  $B=\{0^n1^n|n\ge1\}$. So $0^*1^*$ must not be regular also.
\end{blockquote}

The proof's mistake lies in the assumption that because $s$ cannot be
pumped by a different language, and because $s$ is in both language,
then our $0^*1^*$ language must also not be regular.

\end{minipage}

\begin{minipage}{\textwidth}
\problem{4.5}

Let $L$ be the language $\{0^n1^m0^n|m,n\ge0\}$. We aim to prove that
$L$ is irregular.

\textbf{Proof}:

Assume that $L$ is regular and that $p$ is the pumping length of
$L$. Let $s=0^p10^p$ be a string in $L$. For the pumping lemma, we
introduce $s=xyz$ for some $|x|\ge0$, $|z|\ge0$, and
$1\le|y|\le{}p$. We then let $x=0^a$, $y=0^b$, and $z=0^c10^p$, such
that $b\ge1$ (to satisfy $|y|\ge1$) and $a+b+c=p$. Clearly,
$|xy|\le{}p$ because $a+b=p-c$ and therefore $a+b\le{}p$. The PL
states that for any $i\ge0$, $xy^iz\in{}L$. We can see that this is
true for $i=1$ because this is our original string $s$ which we
already know is in the language. However, for $i=0$, we can see that
$xy^0z=0^a0^c10^p\not\in{}L$ because $a+c=p-b$ and $b\ge1$, so
therefore $a+c<p$, and the number of $0$s on either side of the $1$ is
not equal. We have reached a contradiction so our premise must be
false, and therefore $L$ is irregular.

\end{minipage}

\end{document}
