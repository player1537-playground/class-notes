\documentclass{article}
\usepackage{amsmath}
\usepackage{amssymb}
\usepackage{booktabs}
\usepackage{xintexpr}
\usepackage{forest}
\usepackage{enumitem}
\usepackage{float}
\usepackage{subcaption}

\newcommand{\T}{1}
\newcommand{\F}{0}
\newcommand{\TF}[1]{\if1#1\T\else\F\fi}
\newcommand{\xintTF}[1]{\xintifboolexpr{#1}{\T}{\F}}

\newcommand{\logicrule}[2]{
\begin{array}{l}
#1 \\
\midrule
\therefore #2 \\
\end{array}
}

\newcommand{\inv}[1]{#1^{-1}}

\renewcommand{\d}[1]{\,\textnormal{d}#1}
\newcommand{\dd}[2]{\frac{\d{#1}}{\d{#2}}}
\newcommand{\ddd}[2]{\dfrac{\d{#1}}{\d{#2}}}

\DeclareMathOperator{\var}{Var}
\DeclareMathOperator{\E}{\mathcal{E}}

\newcommand{\multistep}[1]{\begin{array}{rl} #1 \end{array}}
\newcommand{\subeq}{\subseteq}
\newcommand{\sub}{\subset}

\newcommand{\conj}[1]{\overline{#1}}

\setlength\parindent{0pt}
\setlength\parskip{1em}

\begin{document}

\section*{Regular Languages}

Use the Pumping Lemma to determine if a language is regular.

Because regular languages are hard, we want to move to grammars and
parse trees.

\section*{Grammars}

Suppose we have a grammar defined by:
$G=\left(\{E,T,F\},\{a,+,\times,(,)\},R,E\right)$. Note that $E$
recognizes an adding expression. $T$ represents a multiplying
expression. And $F$ represents a variable?

Each of the expressions can be written as:

\begin{description}
\item[$E$]$\rightarrow{}E+T|T$
\item[$T$]$\rightarrow{}T\times{}F|F$
\item[$F$]$\rightarrow{}(E)|a$
\end{description}

We might want to recognize the expression $a+a\times{}a$

\begin{forest}
[
  [$E$
    [$E$ [$T$ [$F$ [$a$]]]]
    [$+$]
    [$T$
      [$T$ [$F$ [$a$]]]
      [$\times$]
      [$F$ [$a$]]
    ]
  ]
]
\end{forest}

\begin{minipage}{\textwidth}
Similarly, the expression $(a+a)$:

\begin{forest}
[$E$
  [$T$
    [$F$
      [$($]
      [$E$
        [$E$ [$T$ [$F$ [$a$]]]]
        [$+$]
        [$T$ [$F$ [$a$]]]
      ]
      [$)$]
    ]
  ]
]
\end{forest}
\end{minipage}

\section*{Context-Free Grammar Design Techniques}

How do you build complex grammars?

\begin{enumerate}
\item Many CFGs are \textbf{unions} of simpler CFGs.
  \begin{enumerate}[label=(\alph*)]
    \item Put all rules together
    \item Construct new rules $S\rightarrow{}S_1|S_2|\cdots|S_k$ where
      $S_i$ with $1\le{}i\le{}k$ are start variables of the individual
      grammars.
  \end{enumerate}

  For example: Design a grammar for
  $\{0^n1^n|n\ge0\}\cup\{1^n0^n|n\ge0\}$

  \begin{enumerate}[label=(\arabic*)]
    \item Construct grammar for $\{0^n1^n|n\ge0\}$

      \[
      S_1\rightarrow 0 S_1 1|\epsilon
      \]

    \item Construct grammar for $\{1^n0^n|n\ge0\}$

      \[
      S_2\rightarrow 1 S_2 0|\epsilon
      \]

    \item Combine rules $S\rightarrow{}S_1|S_2$

      \[
      \begin{cases}
        S\rightarrow S_1|S_2 \\
        S_1\rightarrow 0S_11|\epsilon \\
        S_2\rightarrow 1S_20|\epsilon \\
      \end{cases}
      \]
  \end{enumerate}

\item See if you can construct a DFA first.
  \begin{enumerate}[label=(\alph*)]
    \item Make a variable $R_i$ for each state $q_i$ of DFA.
    \item Add a rule $R_i\rightarrow{}aR_j$ if $\delta(q_i,a)=q_j$ for
      some $a\in\Sigma$.
    \item Add the rule $R-i\rightarrow\epsilon$ if $q_i$ is accept
      state of DFA.
    \item If $q_0$ is the start state of DFA, make $R_0$ the start
      variable of CFG.
  \end{enumerate}

\item Certain context-free languages contain strings with related
  substrings, such as $0^n$ and $1^n$ in $\{0^n1^n|n\ge0\}$.

  Note that the machine would need to remember a potentially unbounded
  amount of information about one of the substrings.

  CFG handles this situation using rules of the form
  $R\rightarrow{}uRv$.

  For example, given the grammar

  \[
  G=\left(\{S,B\},\{a,b\},\left\{S\rightarrow{}aSB|B|\epsilon,B\rightarrow{}bB|b\right\},S\right)
  \]

  What is the language that is recognized?

  \begin{enumerate}[label=(\arabic*)]
  \item Let's do a few direct derivations:

    \[
    \boxed{S}\Rightarrow \underbrace{a\boxed{S}B}\Rightarrow a\underbrace{aS\boxed{B}}B\Rightarrow aaS\underbrace{bB}B
    \]

    \[
    S\Rightarrow aSB\Rightarrow aaSBB\Rightarrow aaSBbB
    \]

    \[
    S\Rightarrow aSB\Rightarrow aaSBB\Rightarrow aasBb
    \]

    \[
    S\Rightarrow aSB\Rightarrow aaSBB\Rightarrow aaBB
    \]

  \item $L(G)=\{a^nb^m|n\le{}m\}$

  \item We might also want a regular language, but that'd be silly
    because you can't write it that way.

  \end{enumerate}
\end{enumerate}

\section*{Context-Free Grammars: Ambiguity}

Suppose you have a CFG that generates the same string ind ifferent
ways, then we say that the string is derived \textbf{ambiguously} in
that grammar.

Similarly, a CFG is ambiguous if any string can be recognized ambiguously.

\subsection*{Example}

Suppose that we go back to our arithmetic recognizer with $G$ having the rule

\[
E\rightarrow E+E|E\times E|(E)|a
\]

If we give it the string $a+a\times{}a$

\begin{figure}[H]
  \centering
  \begin{subfigure}{.5\textwidth}
    \centering
    \begin{forest}
      [$E$
        [$E$
          [$E$ [$a$]]
          [$+$]
          [$E$ [$a$]]
        ]
        [$\times$]
        [$E$ [$a$]]
      ]
    \end{forest}
  \end{subfigure}%
  %
  \begin{subfigure}{.5\textwidth}
    \centering
    \begin{forest}
      [$E$
        [$E$ [$a$]]
        [$+$]
        [$E$
          [$E$ [$a$]]
          [$\times$]
          [$E$ [$a$]]
        ]
      ]
    \end{forest}
  \end{subfigure}
\end{figure}

We need to address the \textbf{order} of how to apply the rules.

\begin{description}
\item[Leftmost Derivation] A derivation of a string $w$ in grammar $G$
  is a leftmost derivation if at every step in the derivation, the
  \textbf{leftmost non-terminal} is replaced.

\item[Rightmost Derivation] A derivation of a string $w$ in grammar $G$
  is a rightmost derivation if at every step in the derivation, the
  \textbf{rightmost non-terminal} is replaced.
\end{description}

\subsection*{Why do we care?}

A leftmost or rightmost derivation of a string $w$ is \textbf{unique}.

\subsection*{Inherent Ambiguity}

Some CFLs can have both ambiguous and unambiguous grammars.

For example given the language: $\{0^i1^j2^k|i=j\vee{}j=k\}$

\end{document}
