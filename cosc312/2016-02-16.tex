\documentclass{article}
\usepackage{amsmath}
\usepackage{amssymb}
\usepackage{booktabs}
\usepackage{xintexpr}
\usepackage{enumitem}

\newcommand{\T}{1}
\newcommand{\F}{0}
\newcommand{\TF}[1]{\if1#1\T\else\F\fi}
\newcommand{\xintTF}[1]{\xintifboolexpr{#1}{\T}{\F}}

\newcommand{\logicrule}[2]{
\begin{array}{l}
#1 \\
\midrule
\therefore #2 \\
\end{array}
}

\newcommand{\inv}[1]{#1^{-1}}

\renewcommand{\d}[1]{\,\textnormal{d}#1}
\newcommand{\dd}[2]{\frac{\d{#1}}{\d{#2}}}
\newcommand{\ddd}[2]{\dfrac{\d{#1}}{\d{#2}}}

\DeclareMathOperator{\var}{Var}
\DeclareMathOperator{\E}{\mathcal{E}}

\newcommand{\multistep}[1]{\begin{array}{rl} #1 \end{array}}
\newcommand{\subeq}{\subseteq}
\newcommand{\sub}{\subset}

\newcommand{\conj}[1]{\overline{#1}}

\setlength\parindent{0pt}
\setlength\parskip{1em}

\begin{document}

\section*{Proving Irregularity}

We want to prove that $C=\{w|\text{$w$ has an equal number of 0s and 1s}\}$ is not regular.

\subsection*{Proof}

\begin{enumerate}
\item Assume $C$ is regular and $p$ is the pumping length for $C$.
\item Let $s=0^p1^p$ with $s\in{}C$ and $|s|\ge{}p$.
\item By the Pumping Lemma (PL), we have $s=xyz$, where $xyz\in{}C$
  for only $i\ge{}0$ (condition number 1).
\item By condition number 3 of the PL, we must have $|xy|\le{}p$.
\item Therefore, $y$ must consist of only 0s.
\item Hence $xyyz\not\in{}C$, because there are more 0s than 1s in $xyyz$.
\item Therefore, we have reached a contradiction: $s$ cannot be pumped
  and $C$ must not be regular.
\end{enumerate}

\section*{Example 2}

Prove that $F=\{ww|w\in\{0,1\}^*\}$ is irregular.

\subsection*{Proof Structure}

Let $s=0^p10^p1$. We definitely have $|s|\ge{}p$. Remember condition
3, that $|xy|\le{}p$.

\section*{Example 3}

Prove that $L=\{0^n1^m0^n|m,n\ge0\}$ is irregular.

\subsection*{Proof Structure}

Suppose that $s=0^p1^p$. Clearly, $|s|\ge{}p$. If $s=xyz$, then we can
suppose that $x=0^a$ and $y=0^b$. Finally, suppose that
$z=0^c10^p$. Where $b\ge1$ and $a+b+c=p$. But, if
$s'=xy^0z=0^{a+c}10^p\not\in{}L$. We just assumed $b\ge1$, but this
causes $b=0$ which is a contradiction.

Remember: we can have $x=\epsilon$ and $z=\epsilon$, but $y$ must
never be the empty string.

\section*{Generating a Grammar}

Give a CFG that generates:

\[
A=\{a^ib^jc^k|i=j\vee{}j=k,\, i,j,k\ge0\}
\]

Let the grammar $G$ be:

\[
G=(V,\Sigma,R,S)
\]

Our variables could then be:

\[
V=\{S,E_{ab}, E_{bc}, C, A\}
\]

Our alphabet is just:

\[
\Sigma=\{a,b,c\}
\]

Now for our rules:

\begin{description}
\item[$S$]$\rightarrow{}E_{ab}C|AE_{bc}$
\item[$E_{ab}$]$\rightarrow{}aE_{ab}b|\epsilon$
\item[$E_{bc}$]$\rightarrow{}bE_{bc}c|\epsilon$
\item[$C$]$\rightarrow{}Cc|\epsilon$
\item[$A$]$\rightarrow{}Aa|\epsilon$
\end{description}

Now, is this language ambiguous? Note that we actually have
$\epsilon\in{}A$ and that there are two ways to get the empty string.

\section*{Chomsky Normal Form}

This is a way to simplify CFGs into CNF.

Every rule in the grammar has the form:

\[
\begin{cases}
A\rightarrow B C \\
A \rightarrow a
\end{cases}
\]

where $A,B,C$ are non terminals, and $a$ is a terminal.

The rule $S\rightarrow\epsilon$ is \textbf{not excluded}.

\subsection*{Theorem}

Any CFL is generated by a CFG in CNF.

\begin{enumerate}
\item Add a new starting symbol $S_0$ and rule $S_0\rightarrow{}S$,
  where $S$ is the original starting symbol. (We don't want S on the
  right hand side of any rule).

\item Eliminate $\epsilon$-rules by repeating until all $\epsilon$-rules are removed:
  \begin{enumerate}[label=(\alph*)]
    \item Eliminate the $\epsilon$-rule $A\rightarrow\epsilon$, where
      $A$ is not the start symbol.

    \item For each occurrence of $A$ on the right hand side of any
      rule, add a new rule with that occurrence of $A$ deleted. For
      example: To delete $A\rightarrow\epsilon$, replace
      $B\rightarrow{}uAv$ by $B\rightarrow{}uAv|uv$.

    \item Replace the rule $B\rightarrow{}A$ (if it is present) with
      $B\rightarrow{}A|\epsilon$.
  \end{enumerate}
\end{enumerate}

\end{document}
