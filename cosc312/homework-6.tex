\documentclass{article}
%\usepackage[margin=0.9in]{geometry}
\usepackage{amsmath}
\usepackage{amssymb}
\usepackage{booktabs}
\usepackage{xintexpr}
\usepackage{tikz}
\usetikzlibrary{arrows,automata}
\usepackage{subcaption}
\usepackage{float}
\usepackage{fancyhdr}
\usepackage{forest}
\usepackage{enumitem}

\pagestyle{fancy}
\lhead{Tanner Hobson}
\rhead{thobson2}

\newcommand{\T}{1}
\newcommand{\F}{0}
\newcommand{\TF}[1]{\if1#1\T\else\F\fi}
\newcommand{\xintTF}[1]{\xintifboolexpr{#1}{\T}{\F}}

\newcommand{\logicrule}[2]{
\begin{array}{l}
#1 \\
\midrule
\therefore #2 \\
\end{array}
}

\newcommand{\inv}[1]{#1^{-1}}

\renewcommand{\d}[1]{\,\textnormal{d}#1}
\newcommand{\dd}[2]{\frac{\d{#1}}{\d{#2}}}
\newcommand{\ddd}[2]{\dfrac{\d{#1}}{\d{#2}}}

\DeclareMathOperator{\var}{Var}
\DeclareMathOperator{\E}{\mathcal{E}}

\newcommand{\multistep}[1]{\begin{array}{rl} #1 \end{array}}
\newcommand{\subeq}{\subseteq}
\newcommand{\sub}{\subset}

\newcommand{\conj}[1]{\overline{#1}}

\newcommand{\problem}[1]{$\boxed{\textbf{#1}}$}

\setlength\parindent{0pt}
\setlength\parskip{0em}
\newenvironment{blockquote}{%
  \par%
  \medskip
  \leftskip=4em\rightskip=2em%
  \noindent\ignorespaces}{%
  \par\medskip}

\begin{document}

\problem{6.1}

\textit{Convert the following CFG into CNF}

\[\begin{cases}
A\rightarrow BAB | B | \epsilon \\
B\rightarrow 00 | \epsilon \\
\end{cases}\]

We want to introduce a new start state. Here, we assume that $A$ is
the starting state because it is not stated elsewhere. We add $S_0$ as
the new start state:

\[\begin{cases}
S_0\rightarrow A \\
A\rightarrow BAB | B | \epsilon \\
B\rightarrow 00 | \epsilon \\
\end{cases}\]

We want to remove the non-terminals and replace them with a separate
rule. Because the only terminal is $0$, we introduce $N_0$ to stand
for it.

\[\begin{cases}
S_0\rightarrow A \\
A\rightarrow BAB | B | \epsilon \\
B\rightarrow N_0 N_0 | \epsilon \\
N_0\rightarrow 0
\end{cases}\]

Now we want to remove the $A\rightarrow{}BAB$ rule because there are
more than 2 non-terminals on the right.

\[\begin{cases}
S_0\rightarrow A \\
A\rightarrow BA_1 | B | \epsilon \\
A_1\rightarrow AB \\
B\rightarrow N_0 N_0 | \epsilon \\
N_0\rightarrow 0
\end{cases}\]

Now we want to remove $\epsilon$ rules. We'll start with the
$B\rightarrow\epsilon$ rule and for any rule that has $B$, we account
for the fact that it might be $\epsilon$.

\[\begin{cases}
S_0\rightarrow A \\
A\rightarrow BA_1 | B | A_1 | \epsilon \\
A_1\rightarrow AB | A \\
B\rightarrow N_0 N_0  \\
N_0\rightarrow 0
\end{cases}\]

Then we need to remove the $A\rightarrow\epsilon$ rule.

\[\begin{cases}
S_0\rightarrow A \\
A\rightarrow BA_1 | B | A_1 \\
A_1\rightarrow AB | A | B \\
B\rightarrow N_0 N_0  \\
N_0\rightarrow 0
\end{cases}\]

Now we need to remove unit rules. We'll start with
$A\rightarrow{}A_1$.

\[\begin{cases}
S_0\rightarrow A \\
A\rightarrow BA_1 | B | AB \\
A_1\rightarrow AB | A | B \\
B\rightarrow N_0 N_0  \\
N_0\rightarrow 0
\end{cases}\]

Then $A\rightarrow{}B$

\[\begin{cases}
S_0\rightarrow A \\
A\rightarrow BA_1 | N_0 N_0 | AB \\
A_1\rightarrow AB | A | B \\
B\rightarrow N_0 N_0  \\
N_0\rightarrow 0
\end{cases}\]

And $A_1\rightarrow{}A$

\[\begin{cases}
S_0\rightarrow A \\
A\rightarrow BA_1 | N_0 N_0 | AB \\
A_1\rightarrow AB | BA_1 | N_0 N_0 | B \\
B\rightarrow N_0 N_0  \\
N_0\rightarrow 0
\end{cases}\]

And $A_1\rightarrow{}B$

\[\begin{cases}
S_0\rightarrow A \\
A\rightarrow BA_1 | N_0 N_0 | AB \\
A_1\rightarrow AB | BA_1 | N_0 N_0 \\
B\rightarrow N_0 N_0  \\
N_0\rightarrow 0
\end{cases}\]

And finally, $S_0\rightarrow{}A$

\[\begin{cases}
S_0\rightarrow BA_1 | N_0 N_0 | AB \\
A\rightarrow BA_1 | N_0 N_0 | AB \\
A_1\rightarrow AB | BA_1 | N_0 N_0 \\
B\rightarrow N_0 N_0  \\
N_0\rightarrow 0
\end{cases}\]

\problem{6.2}

Let $G$ be the grammar:

\[\begin{cases}
S\rightarrow SS | T \\
T\rightarrow a T b | ab \\
\end{cases}\]

The language produced by this grammar, $L(G)$, contains the following
strings and more (generated from a Python program):

\[\begin{array}{c}
\text{Some words in language} \\
\midrule
ab \\ aabb \\ abab \\ aaabbb \\ abaabb \\ ababab \\ aaaabbbb \\ aabbaabb \\ aabbabab \\
abaaabbb \\ ababaabb \\ aaaaabbbbb \\ aaaabbbbab \\ aaabbbaabb \\ aabbaaabbb \\ aabbabaabb \\
abaaaabbbb \\ abaaabbbab \\ aaaaaabbbbbb \\ aaaabbbbabab \\ aaabbbababab \\ aabbaaaabbbb \\ aabbabaabbab \\
abaabbabaabb \\ aaaaaaabbbbbbb \\ aabbaabbaaabbb \\ aabbabaabbaabb \\ aaaaaaaabbbbbbbb \\ aabbababaaaabbbb \\ abaabbababaabbab \\
\end{array}\]

By extension, we can conclude that $L(G)$ is an extension of a simpler
language, $X=\{a^nb^n|n\ge1\}$ and that $L(G)=\{w|w\in{}X^*\}$.

We can also show that $G$ is ambiguous by showing that the string
$w=ababab$ can be parsed in two ways:

\begin{figure}[H]
  \centering
  \begin{subfigure}{.5\textwidth}
    \centering
    \begin{forest}
      [$S$
        [$S$ [$T$ [$ab$]]]
        [$S$
          [$S$ [$T$ [$ab$]]]
          [$S$ [$T$ [$ab$]]]
        ]
      ]
    \end{forest}
  \end{subfigure}%
  %
  \begin{subfigure}{.5\textwidth}
    \centering
    \begin{forest}
      [$S$
        [$S$
          [$S$ [$T$ [$ab$]]]
          [$S$ [$T$ [$ab$]]]
        ]
        [$S$ [$T$ [$ab$]]]
      ]
    \end{forest}
  \end{subfigure}
\end{figure}

\end{document}
