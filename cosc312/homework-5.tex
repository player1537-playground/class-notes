\documentclass{article}
%\usepackage[margin=0.9in]{geometry}
\usepackage{amsmath}
\usepackage{amssymb}
\usepackage{booktabs}
\usepackage{xintexpr}
\usepackage{tikz}
\usetikzlibrary{arrows,automata}
\usepackage{subcaption}
\usepackage{float}
\usepackage{fancyhdr}
\usepackage{forest}
\usepackage{enumitem}

\pagestyle{fancy}
\lhead{Tanner Hobson}
\rhead{thobson2}

\newcommand{\T}{1}
\newcommand{\F}{0}
\newcommand{\TF}[1]{\if1#1\T\else\F\fi}
\newcommand{\xintTF}[1]{\xintifboolexpr{#1}{\T}{\F}}

\newcommand{\logicrule}[2]{
\begin{array}{l}
#1 \\
\midrule
\therefore #2 \\
\end{array}
}

\newcommand{\inv}[1]{#1^{-1}}

\renewcommand{\d}[1]{\,\textnormal{d}#1}
\newcommand{\dd}[2]{\frac{\d{#1}}{\d{#2}}}
\newcommand{\ddd}[2]{\dfrac{\d{#1}}{\d{#2}}}

\DeclareMathOperator{\var}{Var}
\DeclareMathOperator{\E}{\mathcal{E}}

\newcommand{\multistep}[1]{\begin{array}{rl} #1 \end{array}}
\newcommand{\subeq}{\subseteq}
\newcommand{\sub}{\subset}

\newcommand{\conj}[1]{\overline{#1}}

\newcommand{\problem}[1]{$\boxed{\textbf{#1}}$}

\setlength\parindent{0pt}
\setlength\parskip{0em}
\newenvironment{blockquote}{%
  \par%
  \medskip
  \leftskip=4em\rightskip=2em%
  \noindent\ignorespaces}{%
  \par\medskip}
\begin{document}

\problem{5.1}

The following CFG has $a$ as the sole terminal and $E$ as the starting
variable. The rules are:

\[\begin{cases}
E\rightarrow E+T|T \\
T\rightarrow T\times F|F \\
F\rightarrow (E)|a \\
\end{cases}\]

\begin{minipage}{\textwidth}
\problem{5.1a}

The given string is $s=a$. The parse tree for $s$ is:

\begin{forest}
  for tree={
    grow=0,reversed,
    parent anchor=east,child anchor=west
  }
  [$E$ [$T$ [$F$ [$a$]]]]
\end{forest}

The derivation is:

\[
E\Rightarrow T\Rightarrow F\Rightarrow a
\]

\end{minipage}

\begin{minipage}{\textwidth}
\problem{5.1b}

The given string is $s=a+a$. The parse tree for $s$ is:

\begin{forest}
  for tree={
    grow=0,reversed,
    parent anchor=east,child anchor=west
  }
  [$E$
    [$E$ [$T$ [$F$ [$a$]]]]
    [$+$]
    [$T$ [$F$ [$a$]]]
  ]
\end{forest}

The derivation is:

\[
E\Rightarrow E+T\Rightarrow F+T\Rightarrow a+T\Rightarrow a+F\Rightarrow a+a
\]

\end{minipage}

\begin{minipage}{\textwidth}
\problem{5.1c}

The given string is $s=a+a+a$. The parse tree for $s$ is:

\begin{forest}
  for tree={
    grow=0,reversed,
    parent anchor=east,child anchor=west
  }
  [$E$
    [$E$
      [$E$ [$T$ [$F$ [$a$]]]]
      [$+$]
      [$T$ [$F$ [$a$]]]
    ]
    [$+$]
    [$T$ [$F$ [$a$]]]
  ]
\end{forest}

The derivation is:

\[
E\Rightarrow E+T\Rightarrow E+T+T\Rightarrow T+T+T\Rightarrow F+T+T
\] \[\Rightarrow a+T+T\Rightarrow a+F+T\Rightarrow a+a+T\Rightarrow a+a+F\Rightarrow a+a+a
\]

\end{minipage}

\begin{minipage}{\textwidth}
\problem{5.1d}

The given string is $s=((a))$. The parse tree for $s$ is:

\begin{forest}
  for tree={
    grow=0,reversed,
    parent anchor=east,child anchor=west
  }
  [$E$
    [$T$
      [$F$
        [$($]
        [$E$
          [$T$
            [$F$
              [$($]
              [$E$ [$T$ [$F$ [$a$]]]]
              [$)$]
            ]
          ]
        ]
        [$)$]
      ]
    ]
  ]
\end{forest}

The derivation is:

\[
E\Rightarrow T\Rightarrow F\Rightarrow (E)\Rightarrow (T)\Rightarrow(F)\Rightarrow((E))\Rightarrow((T))\Rightarrow((F))\Rightarrow((a))
\]

\end{minipage}

\begin{minipage}{\textwidth}
\problem{5.2a}

We want to produce a grammar which matches the language $L$ where

\[
L=\{w|\text{$w$ starts and ends with the same symbol}\}
\]

The grammar is given by the 4 tuple $G=(\{S,X\},\{0,1\},R,S)$, where
$R$ is:

\[\begin{cases}
S\rightarrow 0X0|1X1 \\
X\rightarrow 0X|1X|\epsilon \\
\end{cases}\]

\end{minipage}

\begin{minipage}{\textwidth}
\problem{5.2b}

We want to produce a grammar which matches the language $L$ where

\[
L=\{w|\text{the length of $w$ is odd}\}
\]

The grammar is given by the 4 tuple $G=(\{S,X,Z\},\{0,1\},R,S)$, where
$R$ is:

\[\begin{cases}
S\rightarrow XZ \\
X\rightarrow 0|1 \\
Z\rightarrow XXZ|\epsilon \\
\end{cases}\]

\end{minipage}

\begin{minipage}{\textwidth}
\problem{5.2c}

We want to produce a grammar which matches the language $L$ where

\[
L=\{w|\text{$w$ is a palindrome}\}
\]

The grammar is given by the 4 tuple $G=(\{S\},\{0,1\},R,S)$, where
$R$ is:

\[\begin{cases}
S\rightarrow 0S0|1S1|\epsilon \\
\end{cases}\]

\end{minipage}


\problem{5.3}

Let $G=(\{S,T,U\},\{0,\#\},R,S)$ be a grammar with $R$ given as:

\[\begin{cases}
S\rightarrow TT|U \\
T\rightarrow 0T|T0|\# \\
U\rightarrow 0U00|\E \\
\end{cases}\]

\begin{minipage}{\textwidth}
\problem{5.3a}

In English (and regular expressions), $L(G)$ is either:

\begin{enumerate}[label=(\alph*)]
\item matched by the regular expression $0^*\#0^*\#0^*$
\item has twice as many $0$s on the right of $\#$ as on the left,
  i.e. is in the language $\{0^n\#0^{2n}|n\ge0\}$
\end{enumerate}

\end{minipage}

\begin{minipage}{\textwidth}
\problem{5.3b}

We want to prove that $L(G)$ is irregular.

Assume that $L(G)$ is regular and that $p$ is the pumping length of
$L(G)$. Let $s=0^p\#0^{2p}$ be a string that is in $L(G)$. By the
pumping lemma, if we decompose $s$ into parts where $s=xyz$ and if
some conditions hold, then we know that $xy^iz\in{}L(G)$ for all
$i\ge0$ should be true for a regular language. By the conditions of
the pumping lemma, we know that $|y|\ge1$ and $|xy|\le{}p$. From this,
we can conclude that $y$ must consist solely of $0$s. The above
property means that we should expect $xy^2z\in{}L(G)$, however because
$y$ is all $0$s, this will unbalance the left and right sides of the
$\#$. Because we have reached a contradiction, we can conclude that
our earlier premise must be incorrect. Therefore, $L(G)$ is irregular.

\end{minipage}

\end{document}
