\documentclass{article}
\usepackage{amsmath}
\usepackage{pgfplots}
\usepackage{xintexpr}
\usepackage{booktabs}

\newcommand\T{1}
\newcommand\F{0}
\newcommand\TF[1]{\if1#1\T\else\F\fi}

\setlength\parindent{0pt}
\setlength\parskip{1em}
\pgfkeys{/pgfplots/MyAxisStyle/.style={xmin=-10,xmax=10, ymin=-10,ymax=10,height=6cm,width=6cm}}

\begin{document}

\section*{Section 2.1}

\begin{tabular}{rl}
Let & p: ``Jill will win the chemistry prize'' \\
    & q: ``Jack will win the math prize''
\end{tabular}

Truth values for statements:

\begin{tabular}{rcl}
0 & : & False \\
1 & : & True
\end{tabular}

\[
\begin{array}{ *{8}{c} }
p & q & \neg p & p \vee q & \neg p \rightarrow q & p \rightarrow q & p\rightarrow (p\vee q) & p\wedge(\neg p\wedge q) \\
\midrule
\xintFor #1 in {1,0}\do {
  \xintFor #2 in {1,0}\do {
    #1 &
    #2 &
    \xintifboolexpr {not(#1)}{\T}{\F} &
    \xintifboolexpr {#1 | #2}{\T}{\F} &
    \xintifboolexpr {not(not(#1)) | #2}{\T}{\F} &
    \xintifboolexpr {not(#1) | #2}{\T}{\F} &
    \xintifboolexpr {not(#1) | (#1 | #2)}{\T}{\F} &
    \xintifboolexpr {#1 & (not(#1) | #2)}{\T}{\F}
    \\
  }}
\end{array}
\]

A tautology is when all the truth values of an expression are true,
and can be shortened to $T_0$.

\section*{Preview of ``Valid Argument''}

An argument is essentially an ``if-then'' implication made up of given
statements to assume and a conclusion statement.

\[
\underset{\text{``premises''}}{\left(p_1\wedge p_2\wedge\cdots\wedge p_n\right)} \rightarrow \underset{\text{``conclusion''}}{q}
\]

An argument is valid if true premises drive a true conclusion.

Consider: $\underset{\text{premises}}{\left(\neg p\wedge\neg q\right)} \rightarrow \underset{conclusion}{\left(p\vee q\right)}$

For $p=0$, $q=0$, the premises are true ($\neg p$ and $\neg q$), but
the conclusion ($p \vee q$) is false, so this is not a valid argument.

We'll revisit this in section 2.3.

Note: If the implication $\left(p_1\wedge p_2\wedge \cdots \wedge p_n\right) \rightarrow q$
is a tautology, then it is a valid argument.

\section*{2.2}

Definition 2.2: Logically Equivalent Statements (L.E.)

Statements $S_1$ and $S_2$ are L.E. if $S_1\rightarrow S_2$ and
$S_2\rightarrow S_1$ (i.e. if $S_1\leftrightarrow S_2$ is a $T_0$), or
if $\neg S_1 \rightarrow \neg S_2$ and $\neg S_2 \rightarrow \neg S_1$
(i.e. if $\neg S_1 \leftrightarrow \neg S_2$ is a $T_0$).

\[
\begin{array}{ *{6}{c} }
p & q & \overset{S_1}{\neg\left(p\vee q\right)} & \overset{S_2}{\neg p\wedge\neg q} & S_1 \leftrightarrow S_2 & \neg S_1\leftrightarrow\neg S_2 \\
\midrule
\xintFor #1 in {0,1} \do {
  \xintFor #2 in {0,1} \do {
    \TF#1 &
    \TF#2 &
    \xintifboolexpr {not(#1 | #2)}{\T}{\F} &
    \xintifboolexpr {not(#1) & not(#2)}{\T}{\F} &
    1 &
    1 \\
  }}
\end{array}
\]

\end{document}
