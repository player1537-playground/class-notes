\documentclass{article}
%\usepackage[margin=0.75in]{geometry}
\usepackage{amsmath}
\usepackage{amssymb}
\usepackage{booktabs}
\usepackage{xintexpr}
\usepackage{multicol}
\usepackage{forest}

\setlength{\columnsep}{0.5cm}
\setlength{\columnseprule}{1pt}

\newcommand{\T}{\text{True}}
\newcommand{\F}{\text{False}}
\newcommand{\TT}{1}
\newcommand{\FF}{0}
\newcommand{\TF}[1]{\if1#1\TT\else\FF\fi}
\newcommand{\xintTF}[1]{\xintifboolexpr{#1}{\TT}{\FF}}

\newcommand{\logicrule}[2]{
\begin{array}{l}
#1 \\
\midrule
\therefore #2 \\
\end{array}
}

\newcommand{\problem}[2]{$\boxed{\text{#1 \##2}}$}
\newcommand{\subproblem}[1]{$\boxed{\text{(#1)}}$}
\newcommand{\subsolution}[2]{\boxed{#2\quad(\text{#1})}}
\newcommand{\solution}[1]{\boxed{#1}}
\newcommand{\RUS}{Rule of Universal Specification}
\newcommand{\RUG}{Rule of Universal Generalization}
\newcommand{\ROP}{Rule of Product}
\newcommand{\ROS}{Rule of Sums}

\newcommand{\multistep}[1]{\begin{array}{rl} #1 \end{array}}
\newcommand{\subeq}{\subseteq}
\newcommand{\sub}{\subset}

\setlength\parindent{0pt}
\setlength\parskip{0em}

\begin{document}

%
\problem{4.1}{2a}

Let $A=\{1,2,3\}$ and $B=\{2,4,5\}$. We want to find relations from
$A$ to $B$, which are defined by:

\[
R\subseteq (A\times B)
\]

where

\[
A\times B=\{\langle a,b\rangle|a\in A,b\in B\}
\]

So three relations are:

\[
\solution{
\multistep{
R_1&=\{\langle 1,2\rangle, \langle 1,4\rangle, \langle 2,4\rangle\} \\
R_2&=\{\langle 1,5\rangle, \langle 3,5\rangle\} \\
R_3&=\{\langle 2,2\rangle, \langle 3,5\rangle,\langle 3,2\rangle\} \\
}
}
\]

%
\problem{4.1}{8}

We want to represent the sample space of the testing process where we
finish when we find two defective chips or when we test five different
chips. We label good chips as $G$ and bad chips as $B$.

\resizebox{\textwidth}{!}{
\begin{forest}
  [\text{.}
    [$\langle{}B\rangle$
      [$\langle{}B{,}B\rangle$]
      [$\langle{}B{,}G\rangle$
        [$\langle B{,} G{,} B\rangle$]
        [$\langle B{,} G{,} G\rangle$
          [$\langle B{,} G{,} G{,} B\rangle$]
          [$\langle B{,} G{,} G{,} G\rangle$
            [$\langle B{,} G{,} G{,} G{,} B\rangle$]
            [$\langle B{,} G{,} G{,} G{,} G\rangle$]
          ]
        ]
      ]
    ]
    [$\langle G\rangle$
      [$\langle G{,} B\rangle$
        [$\langle G{,} B{,} B\rangle$]
        [$\langle G{,} B{,} G\rangle$
          [$\langle G{,} B{,} G{,} B\rangle$]
          [$\langle G{,} B{,} G{,} G\rangle$
            [$\langle G{,} B{,} G{,} G{,} B\rangle$]
            [$\langle G{,} B{,} G{,} G{,} G\rangle$]
          ]
        ]
      ]
      [$\langle G{,} G\rangle$
        [$\langle G{,} G{,} B\rangle$
          [$\langle G{,} G{,} B{,} B\rangle$]
          [$\langle G{,} G{,} B{,} G\rangle$
            [$\langle G{,} G{,} B{,} G{,} B\rangle$]
            [$\langle G{,} G{,} B{,} G{,} G\rangle$]
          ]
        ]
        [$\langle G{,} G{,} G\rangle$
          [$\langle G{,} G{,} G{,} B\rangle$
            [$\langle G{,} G{,} G{,} B{,} B\rangle$]
            [$\langle G{,} G{,} G{,} B{,} G\rangle$]
          ]
          [$\langle G{,} G{,} G{,} G\rangle$
            [$\langle G{,} G{,} G{,} G{,} B\rangle$]
            [$\langle G{,} G{,} G{,} G{,} G\rangle$]
          ]
        ]
      ]
    ]
  ]
\end{forest}
}

%
\problem{4.1}{14}

%
\subproblem{a} We want to provide a recursive definition for the
relation $R\subseteq{}\mathbb{Z}^+\times\mathbb{Z}^+$ where
$(m,n)\in{}R$ iff $m\ge{}n$.

We start with the base case, where we use the smallest element of
$\mathbb{Z}^+$ for both $m$ and $n$. In this case, that would be:

\[
(1,1)\in R
\]

Then we go with the second case that if we add 1 to both elements of
the tuple, then the inequality still holds:

\[
(i,j)\in R\Rightarrow (i+1,j+1)\in R
\]

And lastly, we note that we can add 1 to the first element of the
tuple and the inequality will still hold:

\[
(i,j)\in R\Rightarrow (i+1,j)\in R
\]

Overall, this gives us a recursive definition:

\[
\subsolution{a}{
\begin{array}{ll}
(1) & (1,1)\in R \\
(2) & (i,j)\in R\Rightarrow (i+1,j+1)\in R \\
(3) & (i,j)\in R\Rightarrow (i+1,j)\in R \\
\end{array}
}
\]

%
\subproblem{b} Now we want to show, using our recursive definition,
that $(5,2)\in{}R$ and $(4,4)\in{}R$.

\[
\begin{array}{lll}
\text{Rule} & \text{Premise} & \text{Conclusion} \\
\midrule
(1) & & (1,1)\in R \\
(2) & (1,1)\in R & (2,2)\in R \\
(3) & (2,2)\in R & (3,2)\in R \\
(3) & (3,2)\in R & (4,2)\in R \\
(3) & (4,2)\in R & \subsolution{b}{(5,2)\in R} \\
\end{array}
\]

Likewise, for $(4,4)\in{}R$:

\[
\begin{array}{lll}
\text{Rule} & \text{Premise} & \text{Conclusion} \\
\midrule
(1) & & (1,1)\in R \\
(2) & (1,1)\in R & (2,2)\in R \\
(2) & (2,2)\in R & (3,3)\in R \\
(2) & (3,3)\in R & \subsolution{b}{(4,4)\in R} \\
\end{array}
\]

%
\problem{4.1}{18}

%
\problem{4.1}{24}

%
\problem{4.2}{12}

%
\problem{4.2}{16}


\end{document}
