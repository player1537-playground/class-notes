\documentclass{article}
%\usepackage[margin=0.75in]{geometry}
\usepackage{amsmath}
\usepackage{amssymb}
\usepackage{booktabs}
\usepackage{xintexpr}
\usepackage{multicol}
\usepackage{forest}

\setlength{\columnsep}{0.5cm}
\setlength{\columnseprule}{1pt}

\newcommand{\T}{\text{True}}
\newcommand{\F}{\text{False}}
\newcommand{\TT}{1}
\newcommand{\FF}{0}
\newcommand{\TF}[1]{\if1#1\TT\else\FF\fi}
\newcommand{\xintTF}[1]{\xintifboolexpr{#1}{\TT}{\FF}}

\newcommand{\logicrule}[2]{
\begin{array}{l}
#1 \\
\midrule
\therefore #2 \\
\end{array}
}

\newcommand{\problem}[2]{$\boxed{\text{#1 \##2}}$}
\newcommand{\subproblem}[1]{$\boxed{\text{(#1)}}$}
\newcommand{\subsolution}[2]{\boxed{#2\quad(\text{#1})}}
\newcommand{\solution}[1]{\boxed{#1}}
\newcommand{\RUS}{Rule of Universal Specification}
\newcommand{\RUG}{Rule of Universal Generalization}
\newcommand{\ROP}{Rule of Product}
\newcommand{\ROS}{Rule of Sums}

\newcommand{\multistep}[1]{\begin{array}{rl} #1 \end{array}}
\newcommand{\subeq}{\subseteq}
\newcommand{\sub}{\subset}

\setlength\parindent{0pt}
\setlength\parskip{0em}

\begin{document}

\section*{Tanner Hobson}

%
\problem{4.1}{2a}

We want to prove that the statement $S(n)$ for all $n\ge1$ where

\[
S(n): \sum\limits_{i=1}^n 2^{i-1}=2^n - 1
\]

We start by proving that this holds for the base case, $n=1$, and
expanding the expression for $S(1)$, we get:

\[
\multistep{
S(1):&\sum\limits_{i=1}^1 2^{i-1}=2^1-1 \\
&2^0=2^1-1 \\
&1=1 \\
}
\]

Therefore we have shown that the statement $S(n)$ for $n=1$ is
true. Next we need to show that, if $S(k)$ is true for some arbitrary
but particular $k\ge1$, then $S(k+1)$ is also true. $S(k)$ expands to:

\[
S(k):\sum\limits_{i=1}^k 2^{i-1}=2^k-1
\]

And $S(k+1)$ as:

\[
S(k+1):\sum\limits_{i=1}^{k+1} 2^{i-1}=2^{k+1}-1
\]

We can expand the sum of the $S(k+1)$ expression as:

\[
S(k+1):2^k+\sum\limits_{i=1}^k 2^{i-1}=2^{k+1}-1
\]

However, we recognize that the sumnation is equivalent to the LHS of
the $S(k)$ statement, so we can replace it with the RHS.

\[
S(k+1):2^k+2^k-1=2^{k+1}-1
\]

This can be simplified with:

\[
\multistep{
S(k+1):&2(2^k+2^k)-1=2^{k+1}-1 \\
&2^{k+1}-1=2^{k+1}-1 \\
}
\]

We recognize that this statement is true. Because we chose $k$
arbitrarily, then by the principle of induction, we have proven that
the statement $S(n)$ is true for any $n\ge1$.

%
\problem{4.1}{8}

We want to determine for which $n\ge1$ that $S(n)$ is true, where
$S(n)$ is defined as:

\[
S(n):\sum\limits_{i=1}^{2n} i=\sum\limits_{i=1}^n i^2
\]

We can try different values of $n$ and find which it's true for:

\[
\begin{array}{ll}
i & S(i) \\
\midrule
1 & 3=1 \\
2 & 10=5 \\
3 & 21=14 \\
4 & 36=30 \\
5 & 55=55 \\
\end{array}
\]

Therefore, we have shown that $S(n)$ is true for $n=5$.

%
\problem{4.1}{14}

We want to prove for all $n\in\mathbb{Z}^+$ that the statement $S(n)$
holds, where $S(n)$ is:

\[
S(n):n>3\Rightarrow 2^n<n!
\]

The only times we need to check that the RHS of $S(n)$ is true is when
the premise is true, because the only time a contradiction can be
reached is when a true premise leads to a false conclusion, and the
premise is true when $n>3$. The first $n>3$ such that
$n\in\mathbb{Z}^+$ is $n=4$, so $n=4$ is our base case. $S(4)$ expands
to:

\[
\multistep{
S(4):&4>3\Rightarrow 2^4<4! \\
&4>3\Rightarrow 16<24 \\
}
\]

Which is a true statement. Now we choose an arbitrary, but particular
$k\ge4$ and want to verify whether $S(k)\Rightarrow{}S(k+1)$. We can
expand both of these as:

\[
S(k):k>3\Rightarrow 2^k<k!
\]

and

\[
S(k+1):k+1>3\Rightarrow 2^{k+1}<(k+1)!
\]

We can expand $2^{k+1}$ and $(k+1)!$ as:

\[
S(k+1):k+1>3\Rightarrow 2\cdot2^k<(k+1)k!
\]

We can then replace $2^k$ with $k!$ from $S(k)$.

\[
S(k+1):k+1>3\Rightarrow 2\cdot k!<(k+1)k!
\]

Because we assume that $k\ge4$, then we also know that $2<k+1$, and
therefore the inequality holds. Since we chose $k$ arbitrarily, then
we can conclude that the statement $S(n)$ is true for any
$n\in\mathbb{Z}^+$.

%
\problem{4.1}{18}

Given the equations:

\[
\begin{array}{lrcl}
1) & 1&=&1 \\
2) & 2+3+4&=&1+8 \\
3) & 5+6+7+8+9&=&8+27 \\
4) & 10+11+12+13+14+15+16&=&27+64 \\
\end{array}
\]

We want to determine an open statement $S(n)$ such that $S(1)$ is the
first equation, $S(2)$ is the second, etc, and then prove that $S(n)$
is true for all $n\ge1$.

%
\subproblem{a} We first recognize that the LHS of $S(n)$ must have the
form:

\[
\sum\limits_{i=f(n)}^{g(n)} i
\]

for some $f(n)$ and $g(n)$. We can recognize that the lower bound is
$f(n)=1+(n-1)^2$. Similarly, the upper bound is $g(n)=n^2$. This means
the LHS of $S(n)$ is actually:

\[
\sum\limits_{i=1+(n-1)^2}^{n^2} i
\]

The RHS of $S(n)$ can be written as:

\[
(n-1)^3+n^3
\]

Ultimately, this makes $S(n)$:

\[
\subsolution{a}{S(n):\sum\limits_{i=1+(n-1)^2}^{n^2} i=(n-1)^3+n^3}
\]

%
\subproblem{b}

%
\problem{4.1}{24}

Let the sequence $a_1,a_2,\cdots$ be defined by: $a_1=1$, $a_2=2$,
$a_n=a_{n-1}+a_{n-2},n\ge3$.

%
\subproblem{a} The next five elements in the sequence are:

\[
\subsolution{a}{
\begin{array}{lll}
i & a_i \\
\midrule
1 & 1 \\
2 & 2 \\
3 & 3 \\
4 & 5 \\
5 & 8 \\
6 & 13 \\
7 & 21 \\
\end{array}
}
\]

%
\subproblem{b} Now we want to prove for all $n\ge1$ that the statement
$S(n)$ is true, where:

\[
S(n):a_n<(7/4)^n
\]

We first recognize that we need two base cases, $S(1)$ and $S(2)$,
where:

\[
\multistep{
S(1):&a_1<(7/4)^1 \\
S(2):&a_2<(7/4)^2 \\
}
\]

Which we can see is true, because $a_1=1<(7/4)$ and
$a_2=2<(7^2/4^2)$. Then, for some arbitrary but particular $k\ge3$
where $S(k-2)$ and $S(k-1)$ are true, we want to show that $S(k)$ must
be true. We can expand $S(k-2)$, $S(k-1)$, and $S(k)$ as:

\[
\multistep{
S(k-2):&a_{k-2}<(7/4)^{k-2} \\
S(k-1):&a_{k-1}<(7/4)^{k-1} \\
S(k):&a_k<(7/4)^k \\
}
\]

We first recognize that, based on the definition of the sequence of
numbers, that $a_k$ can be rewritten as $a_{k-2}+a_{k-1}$:

\[
S(k):a_{k-2}+a_{k-1}<(7/4)^k
\]

We can then use $S(k-2)$ and $a_{k-2}$ as well as $S(k-1)$ and
$a_{k-1}$ to write:

\[
S(k):a_{k-2}+a_{k-1}<(7/4)^{k-2}+(7/4)^{k-1}
\]

Next we need to show that $(7/4)^{k-2}+(7/4)^{k-1}$ is bounded above
by $(7/4)^k$:

\[
(7/4)^{k-2}+(7/4)^{k-1}<(7/4)^k
\]

If we graph the functions:

\[
\begin{cases}
y_1=(7/4)^{x-2}+(7/4)^{x-1} \\
y_2=(7/4)^x \\
\end{cases}
\]

Then we see that $y_2>y_1$ for all values of $x$, which means that
$(7/4)^{k-2}+(7/4)^{k-1}<(7/4)^k$, and therefore we have shown that:

\[
S(k):a_{k-2}+a_{k-1}<(7/4)^{k-2}+(7/4)^{k-1}<(7/4)^k
\]

Because we chose $k$ arbitrarily, we have proven that the statement is
true for all $k\ge1$.

%
\problem{4.2}{12}

Let the fibonacci numbers be defined by: $F_0=0$, $F_1=1$,
$F_n=F_{n-2}+F_{n-1},n\ge2$. We want to show that the $n$th fibonacci
number is equivalent to $F_{n+2}-1$. Let the statement $S(n)$
represent this:

\[
S(n):F_n=F_{n+2}-1
\]

First we want to use two base cases $S(0)$ and $S(1)$:

\[
\multistep{
S(0):&F_0=F_2-1 \\
S(1):&F_1=F_3-1 \\
}
\]

which is equivalent to:

\[
\multistep{
S(0):&0=1-1 \\
S(1):&1=2-1 \\
}
\]

These statements are true. Now we can look at the statement $S(k-2)$,
$S(k-1)$, and $S(k)$ for some arbitrary but particular $k\ge2$. These
expand to:

\[
\multistep{
S(k-2):&F_{k-2}=F_{k}-1 \\
S(k-1):&F_{k-1}=F_{k+1}-1 \\
S(k):&F_{k}=F_{k+2}-1 \\
}
\]

We then recognize that we can rewrite $F_k$ given the definition for
fibonacci numbers:

\[
S(k):F_{k-2}+F_{k-1}=F_{k+2}-1
\]

Using $S(k-2)$, we can replace $F_{k-2}$ with $F_{k}-1$:

\[
S(k):F_{k}-1+F_{k-1}=F_{k+2}-1
\]

Similarly, we can use $S(k-1)$ to replace $F_{k-1}$ with $F_{k+1}-1$:

\[
S(k):F_k-1+F_{k+1}-1=F_{k+2}-1
\]

We can then recognize that $F_k+F_{k+1}$ is the definition of
$F_{k+2}$:

\[
S(k):F_{k+2}-2=F_{k+2}-1
\]

If these were the same, we could conclude that, because $k$ was
arbitrary, then we have shown that $S(k)$ is true for any $k\ge0$.

%
\problem{4.2}{16}

We want to find a recursive definition for (a) positive even integers
and (b) nonnegative even integers.

%
\subproblem{a}

\[
\multistep{
1) & 2\in S \\
2) & \forall x\in S(x+2\in S) \\
}
\]

%
\subproblem{b}

\[
\multistep{
1) & 0\in S \\
2) & \forall x\in S(x+2\in S) \\
}
\]

\end{document}
