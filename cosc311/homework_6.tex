\documentclass{article}
\usepackage[margin=0.75in]{geometry}
\usepackage{amsmath}
\usepackage{amssymb}
\usepackage{booktabs}
\usepackage{xintexpr}
\usepackage{multicol}

\setlength{\columnsep}{0.5cm}
\setlength{\columnseprule}{1pt}

\newcommand{\T}{1}
\newcommand{\F}{0}
\newcommand{\TF}[1]{\if1#1\T\else\F\fi}
\newcommand{\xintTF}[1]{\xintifboolexpr{#1}{\T}{\F}}

\newcommand{\logicrule}[2]{
\begin{array}{l}
#1 \\
\midrule
\therefore #2 \\
\end{array}
}

\newcommand{\problem}[2]{$\boxed{\text{#1 \##2}}$}
\newcommand{\subproblem}[1]{$\boxed{\text{(#1)}}$}
\newcommand{\subsolution}[2]{\boxed{#2\quad(\text{#1})}}
\newcommand{\RUS}{Rule of Universal Specification}
\newcommand{\RUG}{Rule of Universal Generalization}
\newcommand{\ROP}{Rule of Product}
\newcommand{\ROS}{Rule of Sums}

\setlength\parindent{0pt}
\setlength\parskip{0em}

\begin{document}
\begin{multicols*}{2}

%
\problem{1.1}{4}

There are $n=10$ board members, denoted:

\[
A=\langle{}a_1,a_2,\cdots,a_n\rangle
\]

%
\subproblem{a} We are looking to fill $r=4$ distinct positions:

\[
B=\langle{}b_1,b_2,\cdots,b_r\rangle, \quad b_i\in\{1,2,\cdots,n\}
\]

Where $b_1$ is the president's position, $b_2$ is the vice president,
$b_3$ is the secretary, and $b_4$ is the treasurer. Because each
position is distinct, then the order of each $b_i$ matters, so we are
looking at permutations. So we can directly calculate $P(n,r)$:

\[
P(n,r)=\frac{n!}{(n-r)!}
\] \[
P(10, 4)=\frac{10!}{6!}
\] \[
\subsolution{a}{P(10, 4)=5040}
\]

%
\subproblem{b i} Suppose that $a_1$, $a_2$, and $a_3$ are the $k=3$
members which are physicians. The event that a physician is nominated
for president can be written as:

\[
b_1\le k
\]

This means that we have 3 choices for the first position, 9 for the
second, 8 for the third, and 7 for the last. By the \ROP, this means
we have $kP(n-1,r-1)$ slates:

\[
\subsolution{b i}{3P(9,3)=1512}
\]

%
\subproblem{b ii} The event that exactly one physician is nominated
for the board of directors can be summarized as:

\[
1=\sum\limits_i \lbrack b_i\le k\rbrack
\]

Suppose we choose one of the 4 $b_i$ positions to be the sole
physician and then choose the remaining ones from the non-physician
pool. This means there are 3 options for the first position, 6 for the
second, 5 for the third, and 4 for the last. This means there are
$kP(n-k-1,r-1)$ slates that match this event:

\[
\subsolution{b ii}{3P(6,3)=360}
\]

%
\subproblem{b iii} The event that at least one physician is on the
board of directors can be summarized by:

\[
1\le \sum\limits_i \lbrack b_i\le k\rbrack
\]

Suppose we choose one position to be filled from the pool of
physicians, and then choose the remaining positions from all the other
members. Then the first position has 3 options, the second has 9
options, the third has 8 options, and the fourth has 7 options. This
means there are $kP(n-1,r-1)$ slates that satisfy the event:

\[
\subsolution{b iii}{3P(9,3)=1512}
\]

%
\problem{1.1}{6}

The runners can be written as:

\[
X=\langle x_1,x_2,\cdots,x_n\rangle
\]

The trophies as:

\[
T=\langle t_1,t_2,\cdots,t_r\rangle,\quad t_i\in\{1,2,\cdots,n\}
\]

Where $\forall{}i\forall{}j((i\ne{}j)\rightarrow(t_i\ne{}t_j))$.

%
\subproblem{a} Suppose $n=30$ and $r=8$. Because $T$ is a tuple, then
order matters, so we are looking at a permutation. This means there
are $P(n,r)$ ways the trophies can be handed out:

\[
\subsolution{a}{P(30,8)=235,989,936,000}
\]

%
\subproblem{b} In this problem, we are looking for the event that
$m=2$ out of the $n=30$ total runners are in one of the first $r=3$
spots of $T$. Assuming, without loss of generality, that the $m$
runners are the first runners in $X$, then this can be written as:

\[
m=\sum\limits_{i=1}^r \lbrack T_i\le m\rbrack
\]

\hspace{2em} One way we can go about calculating this is to look at the three
separate events that the first place winner is not one the interesting
runners, or the second place winner is not one of the interesting
ones, and so on.

\hspace{2em} If the first place runner is uninteresting, then we have $n-m=28$
possible choices for the uninteresting one. Then we have $m=2$ choices
for the second place, interesting runner, and $m-1=1$ choice for the
third place winner. Together, this means there are $28\cdot2\cdot1=56$
ways to have our two interesting runners in the top 3 places, where
neither runner is in first place.

\hspace{2em} Similarly, if the second place winner is uninteresting, then we have:
first place is one of the $m=2$ interesting runners, second place is
one of the $n-m=28$ uninteresting runners, and third has $m-1=1$
interesting runners. Again, this comes to $2\cdot28\cdot1=56$
possibilities.

\hspace{2em} Lastly, if the third place winner is uninteresting, we will also
arrive at the same value: $56$ possibilities.

\hspace{2em} Because each of these events cannot be performed
simultaneously, then by the \ROS, we can add up their possibilities to
get the total number of possibilities that the top 3 places will
contain our two interesting runners, which is:

\[
\subsolution{b}{3\cdot28\cdot2\cdot1=168}
\]

%
\problem{1.1}{10}

Let the $k=2$ shelves be represented by $S$, where:

\[
S=\langle s_1,s_2,\cdots,s_k\rangle
\]

Suppose there is an initial, unique ordering of books. Let the tuple
$B$ represent which of the $k=2$ shelves each of the $n=15$ books are
on, where;

\[
B=\langle b_1,b_2,\cdots,b_n\rangle,\quad b_i\in S
\]

Subject to the condition

\[
\forall s\in S\left(1\le\sum\limits_{j=1}^n\lbrack b_j=s\rbrack\right)
\]

Then suppose we have another tuple $R$ which denotes a reordering of
$B$:

\[
R=\langle r_1,r_2,\cdots,r_n\rangle,\quad r_i\in\{1,2,\cdots,n\}
\]

Subject to the condition:

\[
\forall i\forall j((i\ne j)\rightarrow(r_i\ne r_j))
\]

In other words, $R$ is a permutation of the set $\{1,2,\cdots,n\}$,
and we can define another tuple $B^*$ as

\[
B^*=\langle b^*_1,b^*_2,\cdots,b^*_n\rangle,\quad b^*_i=b_{r_i}
\]

From here, we can look at the properties of the $B$ tuple and later
apply them to each of the possible $B^*$ tuples. Without loss of
generality, we can let the first $k$ elements of $B$ be each of the
elements of $S$ in turn:

\[
B=\langle s_1, s_2,\cdots,s_k,b_{k+1},b_{k+2},\cdots,b_n\rangle
\]

From here, we can choose each of the $b_{k+1},b_{k+2},\cdots,b_n$
elements from $S$ randomly. This means for each of those $n-k$
elements, we have $k$ different choices for them. Overall, we can
write the total number of ways to choose $B$ as:

\[
\prod\limits_{i=k+1}^n k=k^{n-k}
\]

For each of these choices for $B$, we also have a unique set of
choices for $B^*$. It is important to note that even though for
different choices of $B$, there may be two values for $B^*$ that are
equivalent, these actual represent different orderings of the books on
their respective shelves, and therefore

\textbf{Note}: I fucked this up. We might end up with two $B^*$ arrays
that actually do mean the same thing, because the top shelf of both
might have books $\langle{}1,2,3\rangle$ in that order, and this is
indistinguishable from each other (aka, there's double counting going
on).

I'm going to have to pick a slightly different model, where I focus
solely on the case for $k=2$, and account for the fact that we can
condition the model on another variable, $m$, for the number of books
on shelf 1, and then determine how many different ways those can be
chosen from the array of books we have (and similarly, for the other
shelf). This might actually help turn the problem into a more
combinatoric one, because I'll have to use $P(n,r)$ and $C(n,r)$.

%
\problem{1.1}{16}

The alphabet of $n=40$ symbols can be represented as:

\[
A=\{a_1,a_2,\cdots,a_n\}
\]

%
\subproblem{a} If we have $k=25$ symbols in a message, and symbols can
be repeated, then the message can be represented as:

\[
M=\langle m_1,m_2,\cdots,m_k\rangle,\quad \forall i(m_i\in A)
\]

Because each symbol in $M$ can be repeated, every option has the same
number of possibilities:

\[
\prod\limits_{i=1}^k n=n^k
\] \[
\subsolution{a}{40^{25}}
\]

\subproblem{b} We can partition our alphabet into two sets of size
$m=30$ and $n-m=10$:

\[
A^{(1)}=\{a_1,a_2,\cdots,a_m\}
\]

The message can now be represented, with the restrictions from the
question, as:

\[
M=\langle m_1,m_2,\cdots,m_k\rangle,\quad\begin{cases}m_1,m_n\in A \\ \forall 1<i<k(m_i\in A^{(1)})\end{cases}
\]

Because repitition is allowed for every symbol in the message, we can
simply multiply the size of each of the sets of possible symbols for
every symbol in the message:

\[
n^2\prod\limits_{i=2}^{k-1} m=n^2m^{k-2}
\] \[
\subsolution{b}{40^2 30^{23}}
\]

%
\problem{1.1}{20}

The alphabet of letters we can use are:

\[
A=\langle
\text{D}_1,
\text{A}_1,
\text{T}_1,
\text{A}_2,
\text{G}_1,
\text{R}_1,
\text{A}_3,
\text{M}_1
\rangle
\]

We can then define an analogous count of each of the $n=8$ letters:

\[
N=\langle n_d,n_a,n_t,n_g,n_r,n_m\rangle=\langle 1,3,1,1,1,1\rangle
\]

%
\subproblem{a} The number of different arrangements can be defined as:

\[
\subsolution{a}{\dfrac{\left(\sum\limits_i n_i\right)!}{\prod\limits_i n_i!}=\dfrac{8!}{3!}}
\]

%
\subproblem{b} We can go about this problem by looking at the set of
all permutations of:

\[
X=\langle{}\text{A}_1,\text{A}_2,\text{A}_3\rangle
\]

and all permutations of:

\[
Y=\langle \text{D}_1,\text{T}_1,\text{G}_1,\text{R}_1,\text{M}_1\rangle
\]

And then we can look at all $k$, where $1\le{}k\le{}n-2$, and the
arrangements:

\[
M^{(k)}=\langle y_1,y_2,\cdots,y_{k-1},x_1,x_2,x_3,y_{k+3},\cdots,y_n\rangle
\]

Then we can simply multiply together the number of arrangements of
$X$, $Y$, and the values for $k$.

\[
\subsolution{b}{P(n-3,n-3)P(3,3)(n-2)=5!\cdot3!\cdot6}
\]

%
\problem{1.1}{30}

Let the alphabet be:

\[
A=\{a_1,a_2,\cdots,a_{26}\}
\]

%
\subproblem{a} If any letter can be repeated, then we have a message
defined as:

\[
M^{(x,y,z)}=\langle x,y,z,y,x \rangle,\quad x,y,z\in A
\]

This means we can simplify the problem to the number of ways we can
choose a permutation $(x,y,z)$ from $A$, which is just
$26^3$. Similarly, for six letters, we have:

\[
M^{(x,y,z)}=\langle x,y,z,z,y,x \rangle,\quad x,y,z\in A
\]

Which again is just $26^3$.

\[
\subsolution{a}{26^3}
\]

%
\subproblem{b} For this part, we add another condition to
$M^{(x,y,z)}$:

\[
x\ne y,\quad y\ne z,\quad x\ne z
\]

In other words, we're now looking at a permutation of size 3 out of
the alphabet. Likewise, for the six letter case, it also comes down to
$P(26,3)$.

\[
\subsolution{b}{P(26,3)=\frac{26!}{23!}}
\]

%
\problem{1.1}{34}



%
\problem{1.1}{38}




\end{multicols*}
\end{document}
