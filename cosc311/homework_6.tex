\documentclass{article}
\usepackage[margin=0.75in]{geometry}
\usepackage{amsmath}
\usepackage{amssymb}
\usepackage{booktabs}
\usepackage{xintexpr}
\usepackage{multicol}

\setlength{\columnsep}{0.5cm}
\setlength{\columnseprule}{1pt}

\newcommand{\T}{1}
\newcommand{\F}{0}
\newcommand{\TF}[1]{\if1#1\T\else\F\fi}
\newcommand{\xintTF}[1]{\xintifboolexpr{#1}{\T}{\F}}

\newcommand{\logicrule}[2]{
\begin{array}{l}
#1 \\
\midrule
\therefore #2 \\
\end{array}
}

\newcommand{\problem}[2]{$\boxed{\text{#1 \##2}}$}
\newcommand{\subproblem}[1]{$\boxed{\text{(#1)}}$}
\newcommand{\subsolution}[2]{\boxed{#2\quad(\text{#1})}}
\newcommand{\RUS}{Rule of Universal Specification}
\newcommand{\RUG}{Rule of Universal Generalization}
\newcommand{\ROP}{Rule of Product}
\newcommand{\ROS}{Rule of Sums}

\setlength\parindent{0pt}
\setlength\parskip{0em}

\begin{document}
\begin{multicols*}{2}

%
\problem{1.1}{4}

There are $n=10$ board members, denoted:

\[
A=\langle{}a_1,a_2,\cdots,a_n\rangle
\]

%
\subproblem{a} We are looking to fill $r=4$ distinct positions:

\[
B=\langle{}b_1,b_2,\cdots,b_r\rangle, \quad b_i\in\{1,2,\cdots,n\}
\]

Where $b_1$ is the president's position, $b_2$ is the vice president,
$b_3$ is the secretary, and $b_4$ is the treasurer. Because each
position is distinct, then the order of each $b_i$ matters, so we are
looking at permutations. So we can directly calculate $P(n,r)$:

\[
P(n,r)=\frac{n!}{(n-r)!}
\] \[
P(10, 4)=\frac{10!}{6!}
\] \[
\subsolution{a}{P(10, 4)=5040}
\]

%
\subproblem{b i} Suppose that $a_1$, $a_2$, and $a_3$ are the $k=3$
members which are physicians. The event that a physician is nominated
for president can be written as:

\[
b_1\le k
\]

This means that we have 3 choices for the first position, 9 for the
second, 8 for the third, and 7 for the last. By the \ROP, this means
we have $kP(n-1,r-1)$ slates:

\[
\subsolution{b i}{3P(9,3)=1512}
\]

%
\subproblem{b ii} The event that exactly one physician is nominated
for the board of directors can be summarized as:

\[
1=\sum\limits_i \lbrack b_i\le k\rbrack
\]

Suppose we choose one of the 4 $b_i$ positions to be the sole
physician and then choose the remaining ones from the non-physician
pool. This means there are 3 options for the first position, 6 for the
second, 5 for the third, and 4 for the last. This means there are
$kP(n-k-1,r-1)$ slates that match this event:

\[
\subsolution{b ii}{3P(6,3)=360}
\]

%
\subproblem{b iii} The event that at least one physician is on the
board of directors can be summarized by:

\[
1\le \sum\limits_i \lbrack b_i\le k\rbrack
\]

Suppose we choose one position to be filled from the pool of
physicians, and then choose the remaining positions from all the other
members. Then the first position has 3 options, the second has 9
options, the third has 8 options, and the fourth has 7 options. This
means there are $kP(n-1,r-1)$ slates that satisfy the event:

\[
\subsolution{b iii}{3P(9,3)=1512}
\]

%
\problem{1.1}{6}

The runners can be written as:

\[
X=\langle x_1,x_2,\cdots,x_n\rangle
\]

The trophies as:

\[
T=\langle t_1,t_2,\cdots,t_r\rangle,\quad t_i\in\{1,2,\cdots,n\}
\]

Where $\forall{}i\forall{}j((i\ne{}j)\rightarrow(t_i\ne{}t_j))$.

%
\subproblem{a} Suppose $n=30$ and $r=8$. Because $T$ is a tuple, then
order matters, so we are looking at a permutation. This means there
are $P(n,r)$ ways the trophies can be handed out:

\[
\subsolution{a}{P(30,8)=235,989,936,000}
\]

%
\subproblem{b} In this problem, we are looking for the event that $m$
runners from $X$ are in one of the first $k$ spots of $T$. Assuming,
without loss of generality, that the $m$ runners are the first runners
in $X$, then this can be written as:

\[
2=\sum\limits_{i=1}^k \lbrack T_i\le m\rbrack
\]

In other words, $m$ of the racers in the top $k$ spots are the runners
we're interested in, and $k-m$ are the $n-m$ remaining racers. Let
$n=30$, $k=3$, and $m=2$. In this circumstance, the order of the
racers does not matter. Suppose we choose the position of the
non-interesting racer first (which means we have $k=3$ possibilities)
and the non-interesting racer himself (which means we have $n-m=28$
possibilities). Then we choose which of the interesting racers comes
before the other (so that's another $m=2$ possibilities), and then the
last interesting racer (which only has $1$ remaining possibility).

In summary, we have $3\cdot28\cdot2\cdot1$ different ways this can be
arranged, or:

\[
\prod\limits_{i=1}^2\binom{n-m}{1}
\]

%
\problem{1.1}{10}



%
\problem{1.1}{16}



%
\problem{1.1}{20}



%
\problem{1.1}{30}



%
\problem{1.1}{34}



%
\problem{1.1}{38}




\end{multicols*}
\end{document}
