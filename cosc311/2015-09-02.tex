\documentclass{article}
\usepackage{amsmath}
\usepackage{amssymb}

\newcommand{\T}{1}
\newcommand{\F}{0}
\newcommand{\TF}[1]{\if1#1\T\else\F\fi}
\newcommand{\xintTF}[1]{\xintifboolexpr{#1}{\T}{\F}}

\newcommand{\logicrule}[2]{
\begin{array}{c}
#1 \\
\midrule
\therefore #2 \\
\end{array}
}

\setlength\parindent{0pt}
\setlength\parskip{1em}

\begin{document}

\section*{Last class}

Last class we discussed the idea of open statements.

\section*{Example 9.02.1}

Using the following open statements:

\[
\begin{array}{rcl}
p(x,y) & : & x+y=7 \\
q(x,y) & : & x>y \\
r(x,y) & : & p(x,y)\rightarrow q(x,y) \\
\end{array}
\]

\subsection*{Example 9.02.1 (a)}

Is $p(3,4)$ true or false? $3+4=7$ is true, therefore $p(3,4)$ is
true.

Is $q(3,4)$ true or false? $3>4$ is false, therefore $q(3,4)$ is
false.

Then $r(3,4):p(3,4)\rightarrow{}q(3,4)$ has a truth value of false. Thus, the argument is invalid.

\subsection*{Example 9.02.1 (b)}

Is $p(4,3)$ true or false? True.

Is $q(4,3)$ true or false? True.

In this one case, $r(4,3):p(4,3)\rightarrow{}q(4,3)$ has a truth value
of true. \textbf{But} recall: that doesn't imply the argument is
valid. Part A gave us a counter example!

\subsection*{Example 9.02.1 (conclusion)}

Through this, based on part A and part B, we have the following true
statements:

``For some $(x,y)$, $r(x,y)$''

``For some $(x,y)$, $\neg{}r(x,y)$''

The ``for some'' part quantifies the open statement, $r(x,y)$, and
allows a truth value to be determined.

\section*{Example 9.02.2}

Using the universe of the real numbers: Let $s_1$ be the open statement:

\[
s_1 : x>0
\]

Where $x$ is a ``free variable''. Note that we cannot determine the
truth value of $s_1$.

Now consider the statement $s_2$: ``There exists an $x$ such that
$x>0$''.

Even though this statement still contains the variable $x$, the use of
``there exists'' quantifies the expression and allows us to assign a
truth value to it. Now $x$ is a ``bound variable''.

$s_2$ has a definite truth value, given a universe: true, because the
value of $x=1$ is a real number, and $1>0$. So, we found a real number
that meets the criteria of $x>0$, so at least one exists.

\section*{Existential Quantifier}

Some \textbf{existential quantifiers} are: ``there exists'', ``for
some'', ``for at least one''.

The notation used is $\exists$, for example:

\[
s_2: \exists x (x>0)
\]

which reads as ``$s_2$ is a statement that there exists an $x$ such
that $x>0$.''

\section*{Universal Quantifier}

Now suppose we have a statement $s_3$: ``For all $x$, $x>0$''.

Is $s_3$ true or false in our universe? False because the real numbers
(our universe here) include $0$ and negative numbers, all of which are
less than zero, not greater than.

For a statement $s_4$: ``For all $x$, $x+1>x$''.

Is $s_4$ true or false in our universe? True.

Some \textbf{universal quantifiers} are: ``for all'', ``for any'',
``for every''.

The notation used is: $\forall$.

The previous statements can be written:

\[
\begin{array}{rcl}
s_3 & : & \forall x (x>0) \\
s_4 & : & \forall x (x+1>x) \\
\end{array}
\]

\section*{Example 9.02.3}

Consider the universe of positive integers:

\[
\mathbb{Z}^+ = \{1,2,3,4,\dots\}
\]

And now for $\mathbb{Z}^+$, is $s_3:\forall{}x(x>0)$ true or false? True.

\textbf{Key idea}: The truth value of a quantified statement may be
dependant on the universe of discourse.

\section*{Logical Equivalence and Logically Infers}

\textit{Definition 2.6 (page 92)}: ``Logical Equivalence'' and
``Logically Infers'' definitions for open statements, $p(x)$, and
$q(x)$:

We have that the definition of logical equivalence (which uses
$\Leftrightarrow$):

\[
p(x)\Leftrightarrow q(x)\text{ means } \forall x(p(x)\leftrightarrow q(x))
\]

In other words, the bicondition is true for every value of $x$ in the
universe.

Similarly, we have that the definition of logically infers (which uses
$\Rightarrow$).

\[
p(x)\Rightarrow q(x)\text{ means }\forall x(p(x)\rightarrow q(x))
\]

In other words, the implication is true for every value of $x$ in the universe.

\textbf{Note}: We can also expand these to 2 or more variables.

\section*{Contrapositive, Converse, and Inverse}

\textit{Definition 2.7 (page 92)}: Contrapositive, Converse, and
Inverse of universally quantified statements:
$forall{x}\lbrack{}p(x)\rightarrow{}q(x)\rbrack$.

For example, the statement: $\forall{}x\left((\text{$x$ is even})\rightarrow(\text{$31x+12$ is even})\right)$,
we have the the following statements:

\begin{itemize}
\item Contrapositive: $\forall x\left((\text{$31x+12$ is not even})\rightarrow (\text{$x$ is not even})\right)$
\item Converse: $\forall x\left((\text{$31x+12$ is even})\rightarrow (\text{$x$ is even})\right)$
\item Inverse: $\forall{}x\left((\text{$x$ is not even})\rightarrow(\text{$31x+12$ is not even})\right)$
\end{itemize}

What if we had general open statements $p(x)$ and $q(x)$?

We would see that the following statement is true:

\[
\forall x (p(x)) \Rightarrow \exists x (p(x))
\]

If, for all $x$, the statement $p(x)$ is true, then we know that there
exists at least one $x$ such that $p(x)$ is true. \textbf{Note}: This
only goes in one direction. We can't go from ``there exists one'' to
``for all''.

We can also look at the following logical equivalence:

\[
\exists x \lbrack p(x)\vee q(x)\rbrack \Leftrightarrow \lbrack \exists x(p(x))\vee \exists x(q(x))\rbrack
\]

However, the following is only a logical implication:

\[
\exists x\lbrack p(x)\wedge q(x)\rbrack \Rightarrow \lbrack \exists x(p(x))\wedge \exists x(q(x))\rbrack
\]

\textbf{Note}: This is only a logical equivalence. Think about it this
way: The statement on the left is about one particular $x$ that
satisfies both $p(x)$ and $q(x)$, however the one on the right means
that there's one $x$ that satisfies $p(x)$, but it could be a
different $x$ that satisfies $q(x)$. They need not be the same person,
so we can't infer the left side from the right side.

Now the following logical equivalence:

\[
\forall x\lbrack p(x)\wedge q(x)\rbrack \Leftrightarrow\lbrack \forall x(p(x))\wedge \forall x(q(x))\rbrack
\]

However, the following is only a logical inference:

\[
\lbrack\forall x(p(x))\vee \forall x(q(x))\rbrack \Rightarrow \forall x\lbrack p(x)\vee q(x)\rbrack
\]

We can think about this in the following way: the left hand side is
saying that either ``for every $x$, $p(x)$ is true'' or ``for every
$x$, $q(x)$ is true''. The right hand side, on the other hand, says
that ``for every $x$, $p(x)$ or $q(x)$ is true.'' In other words, the
left is ``every student in the class is wearing flip flops'' or
``every student is wearing shoes'', whereas the right side is ``every
student is wearing flip flops or shoes''.

\section*{Example 2.4 (3)}

The following logical equivalences show that we can manipulate the
inside of a ``for all'' statement.

\[
\forall x (\neg\neg p(x)) \Leftrightarrow \forall x(p(x))
\]

\[
\forall x(\neg\lbrack p(x)\wedge q(x)\rbrack) \Leftrightarrow \forall x(\lbrack\neg p(x)\vee \neg q(x)\rbrack)
\]

From Table 2.23, we can see that you can also negate quantifiers:

\[
\begin{array}{rcl}
\neg\lbrack\forall x(p(x))\rbrack & \Leftrightarrow & \exists x(\neg p(x)) \\
\neg\lbrack\exists x(p(x))\rbrack & \Leftrightarrow & \forall x(\neg p(x)) \\
\neg\lbrack\exists x(\neg p(x))\rbrack & \Leftrightarrow & \exists x(\neg\neg p(x)) \\
                                       & \Leftrightarrow & \exists x(p(x)) \\
\end{array}
\]


\end{document}
