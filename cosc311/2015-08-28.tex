\documentclass{article}
\usepackage{amsmath}
\usepackage{amssymb}
\usepackage{pgfplots}
\usepackage{booktabs}
\usepackage{xintexpr}

\setlength\parindent{0pt}
\setlength\parskip{1em}
\pgfkeys{/pgfplots/MyAxisStyle/.style={xmin=-10,xmax=10, ymin=-10,ymax=10,height=6cm,width=6cm}}

\begin{document}

\section*{Recall}

Last class we showed that

\[
\left(p\wedge\left(p\rightarrow q\right)\right)\rightarrow q
\]

was a $T_0$ which means that the argument is valid.

This result is for arbitrary primitive statements, $p$ and
$q$. \textbf{But} the rules of substitution allow us to extend this
result to \textbf{any} statements $p$ and $q$ (primitive or
not). Throught his process, we just established the first of the
\textit{``Rules of Inference''} --- a general, fundamental argument
that is valid (i.e. implications that are $T_0$).

The notation for one of these rules is:

\[
\begin{array}{c}
p \\
p\rightarrow q \\
\midrule
\therefore q \\
\end{array}
\Leftrightarrow
\left\lbrack p\wedge (p\rightarrow q) \right\rbrack\rightarrow q
\]

Or, in English, ``If $p$ and $p\rightarrow{}q$, then $q$''.

\section*{Definition 2.4}

Let $p$, $q$ be arbitrary statements such that $p\rightarrow{}q$ is a
$T_0$. In this case, we say that $p$ \textbf{logically implies}
$q$. ``$p\Rightarrow{}q$''.

So when $p\Rightarrow{}q$, then $p\rightarrow{}q$ is a tautology and
we say $p\rightarrow{}q$ is \textbf{logical implication}.

So, if we said $p\leftrightarrow{}q$ is a $T_0$, then
$p\Leftrightarrow{}q$ (or $p\Rightarrow{}q$ and $q\Rightarrow{}p$).

\section*{The Rules of Inference}

On some page, there is a big list of these arguments.

Consider these as ``mini valid arguments'' upon which larger arguments
are shown to be valid.

\subsection*{Exaple 8.28.1 (page 85 \#106)}

Establish the validity of the following argument:

\[
\left\lbrack \underset{3\text{ premises}}{p\wedge\left(p\rightarrow q\right)\wedge (\neg q\vee r)}\right\rbrack \rightarrow \underset{conclusion}{r}
\]

We could have generated a truth table, but sometimes it is infeasible
to generate an entire truth table when the statements become
complex. Instead, we use the Rules of Inference and write a column
proof. Note: Column proofs technically do not prove something is a
valid argument, but it suffices to show the general idea of proving.

\begin{tabular}{c|c|c}
  & Steps & Reaons \\
\hline
1 & $p\wedge(p\rightarrow{}q)$ & Given premise \\
2 & $q$ & Step 1 and R.I. \#1 \\
3 & $\neg{}q\vee{}r$ & Given premise \\
4 & $q\rightarrow{}r$ & Step 3 and conditional law \\
5 & $r$ & Step 2 and Step 4 and R.I \#1 \\
\end{tabular}

So, the statement is a valid argument (i.e. the implication is a
$T_0$).

Another way (but perhaps more difficult/longer) is to use a truth
table to show validity.

\subsection*{Some invalid arguments}

Some invalid arguments are discussed on pages 74 and 75. One example is

\[
\begin{array}{c}
p\rightarrow q \\
q \\
\midrule
\therefore p \\
\end{array}
\]

Also

\[
\begin{array}{c}
p\rightarrow q \\
\neg p \\
\midrule
\therefore \neg q \\
\end{array}
\]

\subsection*{How do we show that thee are not valid?}

\begin{enumerate}
\item Show each implication is not a tautology or look at truth table
  rows where all premises true.
\item Find one counter example in which true premises lead to a false
  conclusion.
\end{enumerate}

We'll look at the second option, because it is not always easy to fill
out an entire truth table.

We start by assigning truth values to $p$ and $q$ that make the
premises true but the conclusion false.

For
$\begin{array}{c}p\rightarrow{}q\\q\\\midrule\therefore{}p\\\end{array}$,
  let $p: 0$ and $:1$ then the premise $p\rightarrow{}q:1$ and $q:1$,
  so $(p\rightarrow{}q)\wedge{}q:1$, but $p:0$ was the conclusion. So,
  \textbf{Not a valid argument!}.


\end{document}
