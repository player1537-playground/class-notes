\documentclass{article}
%\usepackage[margin=0.75in]{geometry}
\usepackage{amsmath}
\usepackage{amssymb}
\usepackage{booktabs}
\usepackage{xintexpr}
\usepackage{multicol}
\usepackage{forest}
\usepackage{xfrac}

\setlength{\columnsep}{0.5cm}
\setlength{\columnseprule}{1pt}

\newcommand{\T}{\text{True}}
\newcommand{\F}{\text{False}}
\newcommand{\TT}{1}
\newcommand{\FF}{0}
\newcommand{\TF}[1]{\if1#1\TT\else\FF\fi}
\newcommand{\xintTF}[1]{\xintifboolexpr{#1}{\TT}{\FF}}

\newcommand{\logicrule}[2]{
\begin{array}{l}
#1 \\
\midrule
\therefore #2 \\
\end{array}
}

\newcommand{\problem}[2]{$\boxed{\text{#1 \##2}}$}
\newcommand{\subproblem}[1]{$\boxed{\text{(#1)}}$}
\newcommand{\subsolution}[2]{\boxed{#2\quad(\text{#1})}}
\newcommand{\solution}[1]{\boxed{#1}}
\newcommand{\RUS}{Rule of Universal Specification}
\newcommand{\RUG}{Rule of Universal Generalization}
\newcommand{\ROP}{Rule of Product}
\newcommand{\ROS}{Rule of Sums}

\newcommand{\multistep}[1]{\begin{array}{rl} #1 \end{array}}
\newcommand{\subeq}{\subseteq}
\newcommand{\sub}{\subset}

\newcommand{\conj}[1]{\overline{#1}}

\setlength\parindent{0pt}
\setlength\parskip{0em}

\begin{document}

%
\problem{5.1}{2}

Let $A=\{1,2,3\}$ and $B=\{2,4,5\}$.

%
\subproblem{a} We want to find three non-empty relations from $A$ to
$B$, which are defined by:

\[
R\subseteq (A\times B)
\]

where

\[
A\times B=\{\langle a,b\rangle|a\in A,b\in B\}
\]

So three relations are:

\[
\subsolution{a}{
\multistep{
R_1&=\{\langle 1,2\rangle, \langle 1,4\rangle, \langle 2,4\rangle\} \\
R_2&=\{\langle 1,5\rangle, \langle 3,5\rangle\} \\
R_3&=\{\langle 2,2\rangle, \langle 3,5\rangle,\langle 3,2\rangle\} \\
}
}
\]

%
\subproblem{b} Now we want to find three non-empty relations on $A$,
defined as:

\[
R\subseteq (A\times A)
\]

Three such relations are:

\[
\subsolution{b}{\multistep{
R_1&=\{\langle 1,1\rangle, \langle 2,1\rangle, \langle 3,3\rangle\} \\
R_2&=\{\langle 2,2\rangle\} \\
R_3&=\{\langle 2,3\rangle, \langle 3,2\rangle\} \\
}}
\]

%
\problem{5.1}{10}

We can model each phone call as a function. For example, if each of
the people in the set $A_1$ calls one person in $A_2$, then we can
model that as:

\[
R_{1,2}\subseteq(A_1\times A_2)
\]

Let $A_1$ be the set containing the originator ($|A_1|=1$), and $A_2$
be the set containing the people the originator calls. Define a
relation $R_{1,2}$ between $A_1$ and $A_2$, where the size of the
relation is the number of phone calls made per person in $A_1$
multiplied by the number of people in $A_1$. In other words:

\[
R_{1,2}\subseteq(A_1\times A_2)
\] \[
|R_{1,2}|=2|A_1|
\]

Because the relation $R_{1,2}$ is made up of tuples $(x,y)$ where
$x\in{}A_1$ and $y\in{}A_2$, and no person is called twice (therefore
if $(x,y)\in{}R_{1,2}$ and $(z,y)\in{}R_{1,2}$ then $x=z$), then the
size of $A_2$ must be the size of the relation $R_{1,2}$:

\[
\multistep{
|A_2|&=|R_{1,2}| \\
&= 2|A_1| \\
&= 2 \\
}
\]

By the same reasoning, we can create $A_3$, the set of people called
by people in $A_2$, and $R_{2,3}$, the relation mapping people in
$A_2$ to people in $A_3$, and conclude that $|A_3|$ must be:

\[
\multistep{
|A_3|&=|R_{2,3}| \\
&=3|A_2| \\
&= 6 \\
}
\]

Finally, we define $A_4$ and $R_{3,4}$ the same way, to find that
$|A_4|$ must be:

\[
\multistep{
|A_4|&=|R_{3,4}| \\
&=5|A_3| \\
&=30 \\
}
\]

%
\subproblem{a} The total number of people who now know the rumor is
the sum of the sizes of each set $A_1$, $A_2$, $A_3$, and $A_4$:

\[
|A_1|+|A_2|+|A_3|+|A_4|=1+2+6+30
\] \[
\subsolution{a}{|A_1|+|A_2|+|A_3|+|A_4|=39}
\]

%
\subproblem{b} The total number of phone calls made is the sum of the
sizes of each relation:

\[
|R_{1,2}|+|R_{2,3}|+|R_{3,4}|=2+6+30
\] \[
\subsolution{b}{|R_{1,2}|+|R_{2,3}|+|R_{3,4}|=38}
\]

%
\problem{5.1}{12}

Let $A$ and $B$ be sets with $|B|=3$. Suppose that there are $N=4096$
total possible relations from $A$ to $B$. We want to find the size of
$A$.

To do this, we rely on the definition for the number of possible
relations between two arbitrary sets, $X$ and $Y$, where $|X|=n$ and
$|Y|=m$. The definition states the number of possible relations is:

\[
S=\{ R | \text{$R$ is a relation from $X$ to $Y$} \}
\] \[
|S|=2^{mn}
\]

Applying this to the sets $A$ and $B$, we see that it works out to be:

\[
N=2^{|A|\cdot|B|}
\] \[
4096=2^{3|A|}
\]

We can solve this to find that $|A|$ must be:

\[
\solution{|A|=4}
\]

%
\problem{5.2}{4}

Let $A$ and $B$ be sets, with $|B|=3$. Suppose there are $N=2187$
functions $f:A\rightarrow{}B$. We want to find $A$. We can do this
using the definition for the number of functions from $A$ to $B$,
which states that:

\[
S=\{ f|f:A\rightarrow B \}
\] \[
|S|=|B|^{|A|}
\]

We can see that this involves solving the equation:

\[
2187=3^{|A|}
\]

Which yields:

\[
\solution{|A|=7}
\]

%
\problem{5.2}{6}

Let $A$, $B$, and $C$ be functions mapping reals to reals, where:

\[
\multistep{
A&=\{(x,y)|y=2x+1\} \\
B&=\{(x,y)|y=3x\} \\
C&=\{(x,y)|x-y=7\} \\
}
\]

%
\subproblem{a i} We want to find $A\cap{}B$. This involves finding the
set:

\[
\multistep{
A\cap B&=\{z|z\in A, z\in B\} \\
&=\{(x,y)|y=2x+1, y=3x\} \\
&=\{(x,y)|0=x-1, y=3x\} \\
&=\{(x,y)|x=1, y=3x\} \\
&=\{(1,3)\} \\
}
\]

Therefore:

\[
\subsolution{a i}{A\cap B=\{(1,3)\}}
\]

%
\subproblem{a ii} We want to find $B\cap{}C$:

\[
\multistep{
B\cap C&=\{(x,y)|(x,y)\in B, (x,y)\in C\} \\
&=\{(x,y)|y=3x, x-y=7\} \\
}
\]

\[
\subsolution{a ii}{B\cap C=\{(\sfrac{-7}{2}, \sfrac{-21}{2})\}}
\]

%
\subproblem{a iii} We want to find $\conj{\conj{A}\cup\conj{C}}$. By
applying DeMorgan's Law, we can see that this is the same as
$A\cap{}C$.

\[
\multistep{
A\cap C&=\{(x,y)|(x,y)\in A, (x,y)\in C\} \\
&=\{(x,y)|y=2x+1, x-y=7\} \\
&=\{(x,y)|x=-8, y=-15\} \\
}
\]

\[
\solution{A\cap C=\{(-8,-15)\}}
\]

%
\subproblem{a iv} We want to find $\conj{B}\cup\conj{C}$. It will be
easier to instead look at an alternative representation of this set
using DeMorgan's Law: $\conj{B\cap{}C}$. We previously found
$B\cap{}C$, which is:

\[
B\cap C=\{(-7/2,-21/2)\}
\]

The conjugate of this set, then, is the set:

\[
\subsolution{a iv}{\conj{B\cap C}=\mathbb{R}^2 - \{(-7/2, -21/2)\}}
\]

%
\subproblem{b i} If we change our universe to
$\mathbb{Z}^+\times\mathbb{Z}^+$, then (a i) does not change because
its solution already lied in this universe.

\[
\solution{A\cap B=\{(1,3)\}}
\]

%
\subproblem{b ii} However, because $\sfrac{-7}{2}\not\in\mathbb{Z}^+$
(and $\sfrac{-21}{2}\not\in\mathbb{Z}^+$), then the set $B\cap{}C$
becomes the empty set.

\[
\solution{B\cap C=\varnothing}
\]

%
\subproblem{b iii} Similarly, $-8\not\in\mathbb{Z}^+$, so:

\[
\solution{A\cap C=\varnothing}
\]

%
\subproblem{b iv} Lastly, because the set we remove from our universe
($\{(\sfrac{-7}{2},\sfrac{-21}{2})\}$) does not lie in our universe
now, then this instead becomes:

\[
\solution{\conj{B\cap C}=\mathbb{Z}^+\times\mathbb{Z}^+}
\]

%
\problem{5.2}{8a} We want to determine with the statement $S$ is true
or false, where:

\[
S:\lfloor a\rfloor =\lceil a\rceil\quad\text{for all $a\in\mathbb{Z}$}
\]

We recall the definitions of floor and ceiling:

\[
\lfloor x\rfloor \overset{def}{=}\text{max}\{z\in\mathbb{Z}|z\le x\}
\] \[
\lceil x\rceil \overset{def}{=}\text{min}\{z\in\mathbb{Z}|z\ge x\}
\]

We note that, if $x\in\mathbb{Z}$, then $\lfloor{}x\rfloor$ and
$\lceil{}x\rceil$ are both just $x$. For instance, with the floor of
$x$, then if we let $z=x$ then we have found an integer
($z\in\mathbb{Z}$) that is less than or equal to $x$ ($z\le{}x$). If
we had a hypothetical $z'=x+1$, then this would also be an integer
($z'\in\mathbb{Z}$) but it would not be less than or equal to $x$
($z\not\le{}x$). Therefore, for integers $\lfloor{}x\rfloor=x$, and
similarly with ceilings.

Likewise, because our statement $S$ deals only with integers
$a\in\mathbb{Z}$, then the floor and ceiling of $x$ must be equal, by
the above reasoning, so the statement is true.

%
\problem{5.2}{16}

Let $f:\mathbb{R}\rightarrow\mathbb{R}$ where $f(x)=x^2$. We want to
find the image of $f$ onto $A$, $f(A)$, for particular subsets
$A\subseteq\mathbb{R}$.

%
\subproblem{a} Let $A=\{2,3\}$. The image $f(A)$ is defined as:

\[
\multistep{
f(A)&=\{b|a\in A, (a,b)\in f\} \\
&=\{b|a\in A, b=a^2\} \\
&=\{2^2, 3^2\}
}
\]

\[
\subsolution{a}{f(A)=\{4,9\}}
\]

%
\subproblem{c} Let $A=\{x\in\mathbb{R}|-3<x<3\}$. The image $f(A)$ is:

\[
\multistep{
f(A)&=\{b|a\in A, (a,b)\in f\} \\
&=\{b|a\in A, b=a^2\} \\
&=\{a^2|-3<a<3\} \\
}
\]

We can split our domain, $-3<a<3$, into two. A useful place to split
is at $a=0$, because this is where the sign of $a$ changes.

\[
\multistep{
f(A)&=\{a^2|-3<a\le 0\}\cup\{a^2|0<a<3\} \\
&=\{x|(-3)^2<x\le (0)^2\}\cup\{x|(0)^2<x<(3)^2\} \\
&=\{x|9<x\le 0\}\cup\{x|0<x<9\} \\
&=\varnothing\cup\{x|0<x<9\} \\
&=\{x|0<x<9\} \\
}
\]

\[
\subsolution{c}{f(A)=\{x\in\mathbb{R}|0<x<9\}}
\]

%
\subproblem{f}

%
\problem{5.2}{20}

\end{document}
