\documentclass{article}
\usepackage{amsmath}
\usepackage{amssymb}
\usepackage{booktabs}
\usepackage{xintexpr}

\newcommand{\T}{1}
\newcommand{\F}{0}
\newcommand{\TF}[1]{\if1#1\T\else\F\fi}
\newcommand{\xintTF}[1]{\xintifboolexpr{#1}{\T}{\F}}

\newcommand{\logicrule}[2]{
\begin{array}{l}
#1 \\
\midrule
\therefore #2 \\
\end{array}
}

\newcommand{\inv}[1]{#1^{-1}}

\renewcommand{\d}[1]{\,\textnormal{d}#1}
\newcommand{\dd}[2]{\frac{\d{#1}}{\d{#2}}}
\newcommand{\ddd}[2]{\dfrac{\d{#1}}{\d{#2}}}

\setlength\parindent{0pt}
\setlength\parskip{1em}

\begin{document}

\section*{Section 1.2 - \# 38}

A committee of 15 -- 9 women and 6 men -- seated at a circular table;
seat 6 men so no 2 men are sitting next to each other.

We can first try just fixing 1 woman in a spot:

\[
W=\langle w_1,w_2,\cdots,w_9\rangle
\]

We can then arrange the remaining 8 women in the 8 spots. This is
equivalent to $8!$ ways.

There are 9 spaces between the women. So of these 9 spaces, we know
that for any 6 of these, there are $P(9,6)$ permutations of the 6
spaces. But since it doesn't matter what order we select the spaces
(to put the men), we really have $\frac{P(9,6)}{6!}$ ways to select 6
of the 9 spaces.

For each of these ways to select the spaces for the men, there are
$6!$ ways to arrange the 6 men in the 6 spots. In total, there are:

\[
8!\cdot\frac{P(9,6)}{6!}6!=\frac{8!9!}{3!}
\]

\end{document}
