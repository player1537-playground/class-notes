\documentclass{article}
\usepackage[margin=0.75in]{geometry}
\usepackage{amsmath}
\usepackage{amssymb}
\usepackage{booktabs}
\usepackage{xintexpr}
\usepackage{multicol}

\setlength{\columnsep}{0.5cm}
\setlength{\columnseprule}{1pt}

\newcommand{\T}{1}
\newcommand{\F}{0}
\newcommand{\TF}[1]{\if1#1\T\else\F\fi}
\newcommand{\xintTF}[1]{\xintifboolexpr{#1}{\T}{\F}}

\newcommand{\logicrule}[2]{
\begin{array}{l}
#1 \\
\midrule
\therefore #2 \\
\end{array}
}

\newcommand{\problem}[2]{$\boxed{\text{#1 \##2}}$}
\newcommand{\subproblem}[1]{$\boxed{(#1)}$}
\newcommand{\RUS}{Rule of Universal Specification}
\newcommand{\RUG}{Rule of Universal Generalization}

\setlength\parindent{0pt}
\setlength\parskip{0em}

\begin{document}
\begin{multicols*}{2}

\problem{5}{1}

\subproblem{a} The problem lies in the first and second sentences. The
proof writer wanted to write the negation of the conclusion, but wrote
a different statement entirely. The problem statement is, in the
universe of the real numbers:

\[
\forall x\forall y((x+y=10)\rightarrow((x\ne 3)\wedge(y\ne 8)))
\]

The conclusion in particular is:

\[
(x\ne 3)\wedge(y\ne 8)
\]

If we negate that, we get:

\[
\neg((x\ne 3)\wedge(y\ne 8))
\] \[
\neg(x\ne 3)\vee\neg(y\ne 8)
\] \[
(x=3)\vee(y=8)
\]

We can see that the proof writer forgot to swap the $\wedge$ for a
$\vee$ when applying DeMorgan's Law.

\subproblem{b} Let $x=3$ and $y=7$. Therefore $x+y=10$, so the premise
is true. But the conclusion $(3\ne 3)\wedge(7\ne 8)$ is false, because
$3=3$.

\problem{5}{2}

Let the universe of discourse be the integers. We want to show that
$15|n$ iff $3|n$ and $5|n$. Written mathematically, we have:

\[
\forall n((15|n)\leftrightarrow((3|n)\wedge(5|n)))
\]

In order to prove this statement, we want to prove two statements:

\[\tag{a}
\forall n((15|n)\rightarrow((3|n)\wedge(5|n)))
\] \[\tag{b}
\forall n(((3|n)\wedge(5|n))\rightarrow(15|n))
\]

\subproblem{a} Let $n$ be an arbitrary integer. Assume $15|n$,
i.e. that $\exists{}a(n=15a)$. We can factor $15$ to get
$15=3\cdot{}5$. We can acknowledge the implication of this equivalence
with the expression $3|15$, or equivalently $\exists{}b(15=3b)$. By
substituting this expression for $15$ in the known value for $n$, we
get:

\[
n=(3b)a
\]

By using the associativity of the integers, we can rewrite this as:

\[
n=3(ba)
\]

If we let $c=ba$, which is an integer because the integers are closed
under multiplication, then we can rewrite this as $n=3c$. By the
definition of ``divides'', we can conclude that $3|n$.

We can apply the same steps as above, replacing $5$ for $3$, and
conclude that $5|n$. Because we have both $3|n$ and $5|n$, we have
$(3|n)\wedge(5|n)$, which is our conclusion. We have shown that this
statement is true for arbitrary $n$, and by the \RUG, we can conclude
that for any integer $n$, if $15|n$ then $3|n$ and $5|n$.

\subproblem{b} Let $n$ be an arbitrary integer. Assume $3|n$ and
$5|n$, or in other words, that $\exists{}b(n=3b)$ and
$\exists{}c(n=5c)$. We can set these two statements equal to each
other, to get:

\[
3b=5c
\]

If we solve for $b$, we get:

\[
b=\frac{5}{3}c
\]

Because we are working in the integers and we are working with an
integer $b$, the only way this expression can evaluate to an integer
is if $c$ is some multiple of $3$, or in other words, that $3|c$ and
$\exists{}x(c=3x)$. If we substitute this new value for $c$ into the
expression from $5|n$, we get:

\[
n=5(3x)
\]

We can redistribute the parenthesis because multiplication follows the
rules of associativity in the integers, so we can rewrite it as:

\[
n=(5\cdot 3)x
\] \[
n=15x
\]

In other words, we have found that $n=15x$ for some integer $x$, which
is the definition for $15|n$. By the \RUG, we have shown that, since
this statement is true for arbitrary $n$, it must be true for any
integer $n$. We have concluded that for any integer $n$, if $3|n$ and
$5|n$, then $15|n$.

Finally, because we have proven statements (a) and (b), we have proven
the original statement.

\problem{5}{3}

Let the universe of discourse be the integers. We want to show whether
the following statement is true or not: if $n$ is odd and $m$ is even,
then $n^2-m^2=n+m$. One obvious counter example is any $n$ and $m$
chosen such that $n<m$, $n$ is odd, and $m$ is even; for example
$(n,m)=(1,2)$.

\problem{5}{4}

Let the universe of discourse be the real numbers. We want to show
that, for any $x$, if $|x-3|>3$, then $x^2>6x$. Mathematically, this
is equivalent to:

\[
\forall x((|x-3|>3)\rightarrow(x^2>6x))
\]

It is important to look at the definition of $|z|$, which is:

\[
|z|=\begin{cases}z & \text{if $z\ge{}0$} \\
-z & \text{if $z<0$} \\
\end{cases}
\]

We can then rewrite the expression $|x-3|$ as:

\[
|x-3|=\begin{cases}x-3 & \text{if $x-3\ge 0$} \\
3-x & \text{if $x-3<0$}
\end{cases}
\] \[
|x-3|=\begin{cases}x-3 & \text{if $x\ge 3$} \\
3-x & \text{if $x<3$}
\end{cases}
\]

Finally, this means we can rewrite the predicate $|x-3|>3$ as:

\[
((x\ge 3)\wedge(x-3>3))\vee((x<3)\wedge(3-x>3))
\]

which can be simplified to:

\[
((x\ge 3)\wedge(x>6))\vee((x<3)\wedge(x<0))
\] \[
(x>6)\vee(x<0)
\]

This statement is equivalent to our premise, so that means we need to
solve this by cases. Namely, they are:

\[\tag{a}
\forall x((x>6)\rightarrow(x^2>6x))
\] \[\tag{b}
\forall x((x<0)\rightarrow(x^2>6x))
\]

\subproblem{a} Let $x$ be an arbitrary real number. Assume $x>6$. This
means that $\exists{}a\in\mathbb{R}^+(x=6+a)$. If we substitute this
expression into $x^2>6x$, we get:

\[
(6+a)^2>6\cdot(6+a)
\] \[
6^2+6a+a^2>6^6+6a
\] \[
a^2>0
\] \[
a^2>0
\]

Because we chose $a$ to be a positive real number (not equal to zero),
then we know that $a^2$ must also be positive (and not equal to zero),
so this case leads to a true conclusion.

\subproblem{b} Let $x$ be an arbitrary real number. Assume $x<0$. This
means that $\exists{}a\in\mathbb{R}^+(x=-a)$. We can substitute this
into the expression $x^2>6x$ to get:

\[
(-a)^2>6(-a)
\] \[
a^2>-6a
\]

The right hand side of this equation will always be negative, because
we chose $a$ to be positive. Similarly, the left hand side will be
positive, because a positive number squared will still be
positive. This means that this inequality will always be true, and
therefore this case is true.

Because we have shown that both cases separately lead to a true
conclusion, the cases ORed together will also be true, and therefore
the original conclusion is true.

\end{multicols*}
\end{document}
