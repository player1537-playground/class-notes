\documentclass{article}
\usepackage{amsmath}
\usepackage{amssymb}

\newcommand{\T}{1}
\newcommand{\F}{0}
\newcommand{\TF}[1]{\if1#1\T\else\F\fi}
\newcommand{\xintTF}[1]{\xintifboolexpr{#1}{\T}{\F}}

\newcommand{\logicrule}[2]{
\begin{array}{c}
#1 \\
\midrule
\therefore #2 \\
\end{array}
}

\setlength\parindent{0pt}
\setlength\parskip{1em}

\begin{document}

\section*{Last class}

Last class, we were working on a table of quantified statements and
their negations. The table (2.23 from the book) is as follows:

\[
\begin{array}{rcl}
\neg\lbrack\forall x(p(x))\rbrack & \Leftrightarrow & \exists x(\neg p(x)) \\
\neg\lbrack\exists x(p(x))\rbrack & \Leftrightarrow & \forall x(\neg p(x)) \\
\neg\lbrack\forall x(\neg p(x))\rbrack & \Leftrightarrow & \exists x(p(x)) \\
\neg\lbrack\exists x(\neg p(x))\rbrack & \Leftrightarrow & \forall x(p(x)) \\
\end{array}
\]

\section*{Mutliple Quantifiers}

\subsection*{Example 9.04.1}

Suppose we let $p_1:\forall{}x\forall{}y(x+y>x)$, or equivalently
$\forall{}x,y(x+y>x)$.

Is $p_1$ true or false in the universe of $\mathbb{R}$? False. We can
show that this is so, by finding one counter example: any time where
$y<0$.

Is $p_1$ true or false in the universe of $\mathbb{R}^+$?
True. Because we have now removed the all the negative numbers. We
should also notice that
$x+y>x\Rightarrow{}-x+x+y>-x+x\Rightarrow{}y>0$.

Is $p_1$ true or false in the universe of $\mathbb{N}=\{0,1,\dots\}$?
False. One counter example is letting $y=0$, which means that
$y\not>0$.

\subsection*{Example 9.04.2}

Suppose we let $p_2:\exists{}x\exists{}y(x+y>x)$.

\begin{tabular}{ll}
Universe & True or False in universe \\
\hline
$\mathbb{R}$ & True \\
$\mathbb{Z}^+$ & True \\
$\mathbb{N}$ & True \\
\end{tabular}

This is true in all of these universes, because we are only interested
in \textbf{one} instance of $y>0$.

\subsection*{Example 9.04.3}

Suppose we let $p_3:\forall{}x\exists{}y(x+y>x)$. In other words, ``For
every $x$ there is a corresponding $y$ such that $x+y>x$.

\begin{tabular}{ll}
Universe & True or False in universe \\
\hline
$\mathbb{R}$ & True \\
$\mathbb{Z}^+$ & True \\
$\mathbb{N}$ & True \\
\end{tabular}

\subsection*{Example 9.04.4}

Suppose we let $p_4:\exists{}y\forall{}x(x+y>x)$. This statement is
subtly different from the previous one. In this, we are saying:
``There exists a (single) $y$ value \textbf{for every} $x$ such that
$x+y>x$.''

\begin{tabular}{ll}
Universe & True or False in universe \\
\hline
$\mathbb{R}$ & True \\
$\mathbb{Z}^+$ & True \\
$\mathbb{N}$ & True \\
\end{tabular}

These are kind of bad examples, because it doesn't depend on $x$.

\subsection*{Example 9.04.3 (a)}

Instead, let's look at a different statement:
$\forall{}x\exists{}y(x+y=0)$. We should also note that
$x+y=0\Rightarrow{}x=-y$.

\begin{tabular}{ll}
Universe & True or False in universe \\
\hline
$\mathbb{R}$ & True \\
$\mathbb{Z}^+$ & False \\
$\mathbb{N}$ & False \\
\end{tabular}

This is false in the $\mathbb{Z}+$ and $\mathbb{N}$ cases, because
there are no negative numbers in them.

\subsection*{Example 9.04.4 (a)}

Again, let's look at: $\exists{}y\forall{}x(x+y=0)$.

\begin{tabular}{ll}
Universe & True or False in universe \\
\hline
$\mathbb{R}$ & False \\
$\mathbb{Z}^+$ & False \\
$\mathbb{N}$ & False \\
\end{tabular}

In the first case, we can think about setting $y$ to arbitrary values
and determining if we can apply all $x$ values to that value of
$y$. For example, let's try $y=4$, then we can attempt $x=-4$ and see
that it's true for one statement. But, that hasn't proved its
correctness, because we have to prove it for every $x$. So now let's
try $x=0$, and we see that the statement no longer holds. Therefore,
it is false for $\mathbb{R}$.

\textbf{Note}: We showed that $\forall{}x\exists{}y(x+y=0)$ is false
in $\mathbb{Z}^+$. Therefore $\neg\forall{}x\exists{}y(x+y=0)$ is true
in $\mathbb{Z}^+$. The negated statement is equivalent to:
$\exists{}x\neg\lbrack\exists{}y(x+y=0)\rbrack$ and
$\exists{}x\forall{}y\neg(x+y=0)$ and
$\exists{}x\forall{}y(x+y\neq{}0)$.

\section*{Changing the order of quantifiers}

Note that in the previous examples, we could not exchange
$\forall{}x\exists{}y$ and $\exists{}y\forall{}x$. However, we can
exchange $\forall{}x\forall{}y$ and $\forall{}y\forall{}x$. Similarly,
$\exists{}x\exists{}y$ and $\exists{}y\exists{}x$.

\section*{Negation of multiple quantifiers}

Suppose we were looking at the negation of multiple quantifiers. We
can show how to reason about those negations and the changing of
quantifiers.

\[
\begin{array}{cll}
 & \neg\lbrack\forall{}x\forall{}y(p(x,y)) & \\
\Leftrightarrow & \neg\forall x\lbrack\forall y(p(x,y))\rbrack & \text{``not $\forall{}x$ ``something happens''} \\
\Leftrightarrow & \exists x\neg\lbrack\forall y(p(x,y))\rbrack & \text{If for every $x$, something doesn't happen, then for some $x$ it doesn't happen} \\
\Leftrightarrow & \exists x\exists y (\neg p(x,y)) & \text{By the same logic} \\
\end{array}
\]

Similarly, we can look at the following example:

\[
\begin{array}{cll}
 & \neg\lbrack\exists x\forall y(p(x,y))\rbrack & \\
\Leftrightarrow & \neg\exists x\lbrack\forall y(p(x,y)) & \text{``there doesn't exists an $x$, such that something happens for all $y$''} \\
\end{array}
\]

I didn't finish that table. But by similar steps as before, we can
determine what the resulting expression is.

\section*{Section 2.5: More formal proof writing}

A note on some notation we'll be using: ``$\in$'' to denote ``in'' or
``element of''. For example, $x\in\mathbb{R}$ for ``$x$ is in
$\mathbb{R}$''.

\subsection*{Mathematical Definitions}

Usually stated like ``if-then" statements, but can be
interpretted/used as a biconditional.

\subsection*{Example 9.04.5: Definition 2.8}

``Let $n$ be an integer. We say $n$ is \textbf{even} if there exists
an integer, $r$, such that $n=2r$.''

We can go from ``$n$ is even'' to $n=2r$ for some particular
$r$. Similarly, we can go from $n=2r$ for some particular $r$ and
determine that $n$ is even. Mathematical definitions go both ways.

For example, we might be told that $m,k\in\mathbb{Z}$ and to assume
that $m=2k$. Without being explictly told, we can conclude that $m$ is
even.

We can also be presented with $z,l\in\mathbb{Z}$, and
$z=2(l+2l+l^2)$. We can assign a new name to the right side, such as
$r=l+2l+l^2$, and then rewrite $z$ as $z=2r$, and conclude that $z$ is
even.

\end{document}
