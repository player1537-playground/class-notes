\documentclass{article}
\usepackage{amsmath}
\usepackage{amssymb}
\usepackage{xintexpr}
\usepackage{booktabs}

\newcommand{\T}{1}
\newcommand{\F}{0}
\newcommand{\TF}[1]{\if1#1\T\else\F\fi}
\newcommand{\xintTF}[1]{\xintifboolexpr{#1}{\T}{\F}}

\newcommand{\numbercircled}[1]{\raisebox{.5pt}{\textcircled{\raisebox{-.9pt} {#1}}}}
\newcommand{\step}[1]{\text{Step \numbercircled{#1}}}

\newcommand{\onespace}{\text{ }}

\newcommand{\logicrule}[2]{
\begin{array}{c}
#1 \\
\midrule
\therefore #2 \\
\end{array}
}

\setlength\parindent{0pt}
\setlength\parskip{1em}

\begin{document}

\section*{Last class}

Last class, we showed that

\[
\logicrule{p\rightarrow q \\ q}{p}
\]

was \textbf{NOT VALID} by providing one example where true premises
did not force a true conclusion.

\section*{Example 8.31.1 (@48 (A))}

What about $\logicrule{p\rightarrow{}q\\\neg{}p}{\neg{}q}$?

We can assign truth values that will lead true premises to a false
conclusion. For example, the values $p:0$ and $q:1$, then we find that
both premises are true ($p\rightarrow{}q:1$ and $\neg{}p:1$). But, we
see that the conclusion is false ($\neg{}q:0$.

Thus,
$\underset{true\onespace{}premises}{\left\lbrack(p\rightarrow{}q)\wedge\neg{}p\right\rbrack:1}$
but $\underset{false\onespace{}conclusion}{\neg{}q:0}$

These particular truth vlaues for $p$ and $q$ demonstrate a
\textbf{counter example}.

\section*{Example 8.31.2 (@48 (b))}

Is the following argument valid?

``If $x+y=7$, then $x>y$''

\subsection*{Sub-Example 8.31.2 (a)}

Let $x=4$ and $y=3$. Then the premise, $x+y=7$, is true. The
conclusion, $x>y$, is also true.

\textbf{But} this one example does \textbf{not} mean the implication
is a $T_0$.

\subsection*{Sub-Example 8.31.2 (b)}

Let $x=3$ and $y=4$. Then the premise, $x+y=7$, is true. The
conclusion, $x>y$, is now false.

\textbf{Note}: One instance of the implication being true doees
\textbf{not} make the argument valid.

One \textit{counter example} \textbf{can} show an argument is invalid,
since it shows an instance where true premises did not lead to a true
conclusion.

\section*{Comments on The Rule of Contradiction (\#6)}

Firstly, a contradiction can be shortened to $F_0$.

The Rule of Contradiction is as follows:

\[
\logicrule{\neg p\rightarrow F_0}{p}
\]

In fact,

\[
\begin{array}{rcl|l}
\neg p\rightarrow F_0 & \Leftrightarrow & \neg(\neg p)\vee F_0 & \text{Conditional Law} \\
                      & \Leftrightarrow & p\vee F_0& \text{Double Negation Law} \\
                      & \Leftrightarrow & p & \text{Identity Law} \\
\end{array}
\]

Thus,
$(\neg{}p\rightarrow{}F_0)\underset{biconditional}{\leftrightarrow}p$
is a $T_0$. \textbf{Note}: This is not the case for all the rules of
inference.

$(\neg{}p\rightarrow{}F_0)\rightarrow{}p$ is a $T_0$ (Rule \#6).

\subsection*{Now consider}

Now consider (from Table 2.18)

\[
(p\rightarrow q)\Leftrightarrow\left\lbrack(p\wedge\neg q)\rightarrow F_0\right\rbrack
\]

The logical equivalence means that establishing the validity of the
R.H.S. implication is the same as if we established validity for the
L.H.S. implication.

The R.H.S. is taking the original premise of $p\rightarrow{}q$ (which
is $p$) and conjoin with the negation of the original premise (which
is $\neg{}q$), as an additional premise.

If these premises being true lead to a $F_0$, then we can conclude
that the ``new premise'' (which is $\neg{}q$) was \textbf{not} true
after all. i.e. that $\neg(\neg{}q)$ \textbf{is} true.

Using substitution rules, we have that for \textbf{any} statements
$p_1,p_2,\dots,p_n$, and $q$ and any $F_0$:

\[
\left(\left(p_1\wedge p_2\wedge\dots\wedge p_n\right)\rightarrow q\right)
\Leftrightarrow
\left\lbrack\left(p_1\wedge p_2\wedge\dots\wedge p_n\wedge\neg q\right)\rightarrow F_0\right\rbrack
\]

One way to think about this is:

We have a goal (show $p\rightarrow{}q$ is a valid argument).

We have an alternative goal (show $(p\wedge\neg{}q)\rightarrow{}F_0$
is valid, then conclude that $p\rightarrow{}q$ is valid).

\section*{Example 8.31.3}

Show $(p\wedge{}q)\rightarrow{}p$ is a valid argument.

Alternatively, based on the Rule of Contradiction, we can instead show
$\left\lbrack(p\wedge{}q)\wedge\neg{}p\right\rbrack\rightarrow{}F_0$
is valid.

\[
\begin{array}{cll}
\step{1} & (p\wedge q)\wedge\neg p & \text{premises and stragegy of Rule of Contradiction} \\
\step{2} & p\wedge (q\wedge\neg p) & \text{\step{1} and Associative Law of $\wedge$} \\
\step{3} & p\wedge(\neg p\wedge q) & \text{\step{2} and Commutative Law of $\wedge$} \\
\step{4} & (p\wedge\neg p)\wedge q & \text{\step{3} and Associative Law of $\wedge$} \\
\step{5} & F_0\wedge q & \text{\step{4} and Inverse Law} \\
\step{6} & F_0 & \text{\step{5} and Domination Law} \\
\step{7} & \lbrack(p\wedge q)\wedge\neg p\rbrack\rightarrow F_0 & \text{\step{1} and \step{6} and Rule of Contradiction} \\
\end{array}
\]

Thus, $(p\wedge{}q)\rightarrow{}p$ is a valid argument.

\section*{Section 2.4: Quantifiers}

\subsection*{Definition 2.5: Open Statements}

A declarative statement that

\begin{enumerate}
\item contains one or more variable
\item is not a statement \textbf{but}
\item becomes a statement when the variables are assigned allowable
  values.
\end{enumerate}

Some examples are: ``$x=2$'', ``$x+y=7$'', ``$x>y$'', ``If $x+y=7$,
then $x>y$''

Variables bring up the question: ``In what ``Universe of Discourse''
are the variables''. In other words, what are the allowable values for
each variable?

The notation for open statements looks like:

\[
\begin{array}{rcl}
p(x) & : & x=2 \\
p(x,y) & : & x+y=7 \\
q(x,y) & : & x>y \\
r(x,y) & : & \lbrack(x+y)=7)\rightarrow(x>y) \\
g(x,y) & : & p(x,y)\rightarrow{}q(x,y) \\
\end{array}
\]

\end{document}
