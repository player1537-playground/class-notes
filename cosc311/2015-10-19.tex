\documentclass{article}
\usepackage{amsmath}
\usepackage{amssymb}
\usepackage{booktabs}
\usepackage{xintexpr}

\newcommand{\T}{1}
\newcommand{\F}{0}
\newcommand{\TF}[1]{\if1#1\T\else\F\fi}
\newcommand{\xintTF}[1]{\xintifboolexpr{#1}{\T}{\F}}

\newcommand{\logicrule}[2]{
\begin{array}{l}
#1 \\
\midrule
\therefore #2 \\
\end{array}
}

\newcommand{\inv}[1]{#1^{-1}}

\renewcommand{\d}[1]{\,\textnormal{d}#1}
\newcommand{\dd}[2]{\frac{\d{#1}}{\d{#2}}}
\newcommand{\ddd}[2]{\dfrac{\d{#1}}{\d{#2}}}

\setlength\parindent{0pt}
\setlength\parskip{1em}

\begin{document}

We saw previously how to count the number of ways to select, with
repetition, $r$ of $n$ distinct objects: $C(n+r-1,r)$. In other words,
this is the number of combinations of $n$ objects taken $r$ at a time
\textit{with repetition}.

\section*{Example 10.19.1 (\#1(b) section 1.4)}

In how many ways can $10$ identical dimes be distributed among five
children if each child gets at least one dime.

\section*{Reading}

Make sure to read and understand example 1.36 and 1.37 (page 30-31)

\end{document}
