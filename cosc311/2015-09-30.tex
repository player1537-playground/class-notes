\documentclass{article}
\usepackage{amsmath}
\usepackage{amssymb}
\usepackage{booktabs}
\usepackage{xintexpr}

\newcommand{\T}{1}
\newcommand{\F}{0}
\newcommand{\TF}[1]{\if1#1\T\else\F\fi}
\newcommand{\xintTF}[1]{\xintifboolexpr{#1}{\T}{\F}}

\newcommand{\logicrule}[2]{
\begin{array}{l}
#1 \\
\midrule
\therefore #2 \\
\end{array}
}

\newcommand{\inv}[1]{#1^{-1}}

\renewcommand{\d}[1]{\,\textnormal{d}#1}
\newcommand{\dd}[2]{\frac{\d{#1}}{\d{#2}}}
\newcommand{\ddd}[2]{\dfrac{\d{#1}}{\d{#2}}}

\setlength\parindent{0pt}
\setlength\parskip{1em}

\begin{document}

\section*{Chapter 1: Counting Techniques}

This chapter focuses on problem solving, mostly.

\subsection*{Example 9.30.1}

An art class is equipped with $100$ different sculpting tools and
$500$ distinct paint brushes. How many art supplies are available to
students to either sculpt or paint?

We need to assume some things:

\begin{itemize}
\item Each sculpting tool is distinct.
\item Each paint brush is different from the others.
\item Students are \textbf{NOT} simultaneously sculpting and painting.
\end{itemize}

We conclude, by the Rule of Sum (defined below), that there are
$500+100=600$ art supplies available to students to either sculpt or
paint.

\section*{The Rule of Sum}

If a first taask can be performed in $m$ ways, while a second task can
be performed in $n$ ways, and the two tasks cannot be performed
simultaneously, then performing either task can be accomplished in any
one of $m+n$ ways.

\section*{Example 9.30.2}

Buick cars come in $4$ models, $12$ colors, $3$ engine sizes, and $2$
transmission types. (A) How many distinct Buicks can be manufactured?
(B) If one of the available colors is red, how many different red
buicks can be manufactured?

Since we only know one rule, it is apt to ask ``Does the Rule of Sum
apply here?'' The answer is no.

Luckily, we can apply the Rule of Product (defined below). By the Rule
of Product, we can separate this into the following stages (using the
given order):

\begin{enumerate}
\item Model - - - - - 4
\item Color - - - - - 12
\item Engine Size - - - - - 3
\item Transmission Type - - - - - 2
\end{enumerate}

By the rule of product: $4\cdot{}12\cdot{}3\cdot{}2=288$ distinct
buicks that can be manufactured.

(B) is an exercise for the students.

\section*{The Rule of Product}

If a procedure can be broken down into first and second stages, and if
there are $m$ possible outcomes for the first stage and if, for each
of these outcomes, there are $n$ possible outcomes for the second
stage, then the total procedure can be carried out, in the designated
order, in $mn$ ways.

\section*{Example 9.30.3}

Suppose a license plate is defined as $2$ letters followed by $4$
numbers.

(A) If no letter or digit ($0$-$9$) can be repeated, then how many
possibilities are there?

Again, we can separate into stages:

\begin{itemize}
\item Stage 1: Pick a letter for 1st space $\ldots{}26$
\item Stage 2: Pick a letter for 2nd space $\ldots{}25$
\item Stage 3: Pick a digit for 3rd space $\ldots{}10$
\item Stage 4: Pick a digit for 4rd space $\ldots{}9$
\item Stage 5: Pick a digit for 5rd space $\ldots{}8$
\item Stage 6: Pick a digit for 6rd space $\ldots{}7$
\end{itemize}

By the Rule of Product: $26\cdot25\cdot10\cdot9\cdot8\cdot7=3276000$
possible license plates.

(B) Same as (A) but with repititions allowed:

$26\cdot26\cdot10\cdot10\cdot10\cdot10=6760000$ possible license
plates.

(C) With repititions allowed, how many license plates have only vowels
(A, E, I, O, U) and even digits (0, 2, 4, 6, 8)? (Exercise)

\section*{Section 1.2: Permutations}

\subsection*{Definition 1.1: Factorial}

For an integer $n\ge0$, \textbf{$n$ factorial} (denoted $n!$) is
defined by:

\[
\begin{array}{l}
0!=1 \\
n!=(n)(n-1)(n-2)\cdots(3)(2)(1), n\ge1 \\
\end{array}
\]

\subsection*{Definition 1.2: Permutation}

Given a collection of \textit{distinct} objects, any linear
arrangement of these objects is called a \textbf{permutation} of the
collection.

The number of possible permutations of $n$ distinct objects is $n!$.

For $n$ distinct objects and $r\in\mathbb{Z}$, $1\le{}r\le{}n$, the
number of permutations of size $r$ for $n$ distinct objects is:

\[
P(n,r)=n(n-1)(n-2)\cdots(n-r+1)
\] \[
P(n,r)=\dfrac{n!}{(n-r)!}
\]

\subsection*{Example 9.30.4}

Collection: X, Y, Z (3 distinct objects).

There are 3 spots to fill with 3 objects, no repititions allowed. This
means the number of possible permutations is $3\cdot2\cdot1=6$.

\textbf{Note}: Order matters; Same letters in each arrangement, but
different order makes the arrangement different.

\subsection*{Example 9.30.5}

Suppose our collection is: A, B, C, D, E, F (6 distinct objects). How
many ways can we arrange them linearly in 3 spots?

The answer is:

\[
P(6,3)=\dfrac{6!}{(6-3)!}=\dfrac{6!}{3!}=\dfrac{720}{6}=120
\]

\subsection*{Example 9.30.6}

$M$ $I$ $S$ $S$ $I$ $P$ $P$ $I$: 11 objects

$M_1$ $I_1$ $S_1$ $S_2$ $I_2$ $S_3$ $S_4$ $I_3$ $P_1$ $P_2$ $I_4$: 11
distinguishable objects

The number of permutations of the 11 \textit{distinguishable} objects
is $11!$ What about in the case where we have
\textit{indistinguishable} objects?

%\begin{tabular}{l{20em}cl{20em}}
%Distinguishable Objects & vs & Indistinguishable Objects \\
%\hline
%Each $S$ is labeled so that they can be distinguished from each other. & \\
%Same for $I$'s and $P$'s and $M$'s & \\
%So 11 distinct objects make $11!$ permutations of the objects & \\
%\end{tabular}

%\begin{itemize}
%Stage 1: For each arrangement of 11 indistinguishable letters, there
%are $4!$ different ones when we distinguish the 4 $S$'s. Thus,
%$4!\cdot(\text{\# of arrangements of 11 objects})=\text{

\end{document}
