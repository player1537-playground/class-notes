\documentclass{article}
\usepackage[margin=0.75in]{geometry}
\usepackage{amsmath}
\usepackage{amssymb}
\usepackage{booktabs}
\usepackage{xintexpr}
\usepackage{multicol}

\setlength{\columnsep}{0.5cm}
\setlength{\columnseprule}{1pt}

\newcommand{\T}{\text{True}}
\newcommand{\F}{\text{False}}
\newcommand{\TF}[1]{\if1#1\T\else\F\fi}
\newcommand{\xintTF}[1]{\xintifboolexpr{#1}{\T}{\F}}

\newcommand{\logicrule}[2]{
\begin{array}{l}
#1 \\
\midrule
\therefore #2 \\
\end{array}
}

\newcommand{\problem}[2]{$\boxed{\text{#1 \##2}}$}
\newcommand{\subproblem}[1]{$\boxed{\text{(#1)}}$}
\newcommand{\subsolution}[2]{\boxed{#2\quad(\text{#1})}}
\newcommand{\solution}[1]{\boxed{#1}}
\newcommand{\RUS}{Rule of Universal Specification}
\newcommand{\RUG}{Rule of Universal Generalization}
\newcommand{\ROP}{Rule of Product}
\newcommand{\ROS}{Rule of Sums}

\setlength\parindent{0pt}
\setlength\parskip{0em}

\begin{document}
\begin{multicols*}{2}

%
\problem{3.1}{2}

\[
A=\{1,\{1\},\{2\}\}
\]

%
\subproblem{a} $1\in A\equiv\boxed{\T}$

%
\subproblem{b} $\{1\}\in A\equiv\boxed{\T}$

%
\subproblem{c} $\{1\}\subseteq A\equiv\boxed{\T}$

%
\subproblem{d} $\{\{1\}\}\subseteq A\equiv\boxed{\T}$

%
\subproblem{e} $\{2\}\in A\equiv\boxed{\T}$

%
\subproblem{f} $\{2\}\subseteq A\equiv\boxed{\F}$

%
\subproblem{g} $\{\{2\}\}\subseteq A\equiv\boxed{\T}$

%
\subproblem{h} $\{\{2\}\}\subset A\equiv\boxed{\T}$

\rule{\linewidth}{0.4pt}

%
\problem{3.1}{8}

\[
A=\{1,2,3,4,5,6,7,8\}
\]

%
\subproblem{a} Subsets of $A$: $\boxed{2^7}$.

%
\subproblem{b} Non-empty subsets of $A$: $\boxed{2^7-1}$.

%
\subproblem{c} Proper subsets of $A$: $\boxed{2^7-1}$.

%
\subproblem{d} Non-empty proper subsets of $A$: $\boxed{2^7-2}$.

%
\subproblem{e} Subsets of $A$ containing three elements:
$\boxed{\binom{7}{3}}$.

%
\subproblem{f} Subsets of $A$ containing $1$ and $2$: $\boxed{2^5}$.

%
\subproblem{g} Subsets of $A$ containing five elements, including $1$
and $2$: $\boxed{\binom{5}{3}}$

%
\subproblem{h} Subsets of $A$ containing an even number of elements:

\[
\boxed{\binom{7}{0}+\binom{7}{2}+\binom{7}{4}+\binom{7}{6}}
\]

%
\subproblem{i} Subsets of $A$ containing an odd number of elements:

\[
\boxed{\binom{7}{1}+\binom{7}{3}+\binom{7}{5}+\binom{7}{7}}
\]

\rule{\linewidth}{0.4pt}

%
\problem{3.1}{10}

Which sets are non-empty?

%
\subproblem{a} The set $\{x|x\in\mathbb{N}, 2x+7=3\}$ does not contain
any elements, because $\mathbb{N}$ only contains non-negative numbers,
and the solution to $2x+7=3$ is a negative
number. $\boxed{\text{Empty}}$.

%
\subproblem{b} The set $\{x\in\mathbb{Z}|3x+5=9\}$ does not contain
any elements, because $\mathbb{Z}$ contains only the integers, and the
solution to $3x+5=9$ is a rational number. $\boxed{\text{Empty}}$.

%
\subproblem{c} The set $\{x|x\in\mathbb{Q},x^2+4=6\}$ is
$\boxed{\text{Empty}}$.

%
\subproblem{d} The set $\{x\in\mathbb{R}|x^2+4=6\}$ is
$\boxed{\text{Non-empty}}$.

%
\subproblem{e} The set $\{x\in\mathbb{R}|x^2+3x+3=0\}$ is
$\boxed{\text{Empty}}$.

%
\subproblem{f} The set $\{x|x\in\mathbb{C},x^2+3x+3=0\}$ is
$\boxed{\text{Non-empty}}$.

\rule{\linewidth}{0.4pt}

%
\problem{3.1}{12}

\[
A=\{1,2,3,4,5,7,8,10,11,14,17,18\}
\] \[
|A|=12
\]

\[
A_{odd}=\{1,3,5,7,11,17\}
\] \[
|A_{odd}|=6
\]

\[
A_{even}=\{2,4,8,10,14,18\}
\] \[
|A_{even}|=6
\]

%
\subproblem{a} The number of subsets of $A$ with $6$ elements is:

\[
\binom{|A|}{6}=\boxed{\binom{12}{6}}
\]

%
\subproblem{b} The number of subsets of $A$ with $6$ elements that
contains four even integers and two odd integers.

\[
\binom{|A_{even}|}{4}\cdot\binom{|A_{odd}|}{2}=\boxed{\binom{6}{4}\cdot\binom{6}{2}}
\]

%
\subproblem{c} The number of subsets of $A$ that contain only odd integers is:

\[
2^{|A_{odd}|}=\boxed{2^6}
\]



\end{multicols*}
\end{document}
