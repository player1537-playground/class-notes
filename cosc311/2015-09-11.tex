\documentclass{article}
\usepackage{amsmath}
\usepackage{amssymb}
\usepackage{booktabs}
\usepackage{xintexpr}

\newcommand{\T}{1}
\newcommand{\F}{0}
\newcommand{\TF}[1]{\if1#1\T\else\F\fi}
\newcommand{\xintTF}[1]{\xintifboolexpr{#1}{\T}{\F}}

\newcommand{\logicrule}[2]{
\begin{array}{l}
#1 \\
\midrule
\therefore #2 \\
\end{array}
}

\setlength\parindent{0pt}
\setlength\parskip{1em}

\begin{document}

\section*{Last class}

Last class, we were looking at different ways to prove that
$\forall{}n((\text{$n$ is odd})\rightarrow(\text{$n+11$ is even}))$.

There were 3 different ways to prove this: direct proof,
contrapositive proof, and proof by contradiction.

Suppose we are interested in the generic proof of
$\forall{}x(p(x)\rightarrow{}q(x))$.

A direct proof would attempt to show directly whether
$p(x)\rightarrow{}q(x)$. A contrapositive proof would try to show that
$\neg{}q(x)\rightarrow\neg{}p(x)$. A proof by contradiction would show
that $(p(x)\wedge\neg{}q(x))\rightarrow{}F_0$.

\section*{Contrapositive Proof}

Remember that we are trying to show whether
$(\text{$n+11$ is odd})\rightarrow(\text{$n$ is even})$.

Note that the conclusion can be restated as
$\exists{}k\in\mathbb{Z}(n=2k)$. It is sometimes useful to restate the
conclusion so we have a better idea of how to get there.

It is also sometimes useful to do scratch work first, and then do the
formal write-up at the end. Our scratch work follows:

\begin{itemize}
\item I know I need to find a $k\in\mathbb{Z}$ such that $n=2k$.
\item Premise/Given: $n+11$ is odd $\Leftrightarrow$
  $\exists{}l\in\mathbb{Z}(n+11=2l+1)$ (from the definition of odd
  numbers).
\item We would really desire having another equation with
  $n=\text{something}$. We can convert our other equation into this
  form, to get $n=(2l+1)-11$.
\item Now we are interested in whether we can rewrite the
  $n=\text{something}$ equation into the form $n=2k$. We can write
  $n=2l-10$, and then $n=2(l-5)$.
\item If we let $k=l-5$, we have found a $k$ such that $n=2k$,
  therefore we have shown that $n+11$ is even.
\item Finally, we should tie this back to the original statement:
  Thus, if $n+11$ is odd, then $n$ is even. By the contrapositive we
  have that if $n$ is odd then $n+11$ is even.
\end{itemize}

\section*{Proof By Contradiction}

Here, we take our premises as $n$ is odd and $n+11$ is odd, and the
conclusion as $F_0$.

Again, we'll do scratch work:

\begin{itemize}
\item $n$ is odd $\Leftrightarrow$ $\exists{}k\in\mathbb{Z}(n=2k+1)$
\item $n+11$ is odd $\Leftrightarrow$ $\exists{}l\in\mathbb{Z}(n+11=2l+1)$
\item We can compute $n+11$ using $n$, to find that $n+11=(2k+1)+11$,
  and distribute to get $n+11=2k+12$.
\item We notice that we can rewrite $2k+12$ as $2(k+6)$. If we
  introduce a new variable, $m=k+6$, we can have $n+11=2m$.
\item However, we note that $\exists{}m(n+11=2m)$ $\Leftrightarrow$
  $n+11$ is even, but we already assumed $n+11$ is odd.
\item Therefore, we have reached a contradiction, so our original
  premise that $n+11$ is odd must be false. Thus, we see that $n$ is
  odd and $n+11$ is odd leads to a false conclusion. Therefore, $n+11$
  must not be odd, so $n+11$ is even if $n$ is odd.
\end{itemize}

\section*{Example 9.11.1}

Let $n\in\mathbb{Z}$. Prove that if $n$ is odd, then $7n+8$ is odd.

\subsection*{Proof By Contradiction}

\begin{itemize}
\item $n$ is odd $\Leftrightarrow$ $\exists{}k(n=2k+1)$
\item $7n+8$ is even $\Leftrightarrow$ $\exists{}l(7n+8=2l)$
\item $7(n)+8=7(2k+1)+8=14k+9$
\item If we introduce a variable $m=7k+4$, we have that $7n+8=2m+1$,
  which by the definition of odd, means that $7n+8$ is odd.
\item However, we previously assumed that $7n+8$ is even, and we can't
  have it be both odd and even, so our new assumption that $7n+8$ is
  even must be false.
\item Because we just showed that we can't have both $n$ odd and
  $7n+8$ even, it must be the case that $n$ is odd leads to $7n+8$ is
  odd.
\end{itemize}

\end{document}
