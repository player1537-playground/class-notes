\documentclass{article}
\usepackage{amsmath}
\usepackage{pgfplots}
\usepackage{xintexpr}
\usepackage{booktabs}

\newcommand\T{1}
\newcommand\F{0}
\newcommand\TF[1]{\if1#1\T\else\F\fi}
\newcommand\xintTF[1]{\xintifboolexpr{#1}{\T}{\F}}

\setlength\parindent{0pt}
\setlength\parskip{1em}
\pgfkeys{/pgfplots/MyAxisStyle/.style={xmin=-10,xmax=10, ymin=-10,ymax=10,height=6cm,width=6cm}}

\begin{document}

\section*{Section 2.2 Continued}

Consider the implication statement (when $x$ is a real number)

\begin{equation}\tag{P}
\text{``If $x^2>16$, then $x<-4$ or $x>4$.''}
\end{equation}

Write down the contrapositive, converse, and inverse of $P$.

It might be a good idea to disect the statement into primitive statements:

\[
\begin{cases}
p: x^2 > 16 \\
q: x<-4 \\
r: x>4
\end{cases}
\]

Similarly, it might be important to write the negations of each primitive statement

\[
\begin{cases}
\neg p: x^2 \le 16 \\
\neg q: x \ge -4 \\
\neg r: x \le 4
\end{cases}
\]

Once it's disected, it would also be good to rewrite $P$ using $p$, $q$, and $r$.

``If $p$, then $q$ or $r$'' (note: exclusive or is technically correct, we can ignore it, though).

And then write it using logical connectives

\[
p\rightarrow (q\vee r)
\]

Now we can focus on finding the 3 additional statements

\begin{tabular}{cccc}
What & Expression & Simplified & Relation used \\
\hline
Original & $p\rightarrow (q\vee r)$ & & \\
Contrapositive & $\neg (q\vee r)\rightarrow \neg p$ & $(\neg q\wedge \neg r)\rightarrow \neg p$ & DeMorgan's Law \\
Converse & $(q \vee r)\rightarrow p$ & & \\
Inverse & $\neg p \rightarrow \neg(q\vee r$ & $\neg p\rightarrow\neg q\wedge\neg r$ & DeMorgan's Law
\end{tabular}

Finally, we want to translate back into original context (English).

\begin{tabular}{rl}
What & English \\
\hline
Original & ``If $x^2>16$, then $x<-4$ or $x>4$'' \\
Contrapositive & ``If $x\ge-4$ and $x\le 4$, then $x^2\le 16$'' \\
Converse & ``If $x<-4$ or $x>4$, then $x^2>16$'' \\
Inverse & ``If $x^2\le 16$, then $x\ge-4$ and $x\le 4$''
\end{tabular}

\section*{Contrapositive and the original statement}

\[
p\rightarrow q\Leftrightarrow \neg q\rightarrow\neg p
\]

In other words, the contrapositive of a given implication is logically equivalent to the given implication.

This is very useful for re-expressing a given implication, especially one that contains a lot of negative statements within.

For example, if we have the statement
$(\neg{}q\wedge\neg{}r)\rightarrow\neg{}p$. Rewriting this in
contrapositive form gives $p\rightarrow{}q\vee{}r$, which is more
straightforward.

\section*{Section 2.3 Logical Implication: Rule of Inference}

Recall from the 8/21 lecture, when we introduced an ``argument'':

\[
\underset{Premises}{(p_1\wedge p_2\wedge \cdots\wedge p_n)} \rightarrow\underset{Conclusion}{q}
\]

Recall that an argument is ``valid'' if true premises drive a true
conclusion (ref: page 53 in section 2.1, page 62 in section 2.3).

\subsection*{Example 8.26.1}

Is the following argument valid?

\[
\underset{p_1}{p}\wedge\underset{p_2}{(p\rightarrow q)}\rightarrow q
\]

Lets look at a truth table:

\[
\begin{array}{ cc|cc|c }
p & q & p_1 & p_2 & q \\
\midrule
\xintFor #1 in {0,1} \do {
  \xintFor #2 in {0,1} \do {
    \TF#1 &
    \TF#2 &
    \TF#1 &
    \xintifboolexpr {not(#1) | #2}{\T}{\F} &
    \TF#2 \\
}}
\end{array}
\]

Which rows have all preimises true?

The last row is the only one where all premises are true. Notice that
the corresponding conclusion is true. Thus, the argument is valid.

\subsection*{Example 8.21.2}

Is the following argument valid?

\[
\underset{p_1}{\neg p}\wedge \underset{p_2}{\neg q} \rightarrow (p\vee q)
\]

Construct a truth table for the statement

\[
\begin{array}{cc|cc|c}
p & q & \neg p & \neg q & p\vee q \\
\midrule
\xintFor #1 in {0,1} \do {
  \xintFor #2 in {0,1} \do {
    \TF#1 &
    \TF#2 &
    \xintTF {not(#1)} &
    \xintTF {not(#2)} &
    \xintTF {#1|#2} \\
}}
\end{array}
\]

The first row is the only one with all true premises. However, the
conclusion is not true, so the statement is \textbf{not} a valid
argument.

\subsection*{Valid arguments through tautologies}

Another way to think about a valid argument is that a valid argument
will be a tautology (Back to example 8.26.1).

\[
\begin{array}{cc|cc|c}
p & q & p\rightarrow q & p\wedge (p\rightarrow q) & \lbrack p\wedge (p\rightarrow q)\rbrack\rightarrow q \\
\midrule
\xintFor #1 in {0,1} \do {
  \xintFor #2 in {0,1} \do {
    \TF #1 &
    \TF #2 &
    \xintTF {not(#1) | #2} &
    \xintTF {#1 & (not(#1) | #2)} &
    \xintTF {not(#1 & (not(#1) | #2)) | #2} \\
}}
\end{array}
\]

Thus, $\lbrack{}p\wedge(p\rightarrow{}q)\rbrack\rightarrow{}q$ is a
valid argument, because the rightmost column is all true, and is thus
a $T_0$.

For further practice, show that example 8.26.2 that the implication is
not a $T_0$.

This result is for arbitrary, primitive statements; but the rules of
substitution allow us to extend this result to \textbf{ANY} statements
$p$ and $q$ (primitive or not).

\textit{Look over page 78, table 2.19 for Friday.}

\end{document}
