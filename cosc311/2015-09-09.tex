\documentclass{article}
\usepackage{amsmath}
\usepackage{amssymb}
\usepackage{booktabs}

\newcommand{\T}{1}
\newcommand{\F}{0}
\newcommand{\TF}[1]{\if1#1\T\else\F\fi}
\newcommand{\xintTF}[1]{\xintifboolexpr{#1}{\T}{\F}}

\newcommand{\logicrule}[2]{
\begin{array}{l}
#1 \\
\midrule
\therefore #2 \\
\end{array}
}

\setlength\parindent{0pt}
\setlength\parskip{1em}

\begin{document}

\section*{Last class}

Last class we were talking about mathematical definitions, how they're
stated, and how they're used. Previously we used the definition of the
even and odd integers. Check post @70 for a list of mathematical
definitions.

Remember, when we're talking about mathematical definitions, then we
usually treat it as a biconditional. For example, the definition of
even is:

\[
\forall n\in\mathbb{Z}\left(\text{$n$ is even}\leftrightarrow\exists r\in\mathbb{Z}\left(n=2r\right)\right)
\]

\section*{The Rule of Universal Specification (R.U.S.)}

If we know that
$\forall{}x\in\mathbb{Z}(x\ge{}0\rightarrow{}x^2{}\ge{}0)$ is true,
then R.U.S. allows us to use this statement to conclude that for a
given $c\in\mathbb{Z}$ where $c\ge{}0$, then $c^2\ge{}0$ sinc ethe
implication under the given premise applies to \textbf{all} members of
$\mathbb{Z}$ that meet the criteria of the premise.

In general, when given a \textbf{premise} of the form
$\forall{}x(p(x))$, then any value, $c$, in the unvierse can be
substituted for $x$, and we can conclude $p(c)$ is true.

\section*{Example 9.09.1}

Consider for universe $U$ and $c\in{}U$ and statements $p(x)$ and $q(x)$ defined in $U$:

$\logicrule{\forall{}x(p(x)\rightarrow{}q(x)) \\ p(c)}{q(c)}$

Also:

$\logicrule{\forall{}x(p(x)\rightarrow{}q(x)) \\ \neg{}q(c)}{\neg{}p(c)}$

Remember that $p(x)\rightarrow{}q(x)\Leftrightarrow\neg{}q(x)\rightarrow\neg{}p(x)$

We can also generalize with
$\forall{}x\forall{}y\forall{}z(p(x,y,z)\rightarrow{}q(x,y,z)$.

\section*{The Rule of Universal Generalization (page 110)}

When given a conclusion to prove of the form

\[
\forall x(p(x))
\]

(or more generally, $\forall{x}\forall{}y(p(x,y))$ or more variables).

\begin{enumerate}
\item Let $x$ (or $x$, $y$, $z$, etc) be a specific but arbitrarily
  chosen element of the universe.
\item Prove the statement $p(x)$ (or $p(x,y,z,\dots)$) to be true for
  this arbitrary element $x$.
\item Conclude that $p(x)$ is true \textbf{for all} $x$ because,
  although $x$ was specific, it was arbitrarily chosen.
\end{enumerate}

\subsection*{Example 9.09.2}

Let $U=\mathbb{R}^+$.

Prove: $\forall{}x\forall{}y(x^2+y^2>0)$

Remember: even though this doesn't have an implication, we can still
prove it.

\[
\begin{array}{@{}|l|@{}}
\text{Scratch Work} \\
\midrule
\text{What are my premises: $x,y\in\mathbb{R}^+$} \\
\text{Goal: show $\forall{}x\forall{}y(x^2+y^2>0)$} \\
\text{$x,y\in\mathbb{R}^+$ arbitrary} \\
\text{Look at $x^2+y^2>0$} \\
\text{Key question: } \\
\text{How could I show $a+b>0$, $a,b\in\mathbb{R}^+$} \\
\text{1. Show $a>0$ and $b>0$} \\
\text{2. Show $a>-b$} \\
\text{3. Show $b>-a$} \\
\text{Using 1:} \\
\text{$x\in\mathbb{R}^+\Rightarrow{}x>0$} \\
\text{$\Rightarrow{}x\cdot{}x>0\cdot{}x$} \\
\text{because $x>0$ the $>$ sign doesn't change} \\
\text{$\Rightarrow{}x^2>0$} \\
\text{Similarly, $y^2>0$} \\
\text{So, $x^2+y^2>0+y^2>0$} \\
\text{$x^2+y^2>0$} \\
\end{array}
\]

Now, let's write the actual proof.

Let $x,y\in\mathbb{R}^+$ be arbitrarily chosen. Then $x>0$. Since
multiplying both sides of an inequality by a positive number does not
change the direction of the inequality and $x>0$, we have:

\[
\begin{array}{rcl}
x>0 & \Leftrightarrow & x \cdot x>x\cdot 0 \\
    & \Leftrightarrow & x^2 > 0 \\
\end{array}
\]

Similarly, since $y\in\mathbb{R}^+$, $y>0$, and we have $y^2>0$. Then
since $x^2>0$ and $y^2>0$

\[
x^2+y^2 > 0+y^2 > 0
\]

And thus, $x^2+y^2>0$.

Since $x$ and $y$ were arbitrary in $\mathbb{R}^+$, the above holds
for any $x,y\in\mathbb{R}^+$. Thus, for all $x$ and for all $y$,
$x^2+y^2>0$.

\section*{Different Proof Methods}

Let the universe $U=\mathbb{Z}$.

Prove: If $n$ is odd, then $n+11$ is even.

\subsection*{Direct Proof}

\[
\begin{array}{l|l}
\text{Given/Premises} & \text{Goal/Conclusion} \\
\midrule
\text{$n$ is odd} & \text{$n+11$ is even} \\
\end{array}
\]

Let $n$ be an odd integer. Then by definition of odd,
$\exists{}k(n=2k+1)$. Thus, $n+11=(2k+1)+11=2k+12=2(k+6)=2l$, where we
define $l=k+6$, $l\in\mathbb{Z}$. By definition of even,
$\exists{}l(n+11=2l)$, thus $n+11$ is even.

\subsection*{Contrapositive Proof}

\[
\begin{array}{l|l}
\text{Given/Premises} & \text{Goal/Conclusion} \\
\midrule
\text{$n+11$ is $\underbrace{\text{not even}}_{\text{odd}}$} & \text{$n$ is $\underbrace{\text{not odd}}_{\text{even}}$} \\
\end{array}
\]

\subsection*{Proof by Contradiction}

\[
\begin{array}{l|l}
\text{Given/Premises} & \text{Goal/Conclusion} \\
\midrule
\text{$n$ is odd} & F_0 \\
\text{$n+11$ is $\underbrace{\text{not even}}_{\text{odd}}$} & \\
\end{array}
\]




\end{document}
