\documentclass{article}
\usepackage{amsmath}
\usepackage{amssymb}
\usepackage{booktabs}
\usepackage{xintexpr}

\newcommand{\T}{1}
\newcommand{\F}{0}
\newcommand{\TF}[1]{\if1#1\T\else\F\fi}
\newcommand{\xintTF}[1]{\xintifboolexpr{#1}{\T}{\F}}

\newcommand{\logicrule}[2]{
\begin{array}{l}
#1 \\
\midrule
\therefore #2 \\
\end{array}
}

\newcommand{\inv}[1]{#1^{-1}}

\renewcommand{\d}[1]{\,\textnormal{d}#1}
\newcommand{\dd}[2]{\frac{\d{#1}}{\d{#2}}}
\newcommand{\ddd}[2]{\dfrac{\d{#1}}{\d{#2}}}

\setlength\parindent{0pt}
\setlength\parskip{1em}

\begin{document}

\section*{General Result of Arrangements of In/Distinguishable Objects}

Page 9. Blue box.

If there are $n$ objects with $n_1$ indistinguishable objects of a
first type, $n_2$ indistinguishable objects of a second type,
$\cdots$, and $n_r$ indistinguishable objects of an $r$th type, where
$n_1+n_2+\cdots+n_r=n$, then there are
$\frac{n!}{n_1!n_2!\cdots{}n_r!}$ linear arrangements of the given $n$
objects.

\textbf{Note} Permutations count linear arrangements (where order
matters) and in which objects \textit{cannot} be repeated.

If repititions were allowed, then the number of possible arrangements
of size $r$ of $n$ objects is $n^r$, with $r\ge0$.

\section*{Section 1.3: Combinations/The Binomial Theorem}

\subsection{Example 10.05.1 (A)}

Consider 4 distinct objects: $\{W,X,Y,Z,\}$. How many possibilities
are there for picking 2 of the 4 objects?

If order matters (e.g. $\langle{}W,X\rangle$ and $\langle{}X,W\rangle$
are different arrangements), then we are asking the number of
permutations of size $r=2$ out of $n=4$ objects, then:

\[
P(n,r)=P(4,2)=\frac{4!}{(4-2)!}=\frac{4!}{2!}=12\text{ permutations}
\]

What if order \textit{does not} matter (e.g. $\langle{}W,X\rangle$ and
$\langle{}X,W\rangle$ are the same selection or ``combinations'' of
the 2 letters $X$ and $W$)?

Then for each selection/combination of 2 objects (of size $r=2$) of
the 4 objects, there are $2!$ permutations (e.g. for
$\langle{}W,X\rangle$ and $\langle{}X,W\rangle$, there are $2\cdot1$
ways to arrange any 2 letters in 2 spots).

So, we have the relation with permutations:

\[
P(4,2)=12\text{ permutations}=2!\text{ permutations}\cdot(\text{number of combinations of size $2$})
\]

\[
\text{number of combinations of size $2$} = \frac{12\text{ permutations}}{2!\text{ permutations}}=6
\]

\subsection*{Example 10.05.1 (B)}

What if $r=3$ in the above example?

Blah blah.

\[
\text{number of combinatins} = \frac{P(4,3)}{3!}
\]

\subsection*{In General}

Page 15.

\[
C(n,r)=\frac{P(n,r)}{r!}=\frac{n!}{r!(n-r)!}\quad 0\le r\le n
\]

We also use the notation:

\[
C(n,r)=\binom{n}{r}
\]

Which is read as ``$n$ choose $r$''.

\end{document}
