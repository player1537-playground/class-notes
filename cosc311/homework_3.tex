\documentclass{article}
\usepackage[margin=0.75in]{geometry}
\usepackage{amsmath}
\usepackage{amssymb}
\usepackage{booktabs}
\usepackage{xintexpr}
\usepackage{multicol}

\setlength{\columnsep}{0.5cm}
\setlength{\columnseprule}{1pt}

\newcommand{\T}{1}
\newcommand{\F}{0}
\newcommand{\TF}[1]{\if1#1\T\else\F\fi}
\newcommand{\xintTF}[1]{\xintifboolexpr{#1}{\T}{\F}}

\newcommand{\logicrule}[2]{
\begin{array}{l}
#1 \\
\midrule
\therefore #2 \\
\end{array}
}

\newcommand{\problem}[2]{$\boxed{\begin{array}{l}\text{#1} \\ \text{\##2}\end{array}}$}
\newcommand{\subproblem}[1]{(#1)}

\setlength\parindent{0pt}
\setlength\parskip{1em}

\begin{document}
\begin{multicols*}{2}

\problem{2.4}{2}

\subproblem{a i} $p(3)\vee(q(3)\vee\neg r(3)) = \boxed{\T}$

\subproblem{a ii} $p(2)\rightarrow(q(2)\rightarrow r(2)) = \T\rightarrow(\T\rightarrow\T) = \boxed{\T}$

\subproblem{a iii} $(p(2)\wedge q(2))\rightarrow r(2) = (\T\wedge\T)\rightarrow\T = \boxed{\T}$

\subproblem{a iv} $p(0)\rightarrow (\neg q(-1)\leftrightarrow r(1)) = \T\rightarrow(\neg \F\leftrightarrow\T) = \boxed{\T}$

\subproblem{b} \[
p(x)\wedge r(x) \Leftrightarrow 0 < x \le 3
\] \[
q(x)\Leftrightarrow \text{$x$ is even}
\] \[
\boxed{x\in\left\{2\right\}}
\]

%

\problem{2.4}{8}

\subproblem{a} \[
p(x): x^2-8x+15=0
\] \[
(x-5)(x-3)=x^2-8x+15=0
\] \[
\forall x\in\{3,5\}(p(x))
\]

A counter example would occur when $p(x)$ is true and $q(x)$ is
false. $p(x)$ is true when $x$ is $3$ or $5$. For both of these
values, $q(x)$ is also true, so there are no counter examples, so the
statement is $\boxed{\text{true}}$.

\subproblem{f} \[
\forall x(\neg q(x)\rightarrow\neg p(x)) \Leftrightarrow \forall x(p(x)\rightarrow q(x))
\]

The second statement we just proved to be true, so we know that the
first statement is also true (because they are contrapositives of each
other). Therefore, the statement is $\boxed{\text{True}}$.

\subproblem{g} \[
\exists x(p(x)\rightarrow (q(x)\wedge r(x)))
\] \[
\exists x\lbrack(\underbrace{p(x)\rightarrow q(x)}_{P_1})\wedge(\underbrace{p(x)\rightarrow r(x)}_{P_2})\rbrack
\]

We already proved $P_1$ to be true, so now we just need to check
$P_2$.

\[
\forall x\in\{3,5\}(p(x))
\]

For a counter example to exist, $p(x)$ would have to be true (which
occurs for values $x\in\{3,5\}$) while $r(x)$ is false (which occurs
for any $x<0$). Because $\forall x\in\{3,5\}(r(x))$ is true, we have
$P_2$ is true. Because both $P_1$ and $P_2$ are true, the statement is
$\boxed{\text{True}}$.

%

\problem{2.4}{12}

The open statement $p(x,y): \text{``x divides y''}$ is equivalent to
$p(x,y):\exists a(y=ax)$ when $(x\ne 0)\wedge(y\ne 0)$. If $x=y=0$,
then $p(x,y)$ is undefined and therefore false.

\subproblem{a i} $p(3,7) = \boxed{\F}$

\subproblem{a ii} $p(3, 27) = \boxed{\T}$

\subproblem{a iii} The only obvious possible counter example would be
$y=0$, but based on the above definition of ``divides'', we can let
$a=0$, and correctly have $0=0\cdot{}1$, therefore: \[
\forall y(p(1, y)) = \boxed{\T}
\]

\subproblem{a iv} For all values of $x\ne 0$, we know that the
statement is true, because, from the above definition, we can let
$a=0$ and have $0=0\cdot x$, which is true for all $x$. However, the
equation is false for $x=y=0$, which is in our domain, and acts as the
only counter example of this statement.

\[
\forall x(p(x, 0)) = \boxed{\F}
\]

\subproblem{a v} This statement is also false for the case that $x=0$:

\[
\forall x(p(x,x)) = \boxed{\F}
\]

\subproblem{a vi} We can trivially relate this problem back to the
part iii, where we let $x=1$, and we already know $\forall y(p(1,y))$,
so we can conclude:

\[
\forall y\exists x(p(x,y)) = \boxed{\T}
\]

\end{multicols*}
\end{document}
