\documentclass{article}
\usepackage[margin=0.75in]{geometry}
\usepackage{amsmath}
\usepackage{amssymb}
\usepackage{booktabs}
\usepackage{xintexpr}
\usepackage{multicol}

\setlength{\columnsep}{0.5cm}
\setlength{\columnseprule}{1pt}

\newcommand{\T}{1}
\newcommand{\F}{0}
\newcommand{\TF}[1]{\if1#1\T\else\F\fi}
\newcommand{\xintTF}[1]{\xintifboolexpr{#1}{\T}{\F}}

\newcommand{\logicrule}[2]{
\begin{array}{l}
#1 \\
\midrule
\therefore #2 \\
\end{array}
}

\newcommand{\problem}[2]{$\boxed{\begin{array}{l}\text{#1} \\ \text{\##2}\end{array}}$}
\newcommand{\subproblem}[1]{(#1)}

\setlength\parindent{0pt}
\setlength\parskip{1em}

\begin{document}
\begin{multicols*}{2}

\problem{2.4}{2}

\subproblem{a i} $p(3)\vee(q(3)\vee\neg r(3)) = \boxed{\T}$

\subproblem{a ii} $p(2)\rightarrow(q(2)\rightarrow r(2)) = \T\rightarrow(\T\rightarrow\T) = \boxed{\T}$

\subproblem{a iii} $(p(2)\wedge q(2))\rightarrow r(2) = (\T\wedge\T)\rightarrow\T = \boxed{\T}$

\subproblem{a iv} $p(0)\rightarrow (\neg q(-1)\leftrightarrow r(1)) = \T\rightarrow(\neg \F\leftrightarrow\T) = \boxed{\T}$

\subproblem{b} \[
p(x)\wedge r(x) \Leftrightarrow 0 < x \le 3
\] \[
q(x)\Leftrightarrow \text{$x$ is even}
\] \[
\boxed{x\in\left\{2\right\}}
\]

%

\problem{2.4}{8}

\subproblem{a} \[
p(x): x^2-8x+15=0
\] \[
(x-5)(x-3)=x^2-8x+15=0
\] \[
\forall x\in\{3,5\}(p(x))
\]

A counter example would occur when $p(x)$ is true and $q(x)$ is
false. $p(x)$ is true when $x$ is $3$ or $5$. For both of these
values, $q(x)$ is also true, so there are no counter examples, so the
statement is $\boxed{\text{True}}$.

\subproblem{f} \[
\forall x(\neg q(x)\rightarrow\neg p(x)) \Leftrightarrow \forall x(p(x)\rightarrow q(x))
\]

The second statement we just proved to be true, so we know that the
first statement is also true (because they are contrapositives of each
other). Therefore, the statement is $\boxed{\text{True}}$.

\subproblem{g} \[
\exists x(p(x)\rightarrow (q(x)\wedge r(x)))
\] \[
\exists x\lbrack(\underbrace{p(x)\rightarrow q(x)}_{P_1})\wedge(\underbrace{p(x)\rightarrow r(x)}_{P_2})\rbrack
\]

We already proved $P_1$ to be true, so now we just need to check
$P_2$.
\[
\forall x\in\{3,5\}(p(x))
\]
For a counter example to exist, $p(x)$ would have to be true (which
occurs for values $x\in\{3,5\}$) while $r(x)$ is false (which occurs
for any $x<0$). Because $\forall x\in\{3,5\}(r(x))$ is true, we have
$P_2$ is true. Because both $P_1$ and $P_2$ are true, the statement is
$\boxed{\text{True}}$.

%

\problem{2.4}{12}

The open statement $p(x,y): \text{``x divides y''}$ is equivalent to
$p(x,y):\exists a(y=ax)$ when $(x\ne 0)\wedge(y\ne 0)$. If $x=y=0$,
then $p(x,y)$ is undefined and therefore false.

\subproblem{a i} $p(3,7) = \boxed{\F}$

\subproblem{a ii} $p(3, 27) = \boxed{\T}$

\subproblem{a iii} The only obvious possible counter example would be
$y=0$, but based on the above definition of ``divides'', we can let
$a=0$, and correctly have $0=0\cdot{}1$, therefore: \[
\forall y(p(1, y)) = \boxed{\T}
\]

\subproblem{a iv} For all values of $x\ne 0$, we know that the
statement is true, because, from the above definition, we can let
$a=0$ and have $0=0\cdot x$, which is true for all $x$. However, the
equation is false for $x=y=0$, which is in our domain, and acts as the
only counter example of this statement.
\[
\forall x(p(x, 0)) = \boxed{\F}
\]

\subproblem{a v} This statement is also false for the case that $x=0$: \[
\forall x(p(x,x)) = \boxed{\F}
\]

\subproblem{a vi} We can trivially relate this problem back to the
part iii, where we let $x=1$, and we already know $\forall y(p(1,y))$,
so we can conclude: \[
\forall y\exists x(p(x,y)) = \boxed{\T}
\]

\subproblem{a vii} Suppose we chose a arbitrary $y=y_0$. For an
arbitrary $x=x_0$, we know that $\exists a(y_0=ax_0)$, so let's choose
$a=a_0$. If we choose a separate $x=x_1$, we could arrive at a
separate $a=a_1$. So now we have $y_0=a_0x_0=a_1x_1$. For this to be
true, we would have to have $y_0$ be a multiple of $x_0$ and $x_1$,
for which the easiest multiple is just $x_0x_1$.

We could then extend this process to an arbitrary number of $x$ values
$x\in\{x_0,x_1,\dots,x_n\}$, with $y=\prod\limits_{i=0}^nx_n$. Because
our universe (the integers) is unbounded, we will have to have
$y=\lim\limits_{n\rightarrow{}\infty}\prod\limits_{i=0}^nx_n$, but this limit
is just $\infty$, which is not in our universe, so there is no way for
there to be a single $y$ value that is divisible by everything in our
universe. Therefore \[
\exists y\forall x(p(x,y)) = \boxed{\F}
\]

\subproblem{a viii} Suppose we chose an arbitrary $y=y_0$ and
$x=x_0$. We note that we choose $x_0\ne{}0$ and $y_0\ne{}0$ because
the premises of the implication would be false for $x_0=0$ or $y_0=0$,
and the only contradictions can come from true premises leading to
false conclusions. The statement we are interested in, then, is
$p(x_0,y_0)\wedge{}p(y_0,x_0)\rightarrow(x_0=y_0)$. Let's assume that
$y_0\ne{}x_0$, and without loss of generality $y_0=x_0+\delta$ for
some $\delta\in\mathbb{Z}$ where $\delta\ne{}0$.

The first premise says that $\exists{}a(y_0=ax_0)$, and the second
that $\exists{}b(x_0=by_0)$. Suppose that we choose $a=a_0$ and
$b=b_0$. With this in mind, we can rewrite the equations from the
premises as $y_0=a_0x_0$ and $x_0=b_0y_0$. From here, we can replace
$y_0$ with $x_0+\delta$, to get $x_0+\delta=a_0x_0$ and
$x_0=b_0(x_0+\delta)$. Finally, we can do a substitution using these
two new equations to get $(b_0(x_0+\delta))+\delta=a_0x_0$.

We can then distribute and simplify the expression to get
\[
b_0x_0+b_0\delta+\delta=a_0x_0
\] \[
(b_0-a_0)x_0=-(b_0+1)\delta
\]

Because we assumed that $\delta\ne{}0$, we know that $b_0\ne{}a_0$ (or
else $b_0-a_0$ would be zero, and we'd have $0=-(b_0+1)\delta$. The
case where $b_0=-1$ is described later).

If we just do a basic substitution of the equations before the
substitution of $y_0$, we end up with
\[
y_0=a_0(b_0y_0)
\] \[
a_0=b_0^{-1}
\]

The only way for this expression to be true is if $a_0=b_0\in\{-1,1\}$, but
this violates our earlier statement that $b_0\ne{}a_0$, therefore
we've reached a contradiction, so our assumption that $\delta\ne{}0$
must be false.

In the case that $b_0=-1$, we would arrive at the previous step, where
$a_0=b_0^{-1}=(-1)^{-1}=-1$. Here we arrive at $a_0=b_0$ again, and
have reached the same contradiction as before. Therefore
\[
\forall{}x\forall{}y((p(x,y)\wedge p(y,x))\rightarrow(x=y)) = \boxed{\T}
\]

\problem{2.4}{18}

\subproblem{a} \[
\neg\lbrack\exists x(p(x)\vee q(x))\rbrack
\] \[
\forall x(\neg(p(x)\vee q(x)))
\] \[
\boxed{\forall x(\neg p(x)\wedge\neg q(x))}
\]

\subproblem{b} \[
\neg\lbrack\forall x(p(x)\wedge\neg q(x))\rbrack
\] \[
\exists x(\neg(p(x)\wedge\neg q(x)))
\] \[
\boxed{\exists x(\neg p(x)\vee q(x))}
\]

\subproblem{c} \[
\neg\lbrack\forall x(p(x)\rightarrow q(x))\rbrack
\] \[
\exists x(\neg(p(x)\rightarrow q(x)))
\] \[
\exists x(\neg(\neg p(x)\vee q(x)))
\] \[
\boxed{\exists x(p(x)\wedge\neg q(x))}
\]

\subproblem{d}

\end{multicols*}
\end{document}
