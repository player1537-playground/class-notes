\documentclass{article}
\usepackage[margin=0.75in]{geometry}
\usepackage{amsmath}
\usepackage{amssymb}
\usepackage{booktabs}
\usepackage{xintexpr}
\usepackage{multicol}

\setlength{\columnsep}{0.5cm}
\setlength{\columnseprule}{1pt}

\newcommand{\T}{1}
\newcommand{\F}{0}
\newcommand{\TF}[1]{\if1#1\T\else\F\fi}
\newcommand{\xintTF}[1]{\xintifboolexpr{#1}{\T}{\F}}

\newcommand{\logicrule}[2]{
\begin{array}{l}
#1 \\
\midrule
\therefore #2 \\
\end{array}
}

\newcommand{\problem}[2]{$\boxed{\text{#1 \##2}}$}
\newcommand{\subproblem}[1]{$\boxed{\text{(#1)}}$}
\newcommand{\subsolution}[2]{\boxed{#2\quad(\text{#1})}}
\newcommand{\solution}[1]{\boxed{#1}}
\newcommand{\RUS}{Rule of Universal Specification}
\newcommand{\RUG}{Rule of Universal Generalization}
\newcommand{\ROP}{Rule of Product}
\newcommand{\ROS}{Rule of Sums}

\setlength\parindent{0pt}
\setlength\parskip{0em}

\begin{document}
\begin{multicols*}{2}

\textbf{Tanner Hobson}

%
\problem{1.3}{8}

%
\subproblem{a} For a particular suit, a flush is just drawing 5 cards
out of the 13 possible ones. Because order does not matter, this is
just:

\[
\binom{13}{5}
\]

But this can occur for each suit, so the actual number of
possibilities is:

\[
\subsolution{a}{4\binom{13}{5}}
\]

%
\subproblem{b} To select 4 aces, we can first select the non-ace,
where we are selecting 1 card out of $52-4=48$ cards:

\[
\binom{48}{1}=48
\]

Then for the remaining 4 cards, we have 4 possibilities (the aces):

\[
\binom{4}{4}=1
\]

The total number of possibilities is:

\[
\subsolution{b}{1\cdot48=48}
\]

%
\subproblem{c} First we can choose the suit which the four-of-a-kind
will reside in, which is choosing 1 suit out of the 4 possible suits:

\[
\binom{4}{1}=4
\]

Then we can choose the 4 cards, which is:

\[
\binom{13}{4}
\]

Finally, for the last card, we can either choose the same suit as the
four-of-a-kind, which is:

\[
\binom{13-4}{1}=9
\]

Or we can choose one that isn't the same suit as the four-of-a-kind:

\[
\binom{52-13}{1}=39
\]

This amounts to the total number of possibilities:

\[
\binom{4}{1}\cdot\binom{13}{4}\cdot\left(\binom{13-4}{1}+\binom{52-13}{1}\right)
\] \[
\subsolution{c}{4\cdot\binom{13}{4}\cdot\left(9+39\right)}
\]

%
\subproblem{d} First we select the 3 aces, whose number of
possibilities is:

\[
\binom{4}{3}
\]

Then we can choose the two jacks:

\[
\binom{4}{2}
\]

Which amounts to:

\[
\subsolution{d}{\binom{4}{3}\cdot\binom{4}{2}}
\]

%
\subproblem{e} First we select the three aces:

\[
\binom{4}{3}=4
\]

Then we select what value the pair will have. We can discard the
possibility of the pair being an ace, because there is only 1 ace
left.

\[
\binom{12}{1}=12
\]

Now we can select which suits the pair will comprise of:

\[
\binom{4}{2}
\]

Which gives us:

\[
\subsolution{e}{4\cdot12\cdot\binom{4}{2}}
\]

%
\subproblem{f} First we select which suit the three-of-a-kind will
belong to:

\[
\binom{4}{1}=4
\]

Then we can select the three card values:

\[
\binom{13}{3}
\]

Then we can select the value of the pair. We can either have the value
be one of those from the three-of-a-kind or not. The number of
possibilities with the value being in the three-of-a-kind is:

\[
\binom{3}{1}=3
\]

If it is one of the three-of-a-kind, then we only have 3 suits we can
select from:

\[
\binom{3}{2}=3
\]

Otherwise, if it's not of the three-of-a-kind values, then the number
of possibilities for the values are:

\[
\binom{13}{1}=13
\]

And the number of possibilities of suits are:

\[
\binom{4}{2}
\]

Therefore the total number of possibilities is:

\[
\subsolution{f}{4\cdot\binom{13}{3}\cdot\left(3\cdot3+13\cdot\binom{4}{2}\right)}
\]

%
\subproblem{g} First we can select the suit:

\[
\binom{4}{1}=4
\]

And then select the card values from that suit:

\[
\binom{13}{3}
\]

For the next cards, we can select each one's suit. We have 4
possibilities: (1) all cards are in same suit, (2) triple and 4th card
are in same suit, 5th in different, (3) triple and 5th in same suit,
4th in different, (4) triple, 4th, and 5th all in different
suits. Conceptually, (2) and (3) are the same, so we can take twice
the number of possibilites of (2) and that will account for (3). The
number of possibilites of the suit for (1) is just:

\[
\binom{1}{1}=1
\]

With the total number of possible values:

\[
\binom{13-3}{2}=\binom{10}{2}
\]

The number of possible suits for the 4th card in (2) is:

\[
\binom{1}{1}=1
\]

With the nunber of values:

\[
\binom{13-3}{1}=10
\]

For the 5th card, the number of possible suits is:

\[
\binom{3}{1}=3
\]

With the number of values:

\[
\binom{13}{1}=13
\]

For (4), the number of possible suits for the 4th card is:

\[
\binom{3}{1}=3
\]

And the number of values for the 4th card:

\[
\binom{13}{1}=13
\]

The number of suits for the 5th card is:

\[
\binom{2}{1}=2
\]

And the number of values:

\[
\binom{13}{1}=13
\]

In total, the number of possibilities is:

\[
\subsolution{g}{4\cdot\binom{13}{3}\cdot\left(1\cdot\binom{10}{2}+2\cdot10\cdot3\cdot13+3\cdot13\cdot2\cdot13\right)}
\]

%
\subproblem{h} Without loss of generality, we can let the first two
cards be the first pair. We first select the value of the pair:

\[
\binom{13}{1}=13
\]

And then the two suits for the pair:

\[
\binom{4}{2}
\]

For the next pair, we have two options: (1) we select the same value
as before, or (2) we select a different value. The number of ways we
can do (1) is:

\[
\binom{1}{1}=1
\]

The number of ways we can select the suit for the second pair is:

\[
\binom{2}{2}=1
\]

For the last card, we can select from the remaining $52-4=48$ cards:

\[
\binom{48}{1}=48
\]

For (2), the number of ways we can select the value of the pair is:

\[
\binom{12}{1}=12
\]

And the number of ways we can select the suit of the second pair is:

\[
\binom{4}{2}
\]

Because it doesn't matter if we get three cards of the same value,
since we will still have two pairs, then we can select the last card
out of the remaining $52-4=48$ cards:

\[
\binom{48}{1}=48
\]

The total number of possibilities is:

\[
13\cdot\binom{4}{2}\cdot\left(1\cdot1\cdot48+12\cdot\binom{4}{2}\cdot48\right)
\] \[
\subsolution{h}{13\cdot\binom{4}{2}\cdot\left(48+12\cdot48\cdot\binom{4}{2}\right)}
\]

%
\problem{1.3}{12}

Let the books be unique and represented by:

\[
B=\{b_1,b_2,\cdots,b_{12}\}
\]

%
\subproblem{a} This problem is that, if we take a particular ordering
of the books in $B$ and insert dividers between the 3rd/4th, 6th/7th,
etc books to divide the books among the kids. Because these divisions
are constant, the problem is simplified to only calculating the number
of permutations of the 12 books. In other words, the number of ways to
distribute the books is:

\[
\subsolution{a}{12!}
\]

%
\subproblem{b} This time the separators go between the 2nd/3rd,
4th/5th, and 8th/9th positions. Therefore, the problem is identical to
the previous one:

\[
\subsolution{b}{12!}
\]

%
\problem{1.3}{14}

We want to find $n$, the total number of seniors, such that:

\[
\binom{n}{11}=12376
\]

If we assume an upper limit of $30$, we can perform a binary search to
find the correct value:

\[
\begin{array}{l|l|l}
n & C(n,11) & High/Low \\
\midrule
30 & 54627300 & High \\
15 & 1365 & Low \\
22 & 705432 & High \\
18 & 31824 & High \\
16 & 4368 & Low \\
17 & 12376 & Equal \\
\end{array}
\]

Therefore the answer is:

\[
\solution{n=17}
\]

%
\problem{1.3}{16}

%
\subproblem{a} We can simply directly calculate this:

\[
\subsolution{a}{\sum\limits_{i=1}^6 (i^2+1)=21}
\]

%
\subproblem{b} Again.

\[
\subsolution{b}{\sum\limits_{j=-2}^2 (j^3-1)=-2}
\]

%
\subproblem{c} Again.

\[
\subsolution{c}{\sum\limits_{i=0}^{10} (1+(-1)^i)=12}
\]

%
\subproblem{d} We can break this up into sub-intervals:

\[
\sum\limits_{k=n}^{2n} (-1)^k
\] \[
\sum\limits_{i=0}^{n-1} \left(\sum\limits_{k=n+2i}^{n+2i+1} (-1)^k\right)
\]

The right sum is just two elements, so we can expand that:

\[
\sum\limits_{i=0}^{n-1} \left((-1)^{n+2i} + (-1)^{n+2i+1}\right)
\] \[
\sum\limits_{i=0}^{n-1} \left((-1)^{n}(-1)^{2i} + (-1)^{n}(-1)^{2i}(-1)^{1}\right)
\]

But we are told that $n$ is an even number, therefore
$\exists{}a\in\mathbb{Z}(n=2a)$. So we can rewrite this as:

\[
\sum\limits_{i=0}^{n-1} \left((-1)^{2a}(-1)^{2i} + (-1)^{2a}(-1)^{2i}(-1)^{1}\right)
\]

When negative one is raised to an even power, it becomes a positive
one, and vice versa with odd powers. So we can now write:

\[
\sum\limits_{i=0}^{n-1} \left((1)(1) + (1)(1)(-1)\right)
\] \[
\sum\limits_{i=0}^{n-1} \left(1-1\right)
\] \[
\sum\limits_{i=0}^{n-1} 0
\]

The summation of zeros will still be zero, so this sum is:

\[
\subsolution{d}{\sum\limits_{k=n}^{2n} (-1)^k=0}
\]

%
\subproblem{e} We can directly compute this one.

\[
\subsolution{e}{\sum\limits_{i=1}^{6} i(-1)^i=3}
\]

%
\problem{1.3}{18}

%
\subproblem{a} We wish to determine the number of strings of length 10
with: four 0's, three 1's, and three 2's. We can do this by first
selecting the positions of the 0's, where we are choosing 4 indexes
out of 10:

\[
\binom{10}{4}
\]

Then we can choose the three indexes for the positions of the 1's from
the remaining $10-4=6$ positions:

\[
\binom{6}{3}
\]

And finally, we select the three indexes for the positions of the 2's
from the remaining $6-3=3$ positions:

\[
\binom{3}{3}=1
\]

Therefore the final solution is:

\[
\subsolution{a}{\binom{10}{4}\binom{6}{3}\cdot1}
\]

%
\subproblem{b} For the strings of length 10 with at least eight 1's,
we can first select the positions of the 1's:

\[
\binom{10}{8}
\]

Then we can choose the positions of the other two numbers out of the
remaining $10-8=2$ positions:

\[
\binom{2}{2}=1
\]

And then we can choose which numbers will take those positions:

\[
10^2
\]

Giving us a total:

\[
\subsolution{b}{\binom{10}{8}\cdot1\cdot10^2}
\]

%
\subproblem{c} The number of strings of length 10 with weight 4, can
be thought of as ``How many ways can we distribute four identical 1's
into 10 containers'', which is given as:

\[
\binom{10+4-1}{4}
\] \[
\subsolution{c}{\binom{13}{4}}
\]

%
\problem{1.3}{26}

In these problems, we are looking for the coefficient of the
$w^2x^2y^2z^2$ term.

%
\subproblem{a} This time, the binomial is:

\[
(w+x+y+z+1)^{10}
\]

If we let $a=1$, then the binomial is:

\[
(w+x+y+z+a)^{10}
\]

And we know that the coefficient of the $w^2x^2y^2z^2a^0$ term is:

\[
\frac{10!}{2!2!2!2!0!}
\] \[
\subsolution{a}{\frac{10!}{2^4}}
\]

%
\subproblem{b} The binomial is:

\[
(2w-x+3y+z-2)^{12}
\]

If we introduce the variables $x_1$ through $x_5$, defined as:

\[
\langle x_1,x_2,x_3,x_4,x_5\rangle=\langle 2w,-1x, 3y, z, -2\rangle
\]

Then the coefficient of the $x_1^2x_2^2x_3^2x_4^2x_5^0$ term is:

\[
\frac{12!}{2!2!2!2!0!}=\frac{12!}{2^4}
\]

But we want the coefficient in terms of $x$, $y$, etc. Because the
above value is the coefficient of the $x_1^2x_2^2\cdots$ term, we can
substitute the values for each $x_i$:

\[
\frac{12!}{2^4}\cdot(2w)^2(-1x)^2(3y)^2(z)^2(-2)^0
\] \[
\frac{12!}{2^4}\cdot2^2\cdot1\cdot3^2\cdot1^2\cdot1\cdot w^2x^2y^2z^2
\]

Therefore the coefficient is:

\[
\subsolution{b}{\frac{12!3^2}{2^2}}
\]

%
\subproblem{c} The binomial is:

\[
(v+w-2x+y+5z+3)^{12}
\]

Again, we introduce the variables $x_1$ through $x_6$:

\[
\langle x_1,x_2,x_3,x_4,x_5,x_6\rangle=\langle v, w, -2x, y, 5z, 3\rangle
\]

Because we are only interested in the coefficient of the term with
$x,x,y,z$ in it, and not $v$, then we know that the $v$ exponent must
be 0. Therefore, we are now looking for the coefficient of the term:

\[
x_1^0x_2^2x_3^2x_4^2x_5^2x_6^0
\]

Which is given as:

\[
\frac{12!}{0!2!2!2!2!0!}=\frac{12!}{2^4}
\]

We can expand each $x_i$ term again to get:

\[
\frac{12!}{2^4}\cdot v^0w^2(-2x)^2y^2(5z)^23^0
\] \[
\frac{12!2^25^2}{2^4} w x y z
\] \[
\subsolution{c}{\frac{12!5^2}{2^2}}
\]


\end{multicols*}
\end{document}
