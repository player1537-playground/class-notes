\documentclass{article}
\usepackage[margin=0.75in]{geometry}
\usepackage{amsmath}
\usepackage{amssymb}
\usepackage{booktabs}
\usepackage{xintexpr}
\usepackage{multicol}

\setlength{\columnsep}{0.5cm}
\setlength{\columnseprule}{1pt}

\newcommand{\T}{1}
\newcommand{\F}{0}
\newcommand{\TF}[1]{\if1#1\T\else\F\fi}
\newcommand{\xintTF}[1]{\xintifboolexpr{#1}{\T}{\F}}

\newcommand{\logicrule}[2]{
\begin{array}{l}
#1 \\
\midrule
\therefore #2 \\
\end{array}
}

\newcommand{\problem}[2]{$\boxed{\begin{array}{l}\text{#1} \\ \text{\##2}\end{array}}$}
\newcommand{\subproblem}[1]{(#1)}

\setlength\parindent{0pt}
\setlength\parskip{1em}

\begin{document}
\begin{multicols*}{2}

\problem{2.5}{6}

\subproblem{a} Valid. Let \[
p(x):\text{``$x$ is a mail carrier''}
\] \[
q(x):\text{``$x$ carries mace''}
\], so we can rewrite the statements as: \[
\forall x(p(x)\rightarrow q(x))
\] \[
\exists y(p(y)
\]
 $p(x):\text{``$x$ carries mace''}$, and
$q(x):\text{``$x$ is a mail carrier''}$, then the first statement is
$\forall{}x(q(x)\rightarrow{}p(x))$, the second is $\exists{}y(q(y))$,
and the last is $p(y)$. Because $q(y)$ is true, and we know that
$\forall{}x(q(x)\rightarrow{}p(x))$ then $q(y)\rightarrow{}p(y)$,
therefore $p(y)$.

%

\problem{2.5}{12}

%

\problem{2.5}{14}

%

\problem{2.5}{16}

\end{multicols*}
\end{document}
