\documentclass{article}
\usepackage[margin=0.75in]{geometry}
\usepackage{amsmath}
\usepackage{amssymb}
\usepackage{booktabs}
\usepackage{xintexpr}
\usepackage{multicol}

\setlength{\columnsep}{0.5cm}
\setlength{\columnseprule}{1pt}

\newcommand{\T}{1}
\newcommand{\F}{0}
\newcommand{\TF}[1]{\if1#1\T\else\F\fi}
\newcommand{\xintTF}[1]{\xintifboolexpr{#1}{\T}{\F}}

\newcommand{\logicrule}[2]{
\begin{array}{l}
#1 \\
\midrule
\therefore #2 \\
\end{array}
}

\newcommand{\problem}[2]{$\boxed{\text{#1 \##2}}$}
\newcommand{\subproblem}[1]{(#1)}
\newcommand{\RUS}{Rule of Universal Specification}

\setlength\parindent{0pt}
\setlength\parskip{1em}

\begin{document}
\begin{multicols*}{2}

\problem{2.5}{6}

\subproblem{a} $\boxed{\text{Valid}}$ Let \[
p(x):\text{``$x$ is a mail carrier''}
\] \[
q(x):\text{``$x$ carries mace''}
\] so we can rewrite the statements as: \[
\forall x(p(x)\rightarrow q(x))
\] \[
\exists y(p(y))
\] and the conclusion we want to determine is \[
q(y)
\] By the \RUS, we know that because $y$ is a member of our universe,
then $p(y)\rightarrow{}q(y)$. One of our givens is $p(y)$, therefore
we have $q(y)$ by the Rule of Detachment.

\subproblem{b} $\boxed{\text{Invalid}}$ Let \[
p(x):\text{``$x$ is a law abiding citizen''}
\] \[
q(x):\text{``$x$ pays their taxes''}
\] so we can rewrite the given statements as \[
\forall x(p(x)\rightarrow q(x))
\] \[
\exists y(q(y))
\] and the conclusion \[
p(y)
\] By the \RUS, we know that since $y$ is an element of our universe,
then $p(y)\rightarrow{}q(y)$. However, we are told that $q(y)$ is
true, and it is not true in general that
$(p(y)\rightarrow{}q(y))\rightarrow(q(y)\rightarrow{}p(y))$ therefore
we can not conclude that $p(y)$ is true.

\subproblem{c} $\boxed{\text{Invalid}}$ Let \[
p(x):\text{``$x$ is concerned about the environment''}
\] \[
q(x):\text{``$x$ recycles plastic containers''}
\] and we can rewrite the statements as \[
\forall x(p(x)\rightarrow q(x))
\] \[
\exists y(\neg p(y))
\] and the conclusion is \[
\neg q(y)
\] By the \RUS, we know that because $y$ is in our universe, then
$p(y)\rightarrow{}q(y)$. However, it is not necessarily true that the
inverse $\neg{}p(y)\rightarrow\neg{}q(y)$ is true if the original
statement is true. Therefore, we can not conclude that $\neg{}q(y)$ is
true.


%

\problem{2.5}{12}

\subproblem{a} By the definition of even, we know that
$\exists{}a(k=2a)$ and $\exists{}b(l=2b)$. From there, we can
substitute in these new values into the sum $k+l=2a+2b$, and then
factor out the $2$ to get $k+l=2(a+b)$. Because the integers are
closed under addition, we know that $a+b\in\mathbb{Z}$ as well, so we
can assign a new name for this sum: $c=a+b$. This means that the sum
$k+l$ now has the form $2c$ for some $c\in\mathbb{Z}$, which is the
definition of even, therefore $k+l$ is even.

\subproblem{b} By the definition of even, we know that
$\exists{}a(k=2a)$ and $\exists{}b(l=2b)$. From there, we can
substitute in these new values into the product $kl=(2a)(2b)$. We can
expand the parentheses to get $kl=2\cdot{}2ab$, and then factor out
the $2$ to get $kl=2(2ab)$. Because the integers are closed under
multiplication, we know that $2ab\in\mathbb{Z}$, so we can assign a
new name for them: $c=2ab$. This means that the product $kl$ now has
the form $2c$ for some $c\in\mathbb{Z}$, which is the definition of
even, therefore $kl$ is even.

%

\problem{2.5}{14}

By the definition of odd, we know that $\exists{}a(n=2a+1)$. We can
then substitute this new value for $n$ into the power $n^2$ to get
$n^2=(2a+1)^2$. By expanding the power, we get
$n^2=2\cdot{}2a^2+2\cdot{}2a+1$. We can then factor out the $2$ in the
first two summands to get $n^2=2\cdot{}(2a^2+2a)+1$. Because the
integers are closed under addition and multiplication, we know that
the sum $2a^2+2a\in\mathbb{Z}$, and can assign a new name for this
value: $b=2a^2+2a$. This means that the power can now be written as
$n^2=2b+1$ for some $b\in\mathbb{Z}$, and by the definition of odd,
this means that $n^2$ is odd.

%

\problem{2.5}{16}

To prove whether $n^2$ is even iff $n$ is even, we split the problem
into two parts: if $n^2$ is even, then $n$ is even, and if $n$ is
even, then $n^2$ is even. Parts (a) and (b) respectively.

\subproblem{a} To tackle this proof, we will instead focus on the
contrapositive: if $n$ is not even (odd), then $n^2$ is not even
(odd). We just proved this statement in 2.5 \#14, so the proof will not
be repeated.

\subproblem{b} To simplify this proof, we rely on the results of 2.5
\#12 (b), which states that if $k$, $l$ are even, then $kl$ is even. If
we let $k=n$ (which is even, by our premise) and $l=n$ (still even),
then $kl=n^2$ must also be even.

\end{multicols*}
\end{document}
