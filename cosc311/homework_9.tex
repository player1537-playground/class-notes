\documentclass{article}
\usepackage[margin=0.75in]{geometry}
\usepackage{amsmath}
\usepackage{amssymb}
\usepackage{booktabs}
\usepackage{xintexpr}
\usepackage{multicol}

\setlength{\columnsep}{0.5cm}
\setlength{\columnseprule}{1pt}

\newcommand{\T}{\text{True}}
\newcommand{\F}{\text{False}}
\newcommand{\TT}{1}
\newcommand{\FF}{0}
\newcommand{\TF}[1]{\if1#1\TT\else\FF\fi}
\newcommand{\xintTF}[1]{\xintifboolexpr{#1}{\TT}{\FF}}

\newcommand{\logicrule}[2]{
\begin{array}{l}
#1 \\
\midrule
\therefore #2 \\
\end{array}
}

\newcommand{\problem}[2]{$\boxed{\text{#1 \##2}}$}
\newcommand{\subproblem}[1]{$\boxed{\text{(#1)}}$}
\newcommand{\subsolution}[2]{\boxed{#2\quad(\text{#1})}}
\newcommand{\solution}[1]{\boxed{#1}}
\newcommand{\RUS}{Rule of Universal Specification}
\newcommand{\RUG}{Rule of Universal Generalization}
\newcommand{\ROP}{Rule of Product}
\newcommand{\ROS}{Rule of Sums}

\newcommand{\multistep}[1]{\begin{array}{rl} #1 \end{array}}
\newcommand{\subeq}{\subseteq}
\newcommand{\sub}{\subset}

\setlength\parindent{0pt}
\setlength\parskip{0em}

\begin{document}
\begin{multicols*}{2}

\section*{Tanner Hobson}

%
\problem{3.2}{2}

Let $A$ be the interval $\lbrack0,3\rbrack$, also denoted by:

\[
A=\{x|0\le x\le 3\}
\]

And $B$ the interval $\lbrack2,7)$, denoted by:

\[
B=\{x|2\le x<7\}
\]

%
\subproblem{a} The intersection $A\cap{}B$ can be calculated with:

\[
\begin{array}{rl}
A\cap B&=\{x|(x\in A)\wedge(x\in B)\} \\
&=\{x|(0\le x\le 3)\wedge(2\le x<7)\} \\
&=\{x|2\le x\le 3\}
\end{array}
\]

\[
\subsolution{a}{A\cap B=\lbrack 2,3\rbrack} \\
\]

%
\subproblem{b} The union $A\cup{}B$ can be calculated with:

\[
\begin{array}{rl}
A\cup B&=\{x|(x\in A)\vee(x\in B)\} \\
&=\{x|(0\le x\le 3)\vee(2\le x<7)\} \\
&=\{x|0\le x<7\} \\
\end{array}
\]

\[
\subsolution{b}{A\cup B=\lbrack 0,7)}
\]

%
\subproblem{c} The complement $\overline{A}$ is given by:

\[
\multistep{
\overline{A}&=\{x|\neg(x\in A)\} \\
&=\{x|\neg(0\le x\le 3)\} \\
&=\{x|\neg((0\le x)\wedge(x\le 3))\} \\
&=\{x|\neg(0\le x)\vee\neg(x\le 3)\} \\
&=\{x|(x<0)\vee(x>3)\} \\
}
\]

\[
\subsolution{c}{\overline{A}=(-\infty,0)\cup(3,\infty)}
\]

%
\subproblem{d} The symmetric difference $A\triangle{}B$ is calculated
from:

\[
\multistep{
A\triangle B&=\{x|(x\in A\cup B)\wedge(x\not\in A\cap B)\} \\
&=\{x|(x\in\lbrack0,7))\wedge(x\not\in \lbrack2,3\rbrack)\} \\
&=\{x|(0\le x<7)\wedge(\neg((2\le x)\wedge(x\le 3)))\} \\
&=\{x|(0\le x<7)\wedge((x<2)\vee(3<x))\} \\
&=\{x|(0\le x<2)\vee(3<x<7)\} \\
}
\]

\[
\subsolution{d}{A\triangle B=\lbrack0,2)\cup(3,7)}
\]

%
\subproblem{e} The difference $A-B$ is calculated via:

\[
\multistep{
A-B&=\{x|(x\in A)\wedge(x\not\in B)\} \\
&=\{x|(0\le x\le 3)\wedge\neg(2\le x<7)\} \\
&=\{x|(0\le x\le 3)\wedge((x<2)\vee(7\le x))\} \\
&=\{x|0\le x<2\} \\
}
\]

\[
\subsolution{e}{A-B=\lbrack0,2)}
\]

%
\subproblem{f} The difference $B-A$ can be computed with:

\[
\multistep{
B-A&=\{x|(x\in B)\wedge(x\not\in A)\} \\
&=\{x|(2\le x<7)\wedge\neg(0\le x\le 3)\} \\
&=\{x|(2\le x<7)\wedge((x<0)\vee(3<x))\} \\
&=\{x|3<x<7\} \\
}
\]

\[
\subsolution{f}{B-A=(3,7)}
\]

%
\problem{3.2}{4a}

Let $A$, $B$, $C$, $D$, and $E$ be sets defined by:

\[
\begin{array}{cc}
A=\{2n|n\in\mathbb{Z}\} & B=\{3n|n\in\mathbb{Z}\} \\
C=\{4n|n\in\mathbb{Z}\} & D=\{6n|n\in\mathbb{Z}\} \\
E=\{8n|n\in\mathbb{Z}\} & \\
\end{array}
\]

Let the universe of discourse $U=\mathbb{Z}$.

%
\subproblem{i} We want to determine if
$E\subseteq{}C\subseteq{}A$. This is equivalent to:

\[
E\subseteq{}C\subseteq{}A\Leftrightarrow((E\subseteq C)\wedge(C\subseteq A))
\]

We can first check whether $E\subseteq{}C$ is true:

\[
\multistep{
E\subseteq{}C&\Leftrightarrow \forall x((x\in E)\rightarrow(x\in C) \\
&\Leftrightarrow \forall x((\exists n(x=8n))\rightarrow(\exists m(x=4m))) \\
&\Leftrightarrow \forall x((\exists n(x=(4\cdot2)n))\rightarrow(\exists m(x=4m))) \\
&\Leftrightarrow \forall x((\exists n(x=4(2n)))\rightarrow(\exists m(x=4m))) \\
}
\]

Because $2n$ is also an integer, then the $m$ that satisfies the
conclusion is twice the $n$ value of the premise. Therefore
$E\subseteq{}C$. By the same logic, when we check whether
$C\subseteq{}A$, we will find that it is also true, because $4$ is a
multiple of $2$.

Therefore, the statement $E\subseteq{}C\subseteq{}A$ is $\subsolution{i}{\T}$.

%
\subproblem{ii} We want to determine if $A\subeq{}C\subeq{}E$ is
true. We can split this up into two parts:

\[
A\subeq{}C\subeq{}E\Leftrightarrow((A\subeq C)\wedge(C\subeq E))
\]

We can look at the first part $A\subeq{}C$:

\[
\multistep{
A\subeq{}C&\Leftrightarrow \forall x((x\in A)\rightarrow(x\in C)) \\
&\Leftrightarrow \forall x((\exists n(x=2n))\rightarrow(\exists m(x=4m))) \\
}
\]

We can immediately determine a case where this is false. Let $x=2$
(which is an integer), then the $n$ that satisfies $2=2n$ is
$n=1$. Therefore the premise of the implication is true. However, the
conclusion is false, because there is no integer $m$ such that $2=4m$,
so true premises led to a false conclusion, and the statement is
$\subsolution{ii}{\F}$.

%
\subproblem{iii} We want to determine if $B\subeq{}D$.

\[
\multistep{
B\subeq D&\Leftrightarrow \forall x((x\in B)\rightarrow(x\in D)) \\
&\Leftrightarrow \forall x((\exists n(x=3n))\rightarrow(\exists m(x=6m))) \\
}
\]

Again, we can come up with a counter-example for this implication. Let
$x=3$, then the $n$ which satisfies $3=3n$ is $n=1$, which means the
premise is true. The conclusion $\exists m(3=6m)$ is false because
there is no value of $m$ in our universe (the integers) which makes
this statement true. Therefore, the statement $B\subeq{}D$ is
$\subsolution{iii}{\F}$.

%
\subproblem{iv} We want to determine if $D\subeq{}B$.

\[
\multistep{
D\subeq B&\Leftrightarrow \forall x((x\in D)\rightarrow(x\in B)) \\
&\Leftrightarrow \forall x((\exists n(x=6n))\rightarrow(\exists m(x=3m))) \\
&\Leftrightarrow \forall x((\exists n(x=3(2n)))\rightarrow(\exists m(x=3m))) \\
}
\]

If, for arbitrary $n$ that satisfies $x=6n$, we let $m=2n$, then we
can see that the conclusion is true. Because $n$ was chosen
arbitarily, the statement is true in general, and therefore
$D\subeq{}B$ is $\subsolution{iv}{\T}$.

%
\subproblem{v}

%
\subproblem{vi}

%
\problem{3.2}{6a}

We want to prove the statement: if $A\subeq{}B$ and $C\subeq{}D$, then
$A\cap{}C\subeq{}B\cap{}D$ and
$A\cup{}C\subeq{}B\cup{}D$. Mathematically, this can be written as:

\[
\begin{array}{l}
((A\subeq B)\wedge(C\subeq D))\Rightarrow \\
\quad \lbrack((A\cap C)\subeq(B\cap D))\,\wedge\,((A\cup C)\subeq(B\cup D))\rbrack
\end{array}
\]

Instead of looking at both parts at once, we can separate this into
two statements:

\[\tag{i}
((A\subeq B)\wedge(C\subeq D))\Rightarrow ((A\cap C)\subeq(B\cap D))
\] and \[\tag{ii}
((A\subeq B)\wedge(C\subeq D))\Rightarrow ((A\cup C)\subeq(B\cup D))
\]

Note that we can rewrite the givens as:

\[\tag{1}
(x\in A)\Rightarrow(x\in B) \\
\] \[\tag{2}
(x\in C)\Rightarrow(x\in D) \\
\]

%
\subproblem{i} We first want to look at what the conclusion really
means:

\[
\begin{array}{l}
((A\cap C)\subeq(B\cap D)) \\
\quad\Leftrightarrow \forall x((x\in A\cap C)\rightarrow(x\in B\cap D)) \\
\quad\Leftrightarrow \forall x((x\in\{y|y\in A\wedge y\in C)\rightarrow(x\in\{z|z\in B\wedge z\in D)) \\
\quad\Leftrightarrow \forall x((x\in A\wedge x\in C)\rightarrow(x\in B\wedge x\in D)) \\
\end{array}
\]

By using equation (1) and (2), we can replace the $x\in{}A$ and
$x\in{}C$ terms with $x\in{}B$ and $x\in{}D$ respectively.

\[
\begin{array}{l}
((A\cap C)\subeq(B\cap D)) \\
\quad\Leftrightarrow \forall x((x\in B\wedge x\in D)\rightarrow(x\in B\wedge x\in D))
\end{array}
\]

Which is trivially true, because for any predicate $p$,
$p\rightarrow{}p$ is true. Overall, this means that when the premises
are true, we reach a true conclusion, and therefore statement (i) is
$\subsolution{i}{\T}$.

%
\subproblem{ii} If we were to rewrite the expression like in part (i),
we would find that the conclusion of (ii) is equivalent to:

\[
\begin{array}{l}
((A\cup C)\subeq(B\cup D)) \\
\quad\Leftrightarrow \forall x((x\in A\vee x\in C)\rightarrow(x\in B\vee x\in D)) \\
\end{array}
\]

Using equations (1) and (2), we can rewrite this as:

\[
\begin{array}{l}
((A\cup C)\subeq(B\cup D)) \\
\quad\Leftrightarrow \forall x((x\in B\vee x\in D)\rightarrow(x\in B\vee x\in D)) \\
\end{array}
\]

Which is trivially true again. Therefore statement (ii) is also
$\subsolution{ii}{\T}$.

%
\problem{3.2}{8a}

%
\problem{3.2}{14c}

To show that $A\cup{}(A\cap{}B)\equiv{}A$, we can construct a
membership tables as follows:

\[
\begin{array}{c|c|c|c}
A & B & A\cap B & A\cup(A\cap B) \\
\midrule
\xintFor #1 in {0,1}\do{
  \xintFor #2 in {0,1}\do{
    \TF#1 &
    \TF#2 &
    \xintTF{#1 & #2} &
    \xintTF{#1 | (#1 & #2)} \\
  }
}
\end{array}
\]

We see that columns 1 ($A$) and 4 ($A\cup(A\cap{}B)$) are equivalent,
which means that the expression $A\equiv{}A\cup(A\cap{}B)$ is
$\solution{\T}$.

%
\problem{3.2}{18}

Let the universe $U=\mathbb{Z}^+$. For every $n\in\mathbb{Z}^+$, let
$A_n$ be the set:

\[
A_n=\{1,2,\cdots,n\}
\]

%
\subproblem{a \& c} We want to determine the values of the sets $X_a$
and $X_c$:

\[
X_a=\bigcup\limits_{n=1}^7 A_n
\] \[
X_c=\bigcup\limits_{n=1}^m A_n
\]

It's easier to first look at $X_c$ and use its result to compute
$X_a$. First we note that, in general:

\[
(n \le m)\Leftrightarrow(A_n\subseteq A_m)
\]

We can also note that, for arbitrary sets $X$ and $Y$, then:

\[
X\subseteq Y \Leftrightarrow (X\cup Y=Y)
\]

Combining these two properties, we have:

\[\tag{1}
(n\le m)\Leftrightarrow(A_n\cup A_m=A_m)
\]

Now we can apply this result to the union. To be more general, we will
look at the union from $n=k$ to $n=m$. The first thing we want to do
is pull out the first 2 terms from the union:

\[
Z = \bigcup\limits_{n=k}^m A_n = A_k\cup A_{k+1}\cup\left(\bigcup\limits_{n=k+2}^m A_n\right)
\]

Based on our result from (1), because $k\le{}k+1$, then
$A_k\cup{}A_{k+1}=A_{k+1}$. Therefore, this union is equivalent to:

\[
Z = A_{k+1}\cup\left(\bigcup\limits_{n=k+2}^m A_n\right)
\]

If we pull out another term from the larger union, then we are looking
at:

\[
Z = A_{k+1}\cup{}A_{k+2}\cup\left(\bigcup\limits_{n=k+3}^m A_n\right)
\]

By the same logic, this distills to:

\[
Z = A_{k+2}\cup\left(\bigcup\limits_{n=k+3}^m A_n\right)
\]

If we were to continue this process, we would find that the union
ultimately evaluates to:

\[
\bigcup\limits_{n=k}^m A_n = A_m
\]

If we let $k=1$ and $m=7$, then we can calculate the value of
$X_a$. Similarly, if we let $k=1$ and $m$ be arbitrary, then $X_c$'s
value can be found.

\[
\subsolution{a \& c}{\begin{array}{l}
X_a=A_7 \\
X_c=A_m \\
\end{array}}
\]

%
\subproblem{b \& d} This problem boils down to the same general idea
as the previous problem. We want to find the values of $X_b$ and
$X_d$, defined as:

\[
X_b=\bigcap\limits_{n=1}^{11} A_n
\] \[
X_d=\bigcap\limits_{n=1}^m A_n
\]

Again, we note that in general:

\[
(n\le m)\Leftrightarrow(A_n\subseteq A_m)
\]

This time, however, we recognize that, for arbitrary sets $X$ and $Y$:

\[
(X\subseteq Y)\Leftrightarrow(X\cap Y=X)
\]

Combining these, we get:

\[\tag{2}
(n\le m)\Leftrightarrow(A_n\cap A_m=A_n)
\]

We can start by taking out the first two terms of $X_d$:

\[
X_d=A_1\cap A_2\cap\left(\bigcap\limits_{n=3}^m A_n\right)
\]

Then using (2), because $1\le2$ then $A_1\cap{}A_2=A_1$. Therefore:

\[
X_d=A_1\cap\left(\bigcap\limits_{n=3}^m A_n\right)
\]

We can pull out another term:

\[
X_d=A_1\cap A_3\cap\left(\bigcap\limits_{n=4}^m A_n\right)
\]

Again, because $1\le3$ then by (2), $A_1\cap{}A_3=A_1$. Therefore:

\[
X_d=A_1\cap\left(\bigcap\limits_{n=4}^m A_n\right)
\]

If we were to continue this process, we would find that, ultimately:

\[
X_d=A_1
\]

Because this result does not depend on $m$, then we know that the
values of $X_b$ and $X_d$ are:

\[
\subsolution{b \& d}{\begin{array}{l}
X_b=A_1 \\
X_d=A_1 \\
\end{array}}
\]

%
\problem{3.3}{2}

Let the set of all balls be represented by:

\[
B=\{1,2,\cdots,20\}
\]

and one outcome (2 balls) represented by:

\[
x=\langle b_1, b_2\rangle,\quad b_1,b_2\in B
\]

%
\subproblem{a} If the first ball is replaced before the second ball is
drawn, then the sample space can be represented as:

\[
\subsolution{a}{\Omega = \{\langle b_1,b_2\rangle|b_1\in B\wedge b_2\in B\}}
\]

%
\subproblem{b} If the first ball is not replaced before the second
ball is drawn, then the sample space is:

\[
\subsolution{b}{\Omega = \{\langle b_1,b_2\rangle|b_1\in B\wedge b_2\in B\wedge b_1\ne b_2\}}
\]

%
\problem{3.3}{4}

Let $\Omega$ be a sample space with $|\Omega|=n$ of equally likely
outcomes and $A\subseteq\Omega$ be an event with $Pr(A)=0.14$ and
$|A|=7$. We want to find out how many outcomes are in $\Omega$, which
is the value $n$.

Because we know that the outcomes are equally likely, then the
probability of event $A$ occurring is:

\[
\multistep{
Pr(A)&=\frac{|A|}{|\Omega|} \\
&=\frac{7}{n}
}
\]

However, we already know that $Pr(A)=0.14$, so we can set those two
expressions equal and solve for $n$:

\[
0.14=\frac{7}{n}
\] \[
n=\frac{7}{0.14}
\] \[
\solution{n=50}
\]

%
\problem{3.3}{8}

Let $A$ be the set of numbers we can draw from:

\[
A=\{1,2,\cdots,100\}
\]

which is divided into two parts:

\[
A_{even}=\{x|x\in A\wedge\exists n(x=2n)\}
\] \[
A_{odd}=\{x|x\in A\wedge\exists n(x=2n+1)\}
\]

Where the sizes of each are:

\[
\begin{array}{cc}
|A_{even}|=50 & |A_{odd}|=50 \\
\end{array}
\]

Let $\Omega$ be the sample space of 3 numbers drawn without
replacement from $A$:

\[
\Omega=\{\langle x_1,x_2,x_3\rangle|x_1,x_2,x_3\in A\wedge 3=|\{x_1,x_2,x_3\}|\}
\]

Note that $|\Omega|$ can be found looking at the number of
combinations without repetition of size $3$ from a set of $100$
distinct objects:

\[
|\Omega|=\binom{100}{3}
\]

Finally, let $B$ be the event that the sum of the numbers from our
outcome is even. In other words:

\[
B=\{\langle x_1,x_2,x_3\rangle|\langle x_1,x_2,x_3\rangle\in\Omega\wedge \exists n(x_1+x_2+x_3=2n)\}
\]

Because we are only interested in whether the sum of the numbers is
even, we can look at some of the properties of even numbers. Let
$E(n)$ denote that $n$ is even . The properties are, for arbitrary
$n$, $m$ and $k$:

\[
\begin{array}{ccc|c}
E(n) & E(m) & E(k) & E(n+m+k) \\
\midrule
0 & 0 & 0 & 0 \\
0 & 0 & 1 & 1 \\
0 & 1 & 0 & 1 \\
0 & 1 & 1 & 0 \\
1 & 0 & 0 & 1 \\
1 & 0 & 1 & 0 \\
1 & 1 & 0 & 0 \\
1 & 1 & 1 & 1 \\
\end{array}
\]

We can further generalize this by looking at the total number of even
numbers in our set of 3:

\[
\begin{array}{c|c}
E(n)+E(m)+E(k) & E(n+m+k) \\
\midrule
0 & 0 \\
1 & 1 \\
2 & 0 \\
3 & 1 \\
\end{array}
\]

Therefore there are only 2 possibilities to cover for finding which
outcomes have an even sum: (i) 1 even number and 2 odd numbers, or
(ii) 3 even numbers. Once we have the total number of possibilities of
each, (iii) we can compute the probability of the event $B$ occuring.

%
\subproblem{i} Because order does not matter for our selection, then
without loss of generality, we can select the two even numbers first
and then the odd number. The number of ways we can select two even
numbers is:

\[
\binom{|A_{even}|}{2}=\binom{50}{2}
\]

And the number of ways to select one odd number is:

\[
\binom{|A_{odd}|}{1}=50
\]

Therefore the total number of possibilities for (i) is:

\[
\binom{50}{2}\cdot 50
\]

%
\subproblem{ii} For (ii), we want the number of possible ways to
select 3 even numbers:

\[
\binom{|A_{even}|}{3}=\binom{50}{3}
\]

%
\subproblem{iii} Using (i) and (ii), the number of outcomes for the
event $B$ is given by:

\[
|B|=\binom{50}{2}\cdot 50 + \binom{50}{3}
\]

Because we assume that each outcome is equally likely, then the
probability of $B$ occurring is given by:

\[
Pr(B)=\frac{|B|}{|\Omega|}
\]

Using our newfound value for $|B|$ and the previous value of
$|\Omega|$, this gives us a probability of:

\[
\solution{Pr(B)=\dfrac{\binom{50}{2}\cdot50+\binom{50}{3}}{\binom{100}{3}}}
\]


\end{multicols*}
\end{document}
