\documentclass{article}
\usepackage{amsmath}
\usepackage{pgfplots}

\newcommand{\norm}[1]{\left|\left|#1\right|\right|}
\setlength\parindent{0pt}
\newcommand{\D}[1]{\mathrm{d}#1}
\newcommand{\partialfrac}[2]{\dfrac{\partial #1}{\partial #2}}

\pgfkeys{/pgfplots/MyAxisStyle/.style={xmin=-10,xmax=10, ymin=-10,ymax=10,height=6cm,width=6cm}}

\begin{document}

\section*{The Final}

Friday, December 5, 2014

\subsection*{Things to Study}

First, focus on the middle parts. Then, try to do all the additional practice problems that the TA gives.

\section*{17.2 Stokes' Theorem}

The idea of this is to generalize Green's Theorem, which has a two dimensional field $\vec{F} = \left<F_1, F_2\right>$ and the integral

\[
\oint_c \vec{F} \cdot \D{\vec{s}} \overset{\text{Green}}{=} \iint_D \left|\begin{array}{cc} \partialfrac{}{x} & \partialfrac{}{y} \\ F_1 & F_2 \end{array}\right| \D{A}
\]

\subsection*{Stokes'}

Given a field $\vec{F}=\left<F_1, F_2, F_3\right>$ and a closed curve $c$, we can calculate

\[
\oint_c \vec{F} \cdot \D{\vec{s}} \overset{\text{Stokes}}{=} \iint_S \text{curl} \vec{F} \cdot \D{\vec{S}}
\]

Where the orientation of the curve follows the right-hand-rule. Suppose there's a sheet of paper which represents the surface, the boundary curve would follow the RHR based on whether you choose the normal to go up, or down from the paper.

You would use these theorems to simplify

\subsection*{Example}

Given a field $\vec{F} = \left<\sin{x^2}, e^{y^2}+x^2, z^4+2x^2\right>$ and a curve given by the points $\left\{(0,0,1), (3,0,0), (0,2,0)\right\}$, calculate

\[
\oint_c \vec{F} \cdot \D{\vec{s}} \overset{\text{Stokes}}{=} \iint_S \text{curl} \vec{F} \cdot \D{\vec{S}}
\]

After evaluating the curl of $\vec{F}$, we must calculate the equation of the surface, which is $S: \frac{1}{3} x + \frac{1}{2} y + z = 1$. This can be parametrized by

\[
z = g(x,y) = 1 - \frac{1}{3} x - \frac{1}{2} y
\]

Then we continue evaluating to get the answer.

\subsection*{Understanding}

Suppose there were two surfaces $S_1$ and $S_2$ which shared the same boundary, how would

\[
\iint_{S_1} \text{curl} \vec{F} \cdot \D{\vec{S_1}}
\]

and

\[
\iint_{S_2} \text{curl} \vec{F} \cdot \D{\vec{S_2}}
\]

compare? They would be equal by Stokes' Theorem, because they can be rewritten as a line integral around the common boundary. Note that this only works because we are dealing with the \textit{curl} of the field, and not the field itself.

Now suppose there's another surface $S_3$ which goes below the two other surfaces, and has the opposite direction, how would its integral compare? It would be negative, because the direction of the boundary curve is reversed.

\subsection*{Knowing When to Use it}

What would let us know that we need to use Stokes' Theorem?

One example would be: Find $\iint_S \text{curl} \vec{F} \cdot \D{\vec{S}}$ where $\vec{F} = \left<x, x y, x y z\right>$.

Rather than evaluate the curl, and then do the surface integral, we can just use Stokes' Theorem to get a simpler line integral of just $\vec{F}$.

\subsection*{Example}

Calculate the surface integral

\[
\iint_S \text{curl} \vec{F} \cdot \D{\vec{S}}
\]

where $\vec{F} = \left<x z, y z, x y\right>$ over the surface $S$, which can be given by a semi-dome with radius $2$ which coincides with a cylinder of radius $1$.

We apply Stokes' Thoerem to get:

\[
\iint_S \text{curl} \vec{F} \cdot \D{\vec{S}} \begin{array}{rl}
  = & \oint_c \vec{F} \cdot \D{\vec{S}} \\
  = & \oint_c x z \D{x} + y z \D{y} + x y \D{z}
\end{array}
\]

Where we take $c$ to be the boundary curve going around the semi-dome.

We can parametrize $c$ by

\[
c: \begin{cases}
  x = \cos{\theta} \\
  y = \sin{\theta} \\
  z = \sqrt{3}
\end{cases}
\]

And substitute this into the equation to get:

\[
= \int_0^{2\pi} \cos{\theta} \sqrt{3} (\sin{\theta}) + \sin{\theta}\sqrt{3} \cos{\theta} + \cos{\theta}\sin{\theta} \D{\theta}
\]

Which we could simplify to get the final answer.

\subsection*{Example}

Given a field $\vec{F} = \left<y + z, \sin{(x y)}, e^{xyz}\right>$ over a surface $S$ given by: an arbitrary surface left of a unit circle in the $Y Z$ plane such that the normal vector points outwards of the surface. Calculate

\[
\iint_S \text{curl} \vec{F} \cdot \D{\vec{S}}
\]

By Stokes' Thoerem, this can be rewritten as

\[
\iint_S \text{curl} \vec{F} \cdot \D{\vec{S}}
=
\oint_c \vec{F} \cdot \D{\vec{s}}
\]

\[
= \oint_c (y + z) \D{x} + \sin(x y) \D{y} + e^{x y z} \D{z}
\]

The parametrized form of the curve $c$ can be given by

\[
c: \begin{cases} x = \cos{\theta} \\ y = 0 \\ z = \sin{\theta} \end{cases}
\]

Therefore we can write the integral as

\[
= \int_0^{2\pi} (0 + \sin{\theta})(-\sin{\theta}) + \sin{0} + \cos{\theta} \D{\theta}
\]

\[
= \int_0^{2\pi} (- \sin^2 \theta + \cos{\theta}) \D{\theta}
\]

\[
= \int_0^{2\pi} (\frac{1 - \cos{2 \theta}}{2} + \cos{\theta}) \D{\theta}
\]

\[
= \pi
\]

Followup: What if the surface pointed the opposite direction, what would the surface integral be?

\[
- \pi
\]

\end{document}
